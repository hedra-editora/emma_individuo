\chapter{O indivíduo, a sociedade e o Estado}

\textsc{A dúvida} reina no espírito dos homens, pois nossa civilização treme em
suas bases. As instituições atuais não mais inspiram confiança e os
mais inteligentes compreendem que a industrialização capitalista vai
contra os próprios objetivos que diz perseguir.

O mundo não sabe como sair disso. O parlamentarismo e a democracia
periclitam e alguns creem encontrar salvação optando pelo fascismo ou
outras formas de governos “fortes”.

Do combate ideológico mundial sairão soluções para os problemas sociais
urgentes que se apresentam atualmente: crises econômicas, desemprego,
guerra, desarmamento, relações internacionais etc. Ora, é dessas
soluções que dependem o bem"-estar do indivíduo e o destino da
sociedade humana.

O Estado, o governo com suas funções e seus poderes, torna"-se, assim,
o centro de interesse do homem que raciocina. Os desenvolvimentos
políticos que ocorreram em todas as nações civilizadas levam"-nos a
fazer essas perguntas: desejamos um governo forte? Devemos preferir a
democracia e o parlamentarismo? O fascismo, sob uma ou outra forma, a
ditadura, quer seja monárquica, burguesa ou do proletariado,
oferecem soluções aos males ou às dificuldades que atormentam nossa
sociedade?

Em outros termos, conseguiremos apagar as taras da democracia com a
ajuda de um sistema ainda mais democrático, ou devemos cortar o
nó górdio do governo popular com a espada da ditadura?

Minha resposta é: nem um, nem outro. Sou contra a ditadura e o fascismo,
e oponho"-me aos regimes parlamentares e às pretensas democracias
populares.

É com razão que se falou do nazismo como de um ataque contra a
civilização. A mesma coisa se poderia dizer de todas as formas de
ditadura, opressão e coerção, pois o que é a civilização? Todo
progresso foi essencialmente marcado pela extensão das liberdades do
indivíduo em detrimento da autoridade exterior, tanto no que concerne à
sua existência física quanto à política ou econômica. No mundo físico, o
homem progrediu até controlar as forças da natureza e utilizá"-las em
seu próprio proveito. O homem primitivo realiza seus primeiros passos
na estrada do progresso quando logra produzir fogo, triunfando assim
sobre o próprio homem, e reter vento e captar água.

Que papel a autoridade ou o governo desempenharam nesse esforço de
melhoria, invenção e descoberta? Nenhum, ou melhor, nenhum positivo. É
sempre o indivíduo quem realiza o milagre, geralmente a despeito das
proibições, das perseguições e da intervenção da autoridade, tanto
humana quanto divina.

Da mesma forma, no campo político, o progresso consiste em afastar"-se
cada vez mais da autoridade do chefe de tribo, de clã, do príncipe e do
rei, do governo e do Estado. Economicamente, o progresso significa mais
bem"-estar para um número de pessoas incessantemente crescente. E,
culturalmente, ele é o resultado de tudo o que se realiza algures:
independência política, intelectual e psíquica cada vez maior.

Nessa perspectiva, os problemas de relação entre o homem e o Estado
revestem uma significação completamente nova. Não é mais questão de
saber se a ditadura é preferível à democracia, se o fascismo italiano é
superior ou não ao hitlerismo. Uma questão muito mais vital se nos
apresenta: o governo político, o Estado, é proveitoso à humanidade?
Qual é sua influência sobre o indivíduo?

O indivíduo é a verdadeira realidade da vida, um universo em si. Ele não
existe em função do Estado, ou dessa abstração denominada “sociedade”
ou “nação”, que não é senão um ajuntamento de indivíduos. O homem
sempre foi e é --- necessariamente --- a única fonte, o único motor de
evolução e progresso. A civilização é o resultado de um combate
contínuo do indivíduo ou dos grupamentos de indivíduos contra o Estado,
e até mesmo contra a “sociedade”, quer dizer, contra a maioria
hipnotizada pelo Estado e submetida a seu culto. As maiores batalhas já
travadas pelo homem foram contra obstáculos e prejuízos artificiais
que ele próprio se impôs e que paralisam seu desenvolvimento. O
pensamento humano sempre foi falseado pelas tradições, pelos costumes,
pela educação enganadora e iníqua, dispensada para servir os interesses
daqueles que detêm o poder e gozam de privilégios; ou seja, pelo Estado
e pelas classes proprietárias. Esse conflito incessante dominou a
história da humanidade.

Podemos dizer que a individualidade é a consciência do indivíduo de ser
o que é, e de viver essa diferença. É um aspecto inerente a todo ser
humano e um fator de desenvolvimento. O Estado e as instituições
sociais fazem"-se e desfazem"-se, enquanto a individualidade
permanece e persiste. A própria essência da individualidade é a
expressão, o sentido da dignidade e da independência --- eis seu terreno
de predileção. A individualidade não é esse conjunto de reflexos
impessoais e maquinais que o Estado considera como um “indivíduo”. O
indivíduo não é apenas o resultado da hereditariedade e do meio, da
causa e do efeito. É isso e muito mais. O homem vivo não pode ser
definido: ele é fonte de toda a vida e de todos os valores; ele não é
uma parte disso ou daquilo: é um todo, um todo individual, um todo que
evolui e se desenvolve, mas que permanece, contudo, um todo constante.

A individualidade assim descrita nada tem em comum com as diversas
concepções do individualismo e, sobretudo, com aquele que denominarei
“individualismo de direita, à americana”, que é tão"-somente uma
tentativa disfarçada de coagir e vencer o indivíduo em sua
singularidade. Esse pretenso individualismo, que sugere fórmulas
como “livre empresa”, “\textit{american way of life}”, arrivismo e sociedade
liberal, é o \textit{laisser"-faire} econômico e social: a exploração das
massas pelas classes dominantes com a ajuda da velhacaria legal; a
degradação espiritual e o doutrinamento sistemático do espírito servil,
processo conhecido sob o nome de “educação”. Essa forma de
“individualismo” corrompido e viciado, verdadeira camisa de força da
individualidade, reduz a vida a uma corrida degradante aos bens
materiais, ao prestígio social; sua sabedoria suprema exprime"-se numa
frase: “Cada um por si e maldito seja o último”.

Inevitavelmente, o “individualismo” de direita desemboca na escravidão
moderna, nas distinções sociais aberrantes, e conduz milhões de pessoas à
sopa dos pobres. Esse “individualismo” em questão é o dos senhores,
enquanto que o povo é arregimentado numa casta de escravos para servir
a um punhado de “super"-homens” egocêntricos. Os Estados Unidos são,
sem dúvida, o melhor exemplo dessa forma de individualismo, em nome do
qual a tirania política e a opressão social são elevadas à posição de
virtudes, enquanto que a menor aspiração, a menor tentativa de vida mais
livre e mais digna será imediatamente considerada como
antiamericanismo intolerável e condenada, sempre em nome desse mesmo
individualismo.

Houve um tempo em que o Estado não existia. O homem vivia em condições
naturais, sem Estado nem governo organizado. As pessoas estavam
agrupadas em pequenas comunidades de algumas famílias, cultivando o
solo e entregando"-se à arte e ao artesanato. O indivíduo,
posteriormente a família, era a célula de base da vida social; cada um
era livre e igual a seu vizinho. A sociedade humana dessa época não
era um Estado, mas uma associação voluntária onde todos se beneficiavam
da proteção de todos. Os mais velhos e os membros mais experientes do
grupo eram os guias e os conselheiros. Eles ajudavam a resolver os
problemas vitais, o que não significa governar e dominar o indivíduo.
Foi só mais tarde que se viu surgir governo político e Estado,
consequências do desejo dos mais fortes de tirar vantagens dos mais
fracos, de alguns contra a maioria. O Estado eclesiástico ou secular
serviu, então, para dar uma aparência de legalidade e de direito aos
danos causados por alguns à maioria. Essa aparência de direito era o
meio mais cômodo de governar o povo, pois um governo não pode existir
sem o consentimento do povo, consentimento verdadeiro, tácito ou
simulado. O constitucionalismo e a democracia são as formas modernas
desse pretenso consentimento, inoculado pelo que se chama “educação”,
autêntico doutrinamento público e privado.

O povo consente porque é persuadido da necessidade da autoridade;
inculcam nele a ideia de que o homem é mau, virulento e demasiado
incompetente para saber o que é bom para ele. É a ideia fundamental de
todo governo e de toda opressão. Deus e o Estado só existem e são
sustentados por causa dessa doutrina.

No entanto, o Estado não é mais que um nome, uma abstração. Assim como
outras concepções do mesmo tipo --- nação, raça, humanidade ---, ele não tem
realidade orgânica. Denominar o Estado de organismo é uma tendência
doentia de fazer de uma palavra um fetiche.

A palavra Estado designa o aparelho legislativo e administrativo que
trata de certos negócios humanos --- e, na maioria das vezes, trata mal. Ele nada
contém de sagrado, de santo ou de misterioso. O Estado não tem
consciência, não é encarregado de uma missão moral, não mais do que uma
companhia comercial seria encarregada de explorar uma mina de carvão ou
uma ferrovia.

O Estado não tem mais realidade do que os deuses ou os diabos. São
apenas reflexos, criações do espírito humano, pois o homem, o indivíduo,
é a única realidade. O Estado é só a sombra do homem, a sombra de seu
obscurantismo, de sua ignorância e de seu medo.

A vida começa e acaba com o homem, o indivíduo. Sem ele não há raça,
humanidade, Estado. Nem mesmo sociedade. É o indivíduo que vive,
respira e sofre. Desenvolve"-se e progride lutando continuamente
contra o fetichismo que ele nutre com respeito às suas próprias
invenções e, em particular, ao Estado.

A autoridade religiosa edificou a vida política à imagem daquela da
Igreja. A autoridade do Estado, os “direitos” dos governantes vinham do
alto; o poder, como a fé, era de origem divina. Os filósofos escreveram
espessos volumes provando a santidade do Estado, às vezes chegando, 
inclusive, a conceder"-lhe a infalibilidade. Alguns desses filósofos
disseminaram a opinião demente de que o Estado é “supra"-humano”,
realidade suprema, “o absoluto”.

A pesquisa era uma blasfêmia, a servidão a mais elevada das virtudes.
Graças a tais princípios, chegou"-se a considerar certas ideias como
evidências sagradas, não porque sua verdade tivesse sido demonstrada,
mas por serem repetidas continuamente.

Os progressos da civilização são essencialmente caracterizados por um
questionamento do “divino” e do “mistério”, do pretenso sagrado e da
“verdade eterna”; é a eliminação gradual do abstrato, ao qual se
substitui pouco a pouco o concreto. Quer dizer, os fatos precedem ao
imaginário, o saber à ignorância, a luz à obscuridade.

O lento e difícil processo de liberação do indivíduo não se realizou com
a ajuda do Estado. Ao contrário, foi empreendendo um combate
ininterrupto e sangrento que a humanidade conquistou o pouco de
liberdade e independência de que dispõe, arrancado das mãos dos reis,
dos czares e dos governos.

A personagem heroica desse longo calvário é o Homem. Sozinho ou unido 
a outros, é sempre o indivíduo que sofre e combate as opressões de toda
espécie, as potências que o subjugam e degradam.

Mais ainda, o espírito do homem, do indivíduo, é o primeiro a
rebelar"-se contra a injustiça e o aviltamento; o primeiro a conceber
a ideia de resistência às condições nas quais ele se debate. O
indivíduo é o gerador do pensamento liberador, assim como do ato
liberador.

E isso não diz respeito apenas ao combate político, mas a toda a gama
dos esforços humanos, em todos os tempos e sob todos os céus. É sempre
o indivíduo, o homem com sua força de caráter e sua vontade de
liberdade, que abre o caminho do progresso humano e dá os primeiros
passos rumo a um mundo melhor e mais livre; nas ciências, na filosofia,
no campo das artes bem como no da indústria, seu gênio eleva"-se em
direção aos cumes, concebe “o impossível”, materializa seu sonho e
comunica seu entusiasmo aos outros, que, por sua vez, se engajam na
peleja. No campo social, o profeta, o visionário, o idealista que sonha
com um mundo segundo seu coração ilumina o caminho das grandes
realizações.

O Estado, o governo, qualquer que seja sua forma, característica ou
tendência, quer seja autoritário ou constitucional, monárquico ou
republicano, fascista, nazista ou bolchevique, é, por sua própria
natureza, conservador, estático, intolerante e oposto à mudança. Se às
vezes evolui de maneira positiva, é que, submetido a pressões fortes o
bastante, é obrigado a operar a mudança que se lhe impõe,
pacificamente às vezes, brutalmente na maioria das vezes, quer dizer,
pelos meios revolucionários. Além do mais, o conservadorismo inerente à
autoridade sob todas as suas formas torna"-se inevitavelmente
reacionário. Duas razões para isso: a primeira, é que é natural para um
governo não apenas conservar o poder que detém, como também
reforçá"-lo, ampliá"-lo e perpetuá"-lo no interior e no exterior de
suas fronteiras. Quanto mais forte a autoridade, quanto maior o
Estado e seus poderes, mais intolerável será para ele uma autoridade
similar ou um poder político paralelo. A psicologia governamental impõe
uma influência e um prestígio em constante crescimento, nacional e
internacionalmente, e o governo agarrará todas as oportunidades para
ampliá"-los. Os interesses financeiros e comerciais que dão
sustentação ao governo que os representa e os serve motivam essa
tendência. A razão de ser fundamental de todos os governos, para a qual
os historiadores dos tempos passados fechavam voluntariamente os olhos, é
hoje tão evidente que os próprios professores não podem mais
ignorá"-la.

O outro fator que obriga os governos a um conservadorismo cada vez mais
reacionário é a desconfiança inerente que eles têm do indivíduo, o
temor da individualidade. Nosso sistema político e social não tolera o
indivíduo com sua constante necessidade de inovação. É, portanto, em
estado de “legítima defesa” que o governo oprime, persegue, pune e às
vezes mata o indivíduo, sendo ajudado por todas as instituições cujo
objetivo é preservar a ordem existente. Ele recorre a todas as formas
de violência e é apoiado pelo sentimento de “indignação moral” da
maioria contra o herético, o dissidente social, o rebelde político,
maioria essa em quem se inculcou desde séculos o culto do Estado,
educada na disciplina, na obediência e na submissão à autoridade e no
respeito a ela, cujo eco se faz ouvir em casa, na escola, na igreja e
na imprensa.

A melhor muralha da autoridade é a uniformidade; a menor divergência de
opinião torna"-se, então, o pior dos crimes. A mecanização em grande
escala da sociedade atual acarreta um acréscimo de uniformização.
Encontramo"-la presente em toda parte: nos hábitos, nos gostos, na
escolha das vestes, nos pensamentos, nas ideias. Contudo, é no que
convimos denominar “opinião pública” que encontramos seu concentrado
mais aflitivo. Bem poucos têm a coragem de opor"-se a ela. Aquele que
recusa submeter"-se é de pronto “bizarro”, “diferente”, “suspeito”,
fautor de desordens no seio do universo estagnante e confortável da
vida moderna.

Sem dúvida nenhuma, mais que a autoridade constituída, é a
uniformidade social que prostra o indivíduo. O fato de ele ser “único”,
“diferente”, separa"-o e torna"-o estrangeiro em seu país e, às vezes, até mesmo
em seu lar, mais que o expatriado cujas opiniões geralmente
coincidem com aquelas dos “autóctones”. Para um ser humano sensível,
não é suficiente encontrar"-se em seu país de origem para se sentir em
casa, a despeito de isso supor tradições, impressões e recordações de
infância, todas as coisas que nos são caras. É muito mais essencial
encontrar uma certa atmosfera de pertencimento, ter consciência de
“fazer corpo” com as pessoas e o meio, para sentir"-se em casa, quer
se trate de relações familiares, de relações de vizinhança ou, então,
daquelas que mantemos na região mais vasta comumente denominada país. O
indivíduo capaz de interessar"-se pelo mundo inteiro jamais se sente
tão isolado, tão incapaz de partilhar os sentimentos de seu círculo
do que quando se encontra em seu país de origem.

Antes da guerra, o indivíduo tinha ao menos a possibilidade de escapar à
prostração nacional e familiar. O mundo parecia aberto a suas buscas,
a seus ímpetos, a suas necessidades. Hoje, o mundo é uma prisão e a
vida uma pena de prisão perpétua a purgar na solidão. Isso é ainda mais
verdadeiro desde o evento da ditadura, tanto de direita quanto de
esquerda.

Friedrich Nietzsche qualificava o Estado de monstro frio. Como
qualificaria a fera hedionda oculta sob o casaco da ditadura moderna?
Não que o Estado tenha alguma vez alocado um campo de ação muito grande 
ao indivíduo, mas os campeões da nova ideologia estatal não lhe
concedem nem sequer o pouco do qual dispunha. “O indivíduo não é nada”,
clamam eles. Só a coletividade conta. Não querem nada menos que a
submissão total do indivíduo para satisfazer o apetite insaciável de
seu novo deus.

Curiosamente, é no seio da \textit{intelligentsia} britânica e americana que
encontramos os mais ferozes advogados da nova causa. No momento,
ei"-los arrebatados pela “ditadura do proletariado”. Apenas em teoria,
é claro. Na prática, eles preferem ainda se beneficiar das poucas
liberdades que lhes são concedidas em seus respectivos países. Eles vão
à Rússia para curtas visitas, ou enquanto representantes da
“revolução”; contudo, eles se sentem, apesar de tudo, mais seguros em
seus países.

Por sinal, talvez não seja apenas a falta de coragem que retém esses
bravos britânicos e esses americanos em seus próprios países. Eles
sentem, talvez inconscientemente, que o indivíduo permanece o fato
fundamental de toda associação humana e que, por mais oprimido e
perseguido que seja, é ele que vencerá a longo prazo.

O “gênio do homem”, que não é outra coisa senão uma maneira diferente de
qualificar a personalidade e sua individualidade, traça um caminho
através do labirinto das doutrinas, através dos muros espessos da
tradição e dos costumes, desafiando os tabus, desafiando a autoridade,
afrontando o ultraje e o cadafalso --- para, às vezes, ser como profeta e
mártir pelas gerações seguintes. Sem esse “gênio do homem”, sem sua
individualidade inerente e inalterável, ainda estaríamos a percorrer as
florestas primitivas.

Piotr Kropotkin mostrou os resultados fantásticos que podemos esperar
quando essa força que é a individualidade humana trabalha em cooperação
com outras. O grande erudito e pensador anarquista disfarçou, desse
modo, biológica e sociologicamente, a insuficiência da teoria
darwiniana no que se refere à luta pela sobrevivência. Em sua
extraordinária obra \textit{O apoio mútuo}, Kropotkin mostra que no reino
animal, tanto quanto na sociedade humana, a cooperação --- por oposição
às lutas intestinas --- opera no sentido da sobrevivência e da evolução
das espécies. Ele demonstra que, ao contrário do Estado devastador e
onipotente, só o apoio mútuo e a cooperação voluntária constituem os
princípios básicos de uma vida livre, fundada sobre o indivíduo e a
associação.

No presente momento, o indivíduo é apenas um peão no tabuleiro da
ditadura e nas mãos dos fanáticos do “individualismo à americana”. Os
primeiros buscam uma desculpa no fato de que estão à procura de um novo
objetivo. Os segundos nem sequer pretendem ser inovadores. De fato, os
zeladores dessa “filosofia” reacionária nada aprenderam e nada
esqueceram. Contentam"-se com zelar para que persista a ideia de um
combate brutal pela sobrevivência, ainda que a necessidade desse
combate tenha desaparecido por completo. É evidente que este se
perpetua justamente porque é inútil. A pretensa superprodução não é a
prova disso? A crise econômica mundial não é a eloquente demonstração
de que esse combate pela sobrevivência só deve sua manutenção à
cegueira dos adeptos do “cada um por si”, ao risco de assistir à
autodestruição do sistema?

Uma das características insensatas dessa situação é a ausência de
relação entre o produtor e o objeto produzido. O operário médio não tem
nenhum contato profundo com a indústria que o emprega; permanece
estranho ao processo de produção, do qual é apenas uma engrenagem. E,
como tal, é substituível a qualquer momento por outros seres humanos
igualmente despersonalizados.

O trabalhador que exerce uma profissão intelectual ou liberal, conquanto
tenha a vaga impressão de ser independente, não é mais bem favorecido.
Ele também não teve grande escolha nem mais possibilidade de encontrar
seu próprio caminho em seu ramo de atividade do que seu vizinho
trabalhador manual. Geralmente são considerações materiais, um desejo
de prestígio social, que determinam a orientação do intelectual. A isso
vem acrescentar"-se a tendência de abraçar a carreira paterna para
tornar"-se professor, engenheiro, assumir o escritório de advocacia ou
o consultório médico etc., pois a tradição familiar e a rotina não
exigem grandes esforços nem personalidade. Em consequência, a maioria
das pessoas insere"-se mal no mundo do trabalho. As massas prosseguem
com grande dificuldade seu caminho, sem procurar ir mais longe, antes de
tudo porque suas faculdades estão entorpecidas por uma vida de trabalho
e rotina; e, depois, eles precisam ganhar a vida. Encontramos a mesma
trama nos círculos políticos, talvez com mais força. Ali não é criado
nenhum espaço para a livre escolha, para o pensamento ou para a
atividade independentes. Só encontramos marionetes boas apenas para
votar e pagar os impostos.

Os interesses do Estado e os do indivíduo são fundamentalmente
antagônicos. O Estado e as instituições políticas e econômicas que ele
instaurou não podem sobreviver senão modelando o indivíduo, a fim de que
ele sirva a seus interesses; eles o condicionam no respeito à lei e à
ordem, ensinando"-lhe obediência, submissão e fé absoluta na sabedoria
e na justiça do governo; exigem antes de tudo o total sacrifício do
indivíduo quando o Estado precisa dele, em caso de guerra, por exemplo.
O Estado considera seus interesses como superiores àqueles da religião e de
Deus. Pune até em seus escrúpulos religiosos ou morais o indivíduo que
se recusa a combater seu semelhante, porque não há individualidade sem
liberdade, e esta é a maior ameaça que pode pesar sobre a
autoridade.

O combate que o indivíduo sustenta em condições tão desfavoráveis --- com
frequência ao preço de sua vida --- é ainda mais difícil porque seus
adversários não estão interessados em saber se ele está certo ou
errado. Não são nem o valor nem a utilidade de seu pensamento ou de
sua ação que erguem contra ele as forças do Estado e da “opinião
pública”. As perseguições contra o inovador, o dissidente, o
contestador, sempre foram causadas pelo temor de que a infalibilidade
da autoridade constituída seja questionada e seu poder solapado.

O homem só conhecerá a verdadeira liberdade, individual e coletiva,
quando se libertar da autoridade e de sua fé nela. A evolução humana
nada mais é que uma penosa caminhada nessa direção. O desenvolvimento
não é em si nem a invenção, nem a técnica. Correr a 150 quilômetros por hora não
é uma prova de civilização. É pelo indivíduo, autêntico modelo social,
que se mede nosso grau de civilização; é por suas faculdades individuais,
pelas possibilidades de ele ser livremente o que é, de desenvolver"-se
e progredir sem intervenção da autoridade coercitiva e onipotente.

Socialmente falando, a civilização e a cultura devem ser medidas pelo
grau de liberdade e pelas possibilidades econômicas de que goza o
indivíduo; devem ser igualmente medidas pela unidade e pela cooperação
social e internacional, sem restrição legal ou qualquer outro obstáculo
artificial; pela ausência de castas privilegiadas; por uma vontade de
liberdade e dignidade humanas. Em resumo, o critério de civilização é o
grau de emancipação real do indivíduo.

O absolutismo político foi abolido porque o homem percebeu, no decorrer
de séculos, que o poder absoluto é um mal destruidor. Mas o mesmo vale
para todos os poderes, quer seja aquele dos privilégios, do dinheiro,
do padre, do político ou da pretensa democracia. Pouco importa a cor
que o caráter específico da coerção reveste: o negro do fascismo, o
pardo do nazismo ou o vermelho pretensioso do bolchevismo. O poder
corrompe e degrada tanto o senhor quanto o escravo, esteja esse
poder nas mãos do autocrata, do parlamento ou do soviete. Mas o poder
de uma classe é ainda mais pernicioso do que o do ditador, e nada é
mais terrível do que a tirania da maioria.

No transcurso do longo processo histórico, o homem aprendeu que a
divisão e a luta conduzem à destruição e que a unidade e a cooperação
fazem progredir sua causa, multiplicam suas forças e favorecem seu
bem"-estar. O espírito governamental trabalha desde sempre contra a
aplicação social dessa lição fundamental, exceto quando o Estado tem
interesse nela. Os princípios conservadores e anti"-sociais do Estado
e da classe privilegiada que o sustenta são responsáveis por todos os
conflitos que colocam os homens uns contra os outros. São cada vez mais
numerosos aqueles que começam a ver claro, sob a superfície da ordem
estabelecida. O indivíduo deixa"-se cegar cada vez menos pelo brilho enganador
dos princípios estatais e pelos “benefícios” do “individualismo”
preconizado pelas sociedades ditas liberais. Ele se esforça para
alcançar as perspectivas mais amplas das relações humanas que só a
liberdade proporciona. Isso porque a verdadeira liberdade não é uma
simples pilha de papéis intitulada “constituição”, “direito legal” ou
“lei”. Também não é uma abstração derivada dessa outra irrealidade
chamada “Estado”. Não é o ato negativo de ser libertado de algo, pois
essa liberdade é apenas a liberdade de morrer de fome. A verdadeira
liberdade é positiva; é a liberdade rumo a algo, a liberdade de ser, de
fazer, e os meios empregados para isso.

Não pode se tratar, então, de uma doação, mas de um direito natural do
homem, de todos os seres humanos. Esse direito não pode ser concedido
ou conferido por nenhuma lei, nenhum governo. A necessidade, o desejo
ardente dele se faz sentir em todos os indivíduos. A desobediência a
todas as formas de coerção é sua expressão instintiva. Rebelião e
revolução são tentativas mais ou menos conscientes de conquistá"-lo.
Essas manifestações individuais e sociais são as expressões
fundamentais dos valores humanos. Para nutrir esses valores, a
comunidade deve compreender que seu apoio mais sólido, mais durável, é
o indivíduo.

No campo religioso, bem como no campo político, fala-se de abstrações
acreditando tratar"-se de realidade. No entanto, quando se vem tratar
verdadeiramente de coisas concretas, parece que a maioria das pessoas é
incapaz de encontrar um interesse vital por isso. Talvez seja porque a
realidade é por demais prosaica, demasiado fria para despertar a 
alma humana. Só os assuntos diferentes, pouco comuns, provocam o
entusiasmo; quer dizer, o Ideal que faz surgir a centelha da imaginação
e do coração humano. É preciso algum ideal para tirar o homem da
inércia e da monotonia de sua existência e transformar o vil escravo em
personagem heroica.

É aqui que intervém, evidentemente, o oponente marxista cujo marxismo
ultrapassa, por sinal, o do próprio Marx. Para ele, o homem é apenas
um boneco nas mãos dessa onipotência metafísica denominada determinismo
econômico, vulgarmente conhecido como luta de classes. A vontade do
homem, individual e coletiva, sua vida psíquica, sua orientação
intelectual, tudo isso conta muito pouco para nosso marxista, e em nada
afeta suas concepções da história humana.

Nenhum estudante inteligente negaria a importância do fator econômico no
progresso social e no desenvolvimento da humanidade. Todavia, só um
espírito obtuso e obstinadamente doutrinário se recusará a ver o
importante papel da ideia, enquanto concepção da imaginação e resultado
das aspirações do homem.

Seria vão e sem interesse tentar comparar dois fatores da história
humana. Nenhum fator pode ser considerado como o único fator decisivo
do conjunto dos comportamentos individuais e sociais. Avançamos muito
pouco em psicologia humana, e talvez nunca seremos muito avançados para
pesar e medir os valores relativos de tal ou qual fator determinante do
comportamento humano. Formular tais dogmas, em suas conotações sociais,
é puro fanatismo; no entanto, veremos uma certa utilidade no fato de
que essa tentativa de interpretação político"-econômica da história
prova a persistência da vontade humana e refuta os argumentos dos
marxistas.

Felizmente, alguns marxistas começam a ver que seu credo não é toda a
verdade; afinal de contas, Marx era um ser humano, demasiado humano
para ser infalível. Atualmente, as aplicações práticas do determinismo econômico na
Rússia abrem os olhos dos marxistas mais inteligentes.
Podemos ver, com efeito, ajustamentos se operando no nível dos
princípios marxistas nas fileiras socialistas e nas
fileiras comunistas dos países europeus, inclusive. Eles lentamente compreendem
que sua teoria não levou muito em consideração o elemento humano --- 
\textit{der Mensch}, como salienta um jornal socialista.

Por mais importante que seja, o fator econômico não é, contudo,
suficiente para determinar sozinho o destino de uma sociedade. A
regeneração da humanidade não se realizará sem a aspiração, a força
energética de um ideal.

Esse ideal, para mim, é a Anarquia, que com toda certeza nada tem a ver
com a interpretação errônea que os adoradores do Estado e da autoridade
se associam para disseminar. Essa filosofia lança as bases de uma nova
ordem social, fundada nas energias liberadas do indivíduo e na
associação voluntária dos indivíduos liberados.

De todas as teorias sociais, a Anarquia é a única a proclamar que a
sociedade deve estar a serviço do homem e não o homem a serviço da
sociedade. O único objetivo legítimo da sociedade é prover as
necessidades do indivíduo e ajudá"-lo a realizar seus desejos. É só
nesse caso que ela se justifica e participa dos progressos da
civilização e da cultura. Sei que os representantes dos partidos
políticos e os homens que lutam com selvageria pelo poder me
estigmatizarão com a marca de anacronismo incorrigível. Pois bem,
aceito com alegria essa acusação. É para mim um conforto saber que
falta consistência à sua histeria e que seus louvores são sempre
temporários.

O homem aspira libertar"-se de todas as formas de autoridade e poder,
e não são os discursos estrepitosos que o impedirão de romper para
sempre seus grilhões. Os esforços do homem devem prosseguir --- e eles
prosseguirão.

\chapter{A preparação militar nos conduz direto ao massacre universal}

\hedramarkboth{A preparação militar nos conduz\ldots{}}{Emma Goldman}

\textsc{Desde o início} da conflagração europeia, a humanidade foi quase
inteiramente anestesiada pela mortífera loucura do belicismo,
embriagada pelos vapores deletérios de um clorofórmio impregnado de
sangue, que obscureceu sua visão e paralisou seu coração. Com efeito, à
exceção de algumas tribos selvagens que não conhecem nem a religião
cristã, nem o amor fraternal, nem os \textit{dreadnaughts},\footnote{ 
Navios de guerra. [N.~do T.]}
os submarinos, as fábricas de munições e os empréstimos
de guerra, o resto da humanidade está mergulhado em uma terrível
narcose. O espírito humano parece interessar"-se apenas por uma coisa:
especular sobre assassinato. Toda a nossa civilização, toda a nossa
cultura está concentrada na louca demanda de armas de destruição, se
possível as mais aperfeiçoadas.

“Munições! Munições! Ó Senhor, tu que reinas sobre a terra e nos céus,
tu, ó Deus do amor, da piedade e da justiça, concede"-nos bastante
munição para destruir nosso inimigo!” Tal é a oração que ascende ao céu
cristão todos os dias. O gado, quando fica assustado pelo fogo,
lança"-se nas chamas. Os povos europeus agem do mesmo modo:
precipitam"-se nas chamas devoradoras da guerra, entrematando"-se.
Quanto à América, levada à beira do abismo por políticos
inescrupulosos, demagogos ruidosos e ávidos tubarões militares, ela se
prepara para um idêntico destino funesto.

Ante esse desastre que se aproxima, cabe aos homens e às mulheres que
ainda não estão inebriados pela loucura guerreira elevar a voz,
protestar, atrair a atenção da população para os crimes e as
atrocidades que serão perpetrados contra eles.

A América é essencialmente um 
\textit{melting"-pot}.\footnote{ Cadinho; mistura.} 
Nesse país, nenhum grupo nacional pode gabar"-se de
pertencer a uma raça pura e superior, ser detentor de uma missão
histórica particular ou de uma cultura mais espiritual. Entretanto, os
chauvinistas e os especuladores belicistas não param de tartamudear.
\textit{slogans} sentimentalistas de nacionalismo hipócrita: “A América aos
americanos”, “A América de início, antes de tudo e sempre”. Esses
\textit{slogans} são em toda parte populares. A crer neles, para salvar a
América, seria preciso que todo mundo seguisse imediatamente uma
formação militar. Um milhão de dólares arrancados do suor e do
sangue do povo serão gastos na compra de encouraçados e submarinos
para o exército e a marinha, e isso tudo para proteger essa preciosa
América.

Esses discursos repletos de \textit{pathos} dissimulam o fato de que a América
que será protegida por uma enorme força militar não será a América do
povo, mas aquela dos privilegiados; da classe que rouba e explora as
massas e controla a sua vida, do berço ao túmulo. É patético que tão
poucas pessoas deem"-se conta de que a preparação militar nunca conduz
à paz, mas leva direto ao massacre universal.
\asterisc 

Com os métodos e a astúcia empregados pelos diplomatas conspiradores e
as corjas dirigentes do exército alemão para impor o militarismo 
prussiano às massas de seu país, os círculos belicistas americanos,
ajudados pelos Roosevelt, os Garrison e os Daniels, agora aliados aos
Wilson, não medem esforços para esmagar o povo americano sob as botas
do militarismo. Se lograrem êxito, lançarão a América na tempestade de
sangue e de lágrimas que já devasta a Europa.

Há quarenta anos, a Alemanha entoou os mesmos discursos: “A Alemanha
acima de tudo”, “A Alemanha para os alemães”, “A Alemanha de início,
antes de tudo e sempre”, “Queremos a paz, e é por isso que devemos nos
preparar para a guerra”, “Só uma nação bem armada e perfeitamente
preparada pode manter a paz, exigir o respeito e estar certa de
conservar sua integridade nacional”. E a Alemanha continuou a
preparar"-se para a guerra, obrigando, assim, as outras nações a
imitá"-la. A terrível guerra europeia atual é só a consequência última
das predicações desse Evangelho de cabeça de hidra: a preparação
militar.

Desde o começo dessa guerra mundial, quilômetros de papel e oceanos de
tinta foram utilizados para provar a barbárie, a crueldade, a opressão
do militarismo prussiano. Em uníssono, conservadores e socialistas
apoiam os Aliados por uma única razão: esmagar esse militarismo que
impede, segundo eles, toda paz e todo progresso na Europa. A América
enriqueceu"-se fabricando toneladas de munições e emprestando dinheiro
aos Aliados para ajudá"-los a esmagar os prussianos. E, agora, os
mesmos \textit{slogans} são ouvidos na América. E se eles traduzem"-se por uma
mobilização nacional, eles criarão um militarismo americano bem mais
terrível do que o militarismo alemão ou prussiano. Por quê? Porque em
nenhum lugar do mundo o capitalismo é tão desavergonhadamente ávido
quanto nos Estados Unidos, e em nenhum lugar o Estado está tão disposto
a ajoelhar"-se aos pés do Capital.

Como uma epidemia, uma onda de loucura ganha o país; o germe mortal do
militarismo contamina os espíritos mais lúcidos e os corações mais
bravos. As ligas de defesa da segurança nacional, que arvoram um canhão
em seus emblemas; as seções da Navy League,\footnote{ Fundada  
em 1902, com os encorajamentos do presidente Theodore Roosevelt, 
a Navy League (Liga Naval) existe ainda hoje e conta atualmente 75 mil membros. 
O papel dessa associação de bons patriotas é “educar” seus concidadãos e 
“apoiar o pessoal da Marinha”, que reagrupa também o corpo dos \textit{marines}, 
a guarda"-costeira e a marinha mercante.} cujos dirigentes
espalharam"-se pelos quatro cantos do país; mulheres que se vangloriam
de pertencer ao “sexo fraco”, mulheres que dão a vida no sofrimento e
no perigo, pois bem, essas mulheres estão prontas a sacrificar sua
prole ao Moloch da Guerra. As sociedades pela americanização,\footnote{ 
Associações beneficentes (ou programas
financiados pelo governo federal) que ensinavam os “valores fundamentais” da
América aos imigrantes desejosos de obter a nacionalidade americana.
Cursos de inglês, história do país e de suas instituições, mas também
cursos de cozinha, conselho para a educação das crianças etc. Esse
movimento, assaz forte antes e durante a \textsc{i} Guerra Mundial, não resistiu
às leis que limitaram a imigração e à ascensão da xenofobia no início dos
anos 1920.}
às quais pertencem pessoas com ideias muito liberais, e que ainda ontem
denunciavam as asneiras patrióticas, aceitam hoje confundir o espírito
da opinião pública e ajudar a construir as mesmas forças de destruição
na América que elas tentam, direta ou indiretamente, destruir na
Alemanha. O militarismo mata a juventude, viola as mulheres, extermina
o melhor da humanidade, aniquila a própria vida.

Até mesmo Woodrow Wilson,\footnote{ Thomas 
Woodrow Wilson (1856"-1924). Advogado, professor de ciência
política, governador de Nova Jersey em 1911, presidente (democrata)
eleito em 1912 e reeleito em 1916. É um dos mentores da
Sociedade das Nações (ancestral da \textsc{onu}), cuja criação ele impôs após a
\textsc{i} Guerra Mundial ao ameaçar concluir uma paz separada com a Alemanha.
Vê-se, assim, que os problemas entre a “velha Europa” e os Estados Unidos não
datam de ontem. É cômico que, nos dicionários e nos livros de história,
Wilson seja sempre apresentado como um grande “anticolonialista”. Com
efeito, por três vezes ele enviou o exército americano contra os povos
haitiano, dominicano e mexicano quando foi presidente dos Estados
Unidos.}
que, ainda há pouco, declarava: “Uma nação é
demasiado orgulhosa para lutar”; que, no começo da guerra, aconselhou
que se rezasse pela paz; ele que, em seus discursos, falava da
necessidade de aguardar com prudência; pois bem, mesmo Woodrow Wilson
aderiu ao discurso. Juntou"-se agora a seus colegas ultrachauvinistas;
ecoou seus clamores para instaurar a preparação militar e doravante
grita também: “A América para os americanos”. A diferença entre Wilson
e Roosevelt é a seguinte: Roosevelt, um bruto de nascença, utiliza o
cassetete; Wilson, o historiador, o professor, porta a máscara
cuidadosamente polida dos universitários, mas sob essa máscara, assim
como Roosevelt, só tem um objetivo: servir aos interesses do grande
capital, para ajudar aqueles que estão se tornando fenomenalmente
ricos, produzindo ainda mais acessórios militares.

Woodrow Wilson, em seu discurso diante das Filhas da Revolução
Americana,\footnote{ Associação patriótica e esnobe criada
em 1891 e reservada aos descendentes dos soldados ou civis que
participaram da luta pela independência americana. Na década de 1980, 
essa organização ainda agrupava 200 mil membros.} 
desmascarou"-se quando exclamou: “Eu preferiria ser
espancado a ser posto no índex”. Efetivamente, erguer"-se contra os
fabricantes de munições e armas --- Bethlehem, Du Pont, Baldwin,
Remington e outros Winchester --- conduz ao ostracismo e à morte política.
Wilson sabe disso; portanto, trai sua posição original, rejeita sua
antiga pretensão de ser “demasiado orgulhoso para combater” e berra tão
forte quanto qualquer político asqueroso que é preciso generalizar a
preparação militar e levar a nação às nuvens. Chega, inclusive, a
sustentar a estúpida reivindicação avançada pelas mulheres da Navy
League, que querem impor em cada escola o seguinte juramento:
“Engajo"-me a fazer tudo o que está ao meu alcance para servir aos
interesses de meu país, apoiar suas instituições e defender a honra de
seu nome e de sua bandeira. Como devo tudo a meu país, consagrarei meu
coração, meu espírito e meu corpo a seu serviço, e prometo trabalhar
para seu progresso e sua segurança em tempo de paz. Engajo"-me a não
hesitar diante de nenhum sacrifício, de nenhuma privação por sua causa,
se eu for chamado a agir para defender a liberdade, a paz e a
felicidade de nosso povo”.

Defender as instituições de nosso país é defender as instituições que
protegem e apoiam um punhado de indivíduos para que eles roubem e
pilhem as massas; instituições que sugam o sangue dos autóctones tanto
quanto dos estrangeiros, e transformam"-no em riquezas e em poder;
instituições que despojam cada imigrado da cultura original que levou
consigo e impõem"-lhe, em troca, esse americanismo barato, cuja única
glória é a mediocridade e a arrogância.

Aqueles que proclamam “A América de início!” traíram desde há muito os
princípios fundamentais dos autênticos valores americanos, aqueles que
Jefferson tinha em mente quando declarou que o melhor governo é aquele
que governa o menos possível; aqueles pelos quais lutou David Thoreau\footnote{
Henry David Thoreau (1817"-1862). Escritor que, em nome do
individualismo, opunha"-se a toda coerção abusiva da comunidade.
Passou uma noite na prisão por ter se recusado a pagar seus impostos,
pois se opunha à guerra contra o México. Considerado um dos precursores
da não"-violência por Gandhi e Luther King, defendeu a invasão de John
Brown e seus partidários ao arsenal de Harpers Ferry com vistas a
distribuir armas aos escravos negros. Pensador inclassificável, seus
textos podem ser utilizados tanto pelos ecologistas quanto pelas
milícias patrióticas de extrema direita ou pelos anarquistas, que
esquecem que ele escreveu um dia: “Entretanto, para exprimir"-me de
maneira concreta, como cidadão, e não à maneira daqueles que se
proclamam hostis a toda forma de governo, não reivindico para já seu
desaparecimento, mas sua melhoria imediata”.}
quando proclamou que o melhor governo é aquele que não governa; ou
aqueles de todos os grandes americanos que quiseram fazer desse país um
refúgio, esperando que os deserdados e os oprimidos que para aqui
viessem pudessem trazer"-lhe um pouco mais de personalidade,
qualidade e senso. Não é a América dos políticos e dos especuladores da
indústria de armas, que foi poderosamente representada por um
jovem escultor nova"-iorquino: uma mão cruel de longos e finos dedos
que esmagam sem piedade a cabeça de um imigrante, fazendo escorrer o
sangue para dele fazer dólares e embalar o imigrante com esperanças
rompidas e aspirações sufocadas.

Tendo em vista sua posição, Woodrow Wilson tem razão em defender essas
instituições. Mas que ideal ele oferece à nova geração? Como se forma
um militar para defender a liberdade, a paz e a felicidade? Escutemos o
Major"-General O’Ryan: “Todo soldado deve ser treinado para
tornar"-se um simples autômato, privado de iniciativa individual,
transformado em máquina. Ele deve passar à força a coleira
militar pela cabeça, ser dinamizado, dirigido por superiores que têm a pistola na
mão”.

Esse discurso não foi pronunciado por um \textit{junker} prussiano, nem por um
bárbaro germânico, nem por Treitschke\footnote{
Heinrich von Treitschke (1834"-1896). Historiador e escritor político alemão
reacionário. Deputado no Reichstag. Partidário da unidade alemã sob o
comando da Prússia. Considerava a Alemanha como a verdadeira herdeira
do Sacro Império Romano-Germânico e pensava que seu país deveria se
tornar uma grande potência imperialista dotada de um Estado forte,
dirigido por uma elite que não fosse paralisada por um Parlamento
pusilânime.} ou Bernhardi,\footnote{
Friedrich von Bernhardi (1849"-1930). General alemão e autor de duas
obras de títulos proféticos: \textit{A Alemanha e a próxima guerra} (1912) e
\textit{Nosso futuro} (1913).} mas por um Major"-General 
americano! E esse homem tem razão. Não se pode conduzir uma
guerra com homens iguais, não se pode impor o militarismo a homens
livres. É preciso ter à sua disposição escravos, autômatos, máquinas,
criaturas obedientes e disciplinadas, que se deslocarão, agirão,
matarão e dispararão sob as ordens de seus superiores. Eis em que
resultará a preparação militar; nada além disso.

Parece que Samuel Gompers\footnote{
Samuel Gompers (1850"-1924). Esteve na origem do American Federation of
Labor, sindicato fundado sobre os ofícios e que se dirigia aos operários
qualificados. Pregava a colaboração com o patronato com vistas a obter
“bons” contratos coletivos. Apoiou Wilson durante a \textsc{i} Guerra Mundial.}
fazia parte dos oradores que tomaram a
palavra diante da Navy League. Se essa informação for exata, então
nunca ultraje mais grave foi infligido ao movimento operário por um de
seus dirigentes. A preparação militar não é dirigida principalmente
contra o inimigo externo; ela visa sobretudo ao inimigo interno, todos
os elementos do movimento operário que aprenderam a nada esperar de
nossas instituições; os trabalhadores conscientes que compreenderam que
a guerra de classes subentende todas as guerras entre as nações;
aqueles que sabem que, se uma guerra é justificada, trata"-se da
guerra contra a dependência econômica e a escravidão política, os dois
principais problemas concernidos pela luta de classes.

O militarismo já desempenhou seu papel sanguinário em cada conflito
econômico, com a aprovação e o apoio do Estado. Washington protestou
quando “nossos homens, nossas mulheres e nossas crianças” foram mortos
em Ludlow?\footnote{
Em 20 de abril de 1914, 20 homens, mulheres e crianças foram
assassinados em Ludlow, Colorado. Os mineiros desse e de outros
estados do Oeste tentavam aderir à \textsc{umwa} (sindicato dos mineiros). 
Em greve, foram expulsos das casas alugadas da
mineradora. Os mineiros em luta e suas famílias dormiam sob
tendas instaladas em um terreno comunal. Um grupo formado por
milicianos, guardas da companhia mineradora, pistoleiros contratados
como detetives particulares e fura"-greves jogaram querosene sobre as
tendas e as incendiaram. As pessoas que conseguiam escapar
das chamas eram metralhadas. No dia do massacre,
os mineiros celebravam a Páscoa ortodoxa, razão pela qual Emma Goldman
alude à “boa acolhida” que recebem os imigrantes (sem dúvida gregos,
neste caso) na América. Nenhum dos responsáveis pelo massacre foi
condenado; em contrapartida, inúmeros mineiros e militantes
sindicalistas foram presos ou demitidos.}
A nota endereçada à Alemanha exprimia um protesto
virulento? Ou será que existe uma diferença entre matar “nossos homens,
nossas mulheres e nossas crianças” em Ludlow e em alto"-mar? Sim, é
este o caso.  Os homens, as mulheres e as crianças de Ludlow eram
trabalhadores, deserdados, danados da terra, imigrantes a quem era
preciso apenas dar um pequeno gosto dos esplendores do americanismo, enquanto
os passageiros de Lusitânia\footnote{
Lusitânia: transatlântico afundado pelos alemães. 1100 pessoas
pereceram (128 americanos). Wilson não declarou 
guerra à Alemanha.} representavam a riqueza e ocupavam uma 
elevada posição social --- eis a diferença.

A preparação militar só servirá para reforçar o poder de uma
minoria privilegiada e ajudará a dominar, reduzir à escravidão e
esmagar o movimento operário. Samuel Gompers sabe bem disso;
se ele aliou"-se aos gritos da corja militar, ele deve ser condenado 
como traidor do movimento operário.

O mesmo ocorre com todas as outras instituições pretensamente criadas
para o bem do povo e que produziram o resultado inverso. E o mesmo
acontece com a preparação militar. A América sustenta preparar"-se
para a paz, mas, na realidade, a preparação militar provocará a
guerra. Foi sempre assim no transcurso da história sangrenta da
humanidade, e isso continuará até que cada nação recuse"-se a combater
contra uma outra nação, até que os povos do mundo cessem de
preparar"-se para o massacre. A preparação militar é como o grão de
uma planta venenosa: uma vez plantada na terra, ela dará frutos
envenenados. Os massacres na Europa são o fruto desse grão venenoso. É
preciso absolutamente que os operários americanos deem"-se conta disso
antes que eles sejam dominados pelos discursos chauvinistas na loucura
guerreira, loucura sempre assombrada pelo espectro do perigo e da
invasão. Os operários americanos devem saber que se preparar para a paz
significa incitar à guerra, deixar desencadearem"-se as fúrias da morte
na terra e no mar.

As massas europeias que combatem nas trincheiras e nos campos de batalha
não são motivadas por um desejo profundo de fazer a guerra; o que as
levou aos campos de batalha foi a competição impiedosa entre ínfimas
minorias de aproveitadores zelosos em desenvolver os equipamentos
militares, exércitos mais eficazes, navios de guerra maiores, canhões
de longo alcance. Não se pode construir um exército e depois arrumá"-lo
em uma caixa como se faz com soldados de chumbo. Quando um exército é
equipado até os dentes com instrumentos mortíferos sofisticados,
quando ele é sustentado pelos interesses de uma corja belicista, a 
dinâmica torna"-se autônoma. Devemos, então, examinar a natureza do
militarismo para compreender por que a preparação militar é um truísmo.

O militarismo destrói os elementos mais sadios e mais produtivos de cada
nação. Desperdiça a maior parte da renda nacional. O Estado não
despende quase nada para o ensino, a arte, a literatura e a ciência em
comparação com as somas consideráveis que ele consagra ao armamento em
tempo de paz. Em tempo de guerra, então, todo o resto não tem nenhuma
importância; a vida estagna, todos os esforços são bloqueados; o suor e
o sangue das massas servem para nutrir o monstro insaciável do
militarismo. Ele se torna, portanto, cada vez mais arrogante,
agressivo, imbuído de sua importância. Para permanecer vivo, o
militarismo necessita constantemente de energia suplementar; eis por
que ele buscará sempre um inimigo ou, em sua ausência, criará um
artificialmente. Em seus objetivos e seus métodos civilizados, é
sustentado pelo Estado, protegido pelas leis, mantido pelos pais e
pelos professores, glorificado pela opinião pública. Em outros
termos, a função do militarismo é matar. Ele só pode viver graças ao
assassinato.

Mas a preparação militar conduz inevitavelmente à guerra por uma outra
razão, ainda mais fundamental. Ela encoraja a criação de grupos de
interesses, que trabalham consciente e deliberadamente para aumentar a
produção de armamentos e manter uma histeria belicista. Esse \textit{lobby}
inclui todos aqueles que estão engajados na fabricação e na venda de
munições e equipamentos militares com vistas a acumular ganhos e
benefícios pessoais. Tomemos como exemplo o caso da família Krupp, que
possui a maior fábrica de munições do mundo; sua sinistra influência na
Alemanha e em muitos outros países estende"-se à imprensa, às escolas,
às igrejas e aos homens de Estado encarregados das mais elevadas
responsabilidades. Pouco antes da guerra, Karl Liebknecht, o único
político corajoso na Alemanha de hoje, chamou a atenção do Reichstag: a
família Krupp pagava os serviços de funcionários ocupando funções
militares muito elevadas, não apenas na Alemanha, mas também na França
e em outros países. Em toda parte seus emissários agiam e atiçavam
sistematicamente os ódios e os antagonismos nacionais. Liebknecht
desmascarou um truste internacional especializado na fabricação de
armamentos. Esse truste zomba completamente do patriotismo e do amor ao
povo, mas utiliza esses dois sentimentos para incitar à guerra e
embolsar milhões de lucro no âmbito desse terrível mercado.

Não é absolutamente impossível que os historiadores da guerra atual
descubram um dia que esse truste internacional do assassínio está na
origem do conflito mundial em curso. Mas será mesmo preciso que cada
geração atravesse oceanos de sangue e produza montanhas de cadáveres
para que a geração seguinte extraia disso algumas lições? Não podemos,
desde hoje, tirar proveito disso para desvelar a causa que conduziu à
guerra europeia? É a preparação militar a causa da guerra, ao fim de
uma preparação aprofundada e eficaz por parte da Alemanha e de outros
países que buscaram reforçar seus exércitos e deles retirar vantagens
materiais? A preparação militar na América deve conduzir e conduzirá ao
mesmo resultado, à mesma barbárie, ao mesmo sacrifício absurdo da vida.
Se a América tomar esse caminho, isso não beneficiará unicamente aos
Krupp americanos, às corjas militares americanas? Isso parece
verossímil quando ouvimos os gritos chauvinistas da imprensa, as
tiradas tonitruantes de Roosevelt, o discurso sentimentaloide enganador
de nosso universitário"-presidente.

Uma razão a mais para aqueles que ainda amam a liberdade e a
humanidade protestarem contra esse crime gigantesco, contra as
atrocidades que hoje se preparam e são impostas ao povo americano. Não
basta dizer"-se neutro; uma neutralidade que verte lágrimas de
crocodilo com um olho e conserva o outro atento às vantagens que
extrairá dos aprovisionamentos militares e dos empréstimos de guerra;
tal neutralidade é uma fraude, que só serve para cobrir com um véu 
hipócrita os crimes dos outros países. Não basta juntar"-se aos
pacifistas burgueses, que proclamam a paz entre as nações ao mesmo
tempo que contribuem para perpetuar a guerra entre as classes, guerra
que, na realidade, subentende todas as outras guerras.

É nessa guerra de classes que devemos concentrar"-nos. Devemos
denunciar os falsos valores, as instituições malfazejas e todas as
atrocidades cometidas pela sociedade burguesa. Aqueles que estão
conscientes da necessidade vital de participar de grandes lutas devem
opor"-se à preparação militar imposta pelo Estado e pelo capitalismo
para a destruição das massas. Eles devem incitar as massas a derrubar
simultaneamente o capitalismo e o Estado. Uma preparação sindical e
social, eis aquilo de que necessitam os trabalhadores. Só isso conduz à
revolução de base contra a destruição de massa planejada pelas elites.
Só isso reforça o autêntico internacionalismo do movimento operário
contra os imperadores, os reis, os diplomatas, as corjas e burocracias
militares. Só essa preparação dará ao povo o meio de tirar as crianças
dos casebres, das oficinas insalubres e das tecelagens de algodão. Só
essa preparação lhes permitirá inculcar na nova geração um ideal de
fraternidade, ensinar"-lhes a brincar, cantar e apreciar a beleza,
educar meninos e meninas para que se tornem adultos livres, não
autômatos. Só essa preparação permitirá que as mulheres sejam as verdadeiras
mães da humanidade; que homens e mulheres mostrem"-se criativos
para a raça humana, e não se tornem soldados que a destroem. Só essa
preparação conduzirá à liberdade econômica e social, e porá um termo a
todas as guerras, a todos os crimes e a todas as injustiças.

\chapter{O patriotismo, uma ameaça à liberdade}

\textsc{O que é} o patriotismo? É o fato de amar o local onde se nasceu, o lugar
onde se manifestaram os sonhos e as esperanças de nossa infância,
nossas mais profundas aspirações? É o lugar onde, em nossa ingenuidade
infantil, observávamos as nuvens desfilarem no céu com rapidez,
perguntando"-nos por que não podíamos nos deslocar tão velozmente? O
local onde contávamos milhares de estrelas cintilantes, assustados com
a ideia de que cada uma delas pudesse ser um dos olhos do Senhor e
fosse capaz de perscrutar os grandes segredos de nossa pequena alma? O
lugar onde escutávamos o canto dos pássaros, e desejávamos ardentemente
ter asas para voar, assim como eles, rumo a regiões distantes? Ou
aquele onde nós nos sentávamos sobre os joelhos de nossa mãe,
fascinados por contos maravilhosos relatando façanhas inauditas e
incríveis conquistas? Em resumo, o patriotismo define"-se pelo amor a
um pedaço dessa terra onde cada centímetro quadrado representa
preciosas recordações, caras ao nosso coração, e nos lembra uma
infância feliz, alegre, vivaz?

Se fosse isso o patriotismo, seria difícil apelar hoje a esses
sentimentos na América. Com efeito, nossos campos de esporte foram
transformados em usinas, fábricas e minas, e o barulho ensurdecedor das
máquinas substituiu a música dos pássaros. Já não nos é possível ouvir
belas histórias, sonhar com nobres façanhas, pois hoje nossas mães só
nos falam de seus sofrimentos, suas lágrimas e sua dor.

Então, o que é o patriotismo? “O patriotismo, senhor, é o último recurso
dos vagabundos”, declarou o dr.~Johnson. Liev Tolstoi, o mais célebre
dos antipatriotas de nossa época, assim o define: o patriotismo é um
princípio que justifica a instrução de indivíduos que cometerão
massacres em massa; um comércio que exige um equipamento bem melhor
para matar outros homens do que para fabricar gêneros de primeira
necessidade --- sapatos, vestimentas ou moradias; uma atividade econômica
que garante maiores lucros e uma glória bem mais cintilante do que
aquela da qual jamais fruirá o operário médio.

Gustave Hervé, um outro grande antipatriota,\footnote{
Gustave Hervé (1871"-1944). Varrido da universidade por suas
posições antimilitaristas em 1901, ele funda o hebdomadário \textit{La Guerre
Sociale} em 1906, publicação com tiragem de até 60 mil exemplares antes
da guerra. Em 1914, torna"-se ultrapatriota, depois desliza cada vez
mais para a direita até fundar um pequeno partido fascista favorável a
Mussolini.} considera o patriotismo
como uma superstição bem mais perigosa, brutal e desumana que a
religião. A superstição da religião provém da incapacidade do homem de
explicar os fenômenos naturais. Com efeito, quando os homens primitivos
ouviam o estrondo do trovão ou viam relâmpagos, eles não podiam achar
explicação para isso. Concluíam que, por trás desses fenômenos,
ocultava"-se uma força mais poderosa do que eles próprios. Assim, os
homens viram uma entidade sobrenatural na chuva e nas diferentes
manifestações da natureza. O patriotismo, por sua vez, é uma
superstição criada artificialmente e mantida por uma rede de mentiras e
falsidades; uma superstição que retira do homem todo o respeito por si
mesmo e toda a dignidade, e aumenta sua arrogância e seu desprezo.

Com efeito, desprezo, arrogância e egoísmo são os três elementos
fundamentais do patriotismo. Permiti"-me dar"-vos um exemplo. Segundo
a teoria do patriotismo, nosso globo seria dividido em pequenos
territórios, cada um cercado por uma cerca metálica. Aqueles que têm a
oportunidade de ter nascido em um território particular consideram"-se
mais virtuosos, mais nobres, maiores, mais inteligentes do que os que
povoam os outros países. É, pois, o dever de todo habitante desse
território lutar, matar e morrer para tentar impor sua superioridade a
todos os outros.

Os ocupantes dos outros territórios raciocinam do mesmo modo,
evidentemente. Resultado: desde seus primeiros anos, o espírito da
criança é envenenado por autênticos relatos de terror concernentes aos
alemães, franceses, italianos, russos etc.

Quando a criança atinge a idade adulta, seu cérebro está completamente
intoxicado: ela crê ter sido escolhida pelo Senhor em pessoa para
defender sua pátria contra o ataque ou a invasão de qualquer
estrangeiro. Por isso tantos cidadãos exigem ruidosamente que se
reforcem as forças armadas, terrestres ou navais, que se construam mais
barcos de guerra e munições. Eis por que a América, em um curtíssimo
período, despendeu 400 milhões de dólares. Refleti sobre esse
número: retiraram 400 milhões de dólares das riquezas
produzidas pelo povo. Pois não são, evidentemente, os ricos que
contribuem financeiramente à causa patriótica. Eles têm um espírito
cosmopolita e estão à vontade em todos os países. Nós, na América,
conhecemos perfeitamente esse fenômeno. Os ricos americanos são
franceses na França, alemães na Alemanha e ingleses na Inglaterra. E
eles desperdiçam, com uma graça totalmente cosmopolita, as fortunas que
acumularam colocando crianças americanas para trabalhar em suas
fábricas e escravos em seus campos de algodão. Seu patriotismo
permite"-lhes enviar mensagens de condolências a um déspota como o
czar da Rússia quando lhe acontece uma desgraça, como 
quando o presidente Roosevelt, em nome do povo americano, apresentou
suas condolências depois que o arquiduque Serguei foi abatido pelos
revolucionários russos.

É o patriotismo que ajudará o superassassino Porfírio Diaz\footnote{
Porfírio Diaz (1830"-1915). Coronel mexicano que se cobre de glória lutando
contra a invasão francesa e o Império de Maximiliano entre 1862 e 1867.
Ditador"-presidente eleito várias vezes entre 1884 e 1910. Renuncia
ante a revolução de maio de 1911.} a suprimir
milhares de vidas no México, ou que prenderá revolucionários
mexicanos em nosso solo e os trancafiará nas prisões americanas, sem o
mínimo motivo.

O patriotismo não concerne àqueles que detêm a riqueza e o poder. É um
sentimento válido unicamente para o povo. Isso me lembra a frase
histórica de Frederico, o Grande, o amigo íntimo de Voltaire: “A
religião é uma fraude, mas é preciso mantê"-la para as massas.” 

O patriotismo é uma instituição dispendiosa e ninguém duvidará disso
após ter lido as estatísticas a seguir. A progressão dos gastos
para os principais exércitos do mundo durante o último quartel de
século é de tal forma fulgurante que só esse fato já deveria fazer com
que toda pessoa que se interesse, ainda que pouco, pelos problemas
econômicos, reagisse. No intervalo de 24 anos, de 1881 a
1905, as despesas evoluíram da seguinte maneira:

\begin{center}
\begin{tabular}{lrr}
 país (\textsc{us}\$)	& de 		& até	 		\\\hline
 Grã"-Bretanha & 2.101.848.936 & 4.143.226.885 	\\
 França 	& 3.324.500.000 & 3.455.109.900 	\\
 Alemanha 	& 725.000.200 	& 2.700.375.600		\\
 Estados Unidos & 1.275.500.750 & 2.650.900.450		\\
 Rússia 	& 1.900.975.500 & 5.250.445.100		\\
 Itália 	& 1.600.975.750 & 1.755.500.100		\\
 Japão 		& 182.900.500 	& 700.925.475 
\end{tabular}
\end{center}

De 1881 a 1905, os gastos militares da Grã"-Bretanha dobraram, assim
como dos Estados Unidos. Os gastos da Rússia quase triplicaram,
enquanto os da Alemanha, da França e do Japão aumentaram
militares dessas nações com suas despesas totais durante esse período
de 24 anos, o aumento é o seguinte: a parte das despesas
militares passou de 20 a 37\% do orçamento global na Grã"-Bretanha; de
15 a 23\% nos Estados Unidos; de 16 a 18\% na França; de 12 
a 15\% na Itália, e de 12 a 14\% no Japão.

Por outro lado, é interessante observar que a proporção na Alemanha
diminuiu de 58 para 25\%, baixa que se deve ao enorme aumento das
despesas imperiais em outras áreas, e ao fato de que as despesas
militares para o período 1901"-1905 eram proporcionalmente mais
elevadas do que em todas as partes nos cinco anos anteriores.

As estatísticas mostram que os países onde as despesas militares
representavam a parte mais importante na renda nacional total eram,
pela ordem, a Grã"-Bretanha, os Estados Unidos, o Japão, a França e a
Itália.

No que concerne às diferentes marinhas nacionais, a progressão é
igualmente impressionante. De 1881 a 1905, as despesas navais
aumentaram do seguinte modo: Grã"-Bretanha, 300\%; França, 60\%;
Alemanha, 600\%; Estados Unidos, 525\%; Rússia, 300\%; Itália, 250\%; e
Japão, 700\%. À exceção da Grã"-Bretanha, os Estados Unidos
despenderam mais por sua marinha do que qualquer outra nação; essa
despesa representa igualmente uma fração mais importante do orçamento
nacional do que todas as outras potências. De 1881 a 1905, as despesas
navais dos Estados Unidos passaram de 6,2 dólares sobre 100 consagrados
ao orçamento do Estado, primeiro para 6,6; depois 8,1; então, 11,7; e, enfim, 16,4 dólares
no último período (1901"-1905). Os números das despesas para o
período 1905"-1910 indicarão certamente um crescimento ainda superior.

O custo cada vez mais elevado do militarismo também pode ser ilustrado se
o calcularmos como um imposto afetando cada contribuinte. De 1889 a
1905, na Grã"-Bretanha, as despesas passaram de 18,47 
a 52,50 dólares por habitante; na França, de 19,66 a 23,62 dólares; na
Alemanha, de 10,17 a 15,51 dólares; nos Estados Unidos, de 5,62 a 13,64
dólares; na Rússia, de 6,14 a 8,37 dólares; na Itália, de 9,59 a 11,24
dólares; enfim, no Japão, de 0,86 a 3,11 dólares.

Esses cálculos mostram a que ponto o custo econômico do militarismo pesa
sobre a população. Que conclusão extrair disso tudo? O aumento do
orçamento militar ultrapassa o crescimento da população em cada um dos
países supracitados. Em outros termos, as exigências crescentes do
militarismo ameaçam esgotar os recursos humanos e materiais de cada uma
dessas nações.

O horrível desperdício acarretado pelo patriotismo deveria ser
suficiente para curar os homens, mesmo medianamente inteligentes, dessa
doença. Entretanto, as exigências do patriotismo não param aí.
Pede"-se ao povo para ser patriota e, para esse luxo, ele paga não
sustentando seus “defensores”, mas sacrificando seus próprios filhos. O
patriotismo exige uma vassalagem total à bandeira, o que implica
obedecer e estar pronto a matar seu pai, sua mãe, seu irmão ou sua
irmã.

“Necessitamos de um exército permanente para proteger o país contra uma
invasão estrangeira”, afirmam nossos governantes. Todo homem e toda
mulher inteligentes sabem, contudo, que se trata de um mito destinado a
apavorar as pessoas crédulas e obrigá"-las a obedecer. Os governos
deste planeta conhecem perfeitamente seus interesses respectivos e não
invadem uns aos outros. Eles aprenderam que podem ganhar muito mais
recorrendo à arbitragem internacional para resolver seus conflitos do
que fazendo guerra e tentando conquistar outros territórios. Na
verdade, assim como disse Carlyle, “a guerra é uma querela entre dois
ladrões demasiado covardes para produzir seu próprio combate; é por
isso que eles escolhem jovens egressos de vilarejos diferentes,
põem"-lhes um uniforme nas costas, dão"-lhes um fuzil e soltam"-nos
como animais selvagens para que eles se entredestruam”.

Não é preciso ser muito douto para encontrar uma causa idêntica para
todas as guerras. Tomemos a guerra hispano"-americana, tida como um
grande evento patriótico na história dos Estados Unidos. Como nossos
corações queimaram de indignação ao tomar conhecimento das atrocidades
espanholas! Reconheçamos que nossa indignação não eclodiu
espontaneamente. Ela foi nutrida pela imprensa, durante meses e meses,
e muito tempo depois que o açougueiro Weyler\footnote{
Valeriano Weyler y Nicolau (1838"-1930). General espanhol que
esmagou por duas vezes movimentos dirigidos contra a dominação
espanhola em Cuba (1868"-1872 e 1896"-1897), mas também nas Filipinas
em 1888. Seus métodos sanguinários serviram de pretexto à guerra
hispano"-americana. Comandante"-em"-chefe do exército espanhol em
1921"-1923.} matou numerosos nobres
cubanos e estuprou numerosas cubanas.

Todavia, façamos justiça à nação americana: não apenas ela indignou"-se
e mostrou sua vontade de lutar, mas combateu corajosamente. Entretanto,
quando a fumaça dissipou"-se, quando os mortos foram enterrados e
o custo da guerra recaiu sobre o povo sob a forma de aumento no
preço das mercadorias e dos aluguéis, quando emergimos de nossa
embriaguez patriótica, compreendemos de repente que a verdadeira causa
da guerra hispano"-americana foi o preço do açúcar. Ou, para ser ainda
mais explícita, que as vidas, o sangue e o dinheiro do povo americano
haviam sido utilizados para proteger os interesses dos capitalistas
americanos, ameaçados pelo governo espanhol.

Não exagero, absolutamente. Minha afirmação fun\-damenta"-se em fatos e
estatísticas incontestáveis, como o prova igualmente a atitude do
governo americano ante os trabalhadores cubanos. Quando Cuba
encontrou"-se espremida entre as garras dos Estados Unidos, os
soldados enviados para libertar Cuba receberam a ordem para fuzilar os
trabalhadores cubanos durante a grande greve das fábricas de charutos,		\EP[.6]
greve que ocorreu pouco após a guerra hispano"-americana.

E não somos os únicos a fazer guerra por tais motivações: mal se
começa a desvelar os verdadeiros motivos da terrível guerra
russo"-japonesa, que custou tanto sangue e lágrimas.

E, de novo, vemos que, por trás do cruel Moloch da Guerra, ergue"-se o
deus ainda mais cruel do Comércio. Kuropatkin, o ministro russo da
Guerra durante esse conflito, revelou o verdadeiro segredo que se
oculta por trás das aparências. O czar e seus grão"-duques haviam
investido dinheiro em concessões coreanas; eles impuseram a guerra
unicamente no interesse das fortunas que estavam edificando a toda
velocidade.

A constituição de um exército permanente é a melhor maneira de assegurar
a paz? Este argumento é absolutamente ilógico: é como se se sustentasse
que o cidadão mais pacífico é aquele que está mais bem armado. A
experiência mostra que indivíduos armados desejam sempre testar sua
força. O mesmo acontece com os governos. Os países verdadeiramente
pacíficos não mobilizam seus recursos e sua energia em preparativos de
guerra, evitando, assim, todo conflito com seus vizinhos.

Aqueles que reivindicam o aumento dos meios do exército e da marinha não
pensam em nenhum perigo externo. Eles observam o crescimento do
descontentamento das massas e do espírito internacionalista entre os
trabalhadores. Eis o que os inquieta de fato. É para afrontar seu
inimigo interno que os governantes de diferentes países preparam"-se
neste momento; um inimigo que, uma vez desperto, revelar"-se"-á mais
perigoso do que qualquer invasor estrangeiro.

Os poderosos que reduziram as massas à escravidão durante séculos
estudaram cuidadosamente sua psicologia. Eles sabem que os povos em
geral são como crianças cujos desespero, sofrimento e choro podem
transformar"-se em alegria à vista de um pequeno brinquedo. E quanto
mais bonita for a apresentação do brinquedo, quanto mais vivas as
cores, mais ele agradará a milhões de crianças.

O exército e a marinha são os brinquedos do povo. A fim de torná"-los
ainda mais atraentes e aceitáveis, gastam"-se centenas de milhares de
dólares para exibi"-los por toda parte. É o objetivo que buscava o
governo americano quando equipou uma frota e a mandou percorrer
a costa do Pacífico, a fim de que cada cidadão americano pudesse
orgulhar"-se das façanhas técnicas dos Estados Unidos. A cidade de San
Francisco gastou 100 mil dólares para a diversão da frota; Los Angeles
60 mil; Seattle e Tacoma aproximadamente 100 mil dólares. Para
divertir a frota, eu disse? Para oferecer excelente banquete e vinhos
finos a alguns oficiais superiores, enquanto os “gentis soldados rasos”
tinham de amotinar"-se para obter uma refeição decente. Sim, 
260 mil dólares foram gastos para financiar fogos de artifício,
espetáculos e festividades, num momento em que milhares de homens,
mulheres e crianças, em todo o país, morriam de fome nas ruas, num
momento em que centenas de milhares de desempregados estavam prestes a
vender seu trabalho a qualquer preço.

Duzentos e sessenta mil dólares! Quantas coisas poderiam ser realizadas
com uma soma tão impressionante! Todavia, em vez de dar"-lhes um teto
e alimentá"-los corretamente, preferiu"-se levar as crianças dessas
cidades para assistir às manobras da frota, pois esse espetáculo, como
disse um jornalista, deixará “uma lembrança inefável em sua memória”.

Que maravilhosa lembrança, não é? Todos os ingredientes necessários a um
massacre civilizado. Se o espírito das crianças é intoxicado por tais
lembranças, que esperança existe para o advento de uma autêntica
fraternidade humana?

Nós, americanos, dizemos amar a paz. Parece que detestamos verter
sangue, que nos opomos à violência. E, contudo, pulamos de júbilo
quando aprendemos que máquinas voadoras poderão lançar bombas recheadas
de dinamite sobre cidadãos sem defesa. Estamos prontos a enforcar,
eletrocutar ou linchar toda pessoa que, levada pela necessidade
econômica, arriscar sua própria vida atentando contra a vida de um
magnata industrial. No entanto, nossos corações inflam"-se de orgulho
ao pensarmos que a América se tornará a nação mais poderosa da
Terra, e que esmagará com suas botas as outras nações.

Tal é a lógica do patriotismo.

Se o patriotismo é nocivo ao comum dos mortais, é pouco em comparação
com os prejuízos e ferimentos que ele inflige ao próprio soldado, esse
homem enganado, vítima da superstição e da ignorância. O que oferece o
patriotismo ao salvador de seu país, ao protetor de sua nação? Uma vida
de escravo submisso, de depravação durante a paz; uma vida de perigo,
de riscos mortais e de morte durante a guerra.

No transcurso de uma recente turnê de conferências em San Francisco,
visitei o Presidio, um lugar maravilhoso que domina a baía e o parque
de Golden Gate. Poder"-se"-ia instalar ali campos de esporte para as
crianças, jardins e orquestras para o lazer da população. Em vez disso,
construíram ali uma caserna constituída de prédios horríveis, cinzentos
e insignificantes, prédios nos quais os ricos não deixariam nem mesmo
seus cães dormirem.

Nesse miserável abarracamento amontoam soldados como gado; eles perdem
seu tempo e sua juventude engraxando as botas e lustrando os botões de
seus oficiais superiores. Lá também pude observar as diferenças de
classe: os robustos filhos de uma República livre, dispostos em fila
como prisioneiros, são obrigados a bater continência sempre 
que um oficial desprezível passa diante deles. Ah! Como a igualdade americana 
degrada a humanidade e exalta o uniforme!

A vida de caserna tende a desenvolver a perversão sexual.\footnote{ 
Refugiando"-se atrás da autoridade de Havelock Ellis, que pertence a
uma longa linhagem de psicólogos ou psicanalistas hostis aos gays, Emma
Goldman julga aqui que a homossexualidade masculina é uma “perversão”,
um “vício” etc. Esta não é mais considerada como uma “doença” pelos psicólogos
americanos desde os anos 1970. Perguntamo"-nos que descoberta
“científica” pôde motivar sua decisão! Observemos, por outro lado, que
toda essa passagem sobre os bordéis militares compostos de prostitutos
parece assaz inverossímil, pois sabemos que a sodomia era considerada
crime na época, e mais ainda no exército.} Ela produz
gradualmente resultados semelhantes nos exércitos europeus. Havelock
Ellis, especialista renomado em matéria de psicologia sexual,
apresentou um estudo detalhado relativo a esse tema.

“Alguns abarracamentos são autênticos bordéis para os prostitutos[\ldots] O
número de soldados que querem prostituir"-se é bem maior do que
estamos prontos a admitir. Em certos regimentos, a maioria dos
conscritos está disposta a vender"-se[\ldots] No verão, vemos soldados da
Guarda Real e outros regimentos exercer seu comércio ao anoitecer, em
Hyde Park e nos arredores de Albert Gate; eles não se escondem, alguns
caminham inclusive fardados. [\ldots] O ganho dessas atividades é uma soma
confortável que vem reforçar seu magro soldo.”

Essa perversão progrediu no exército, a ponto de se criar casas
especializadas para essa forma de prostituição. A prática não se limita
à Inglaterra, ela é universal. “Os soldados são procurados tanto na
França como na Inglaterra ou na Alemanha, e bordéis especializados na
prostituição militar existem tanto em Paris como nas cidades de
guarnição.”

Se o sr.~Havelock Ellis tivesse pesquisado sobre a perversão sexual na
América, ele teria descoberto que a mesma situação existe em nosso
exército. O crescimento de um exército permanente só pode aumentar a
extensão da perversão sexual; as casernas são suas incubadoras.

Fora das consequências sexuais deploráveis da vida comum nas casernas, o
exército tende a tornar o soldado inapto para trabalhar quando
deixa suas fileiras. É raro que homens qualificados engajem"-se, mas
quando acontece de o fazerem, ao final de alguns anos de experiência
militar eles têm dificuldade para retomar suas ocupações anteriores.
Tendo aprendido a gostar do ócio, de certas formas de excitação e
aventura, nenhuma ocupação pacífica pode satisfazê"-los mais. Livres
de suas obrigações militares, tornam"-se incapazes de efetuar o mínimo
trabalho útil. Mas, habitualmente, o recrutamento dá"-se sobretudo
entre a canalha ou é proposto a prisioneiros libertados com esse
objetivo. Estes aceitam para sobreviver ou porque são levados por suas
tendências criminais. É sabido que nossas prisões pululam de
ex"-soldados, enquanto que, por outro lado, o exército e a marinha
acolhem muitos ex"-condenados. Esses indivíduos, quando seu tempo na
caserna expira, retornam à sua vida criminal anterior, ainda mais
violentos e depravados do que antes.

De todos os fenômenos negativos que acabo de descrever, nenhum me parece
mais nocivo à integridade humana do que as consequências do patriotismo
para o segunda-classe William Buwalda. Porque ele cometeu a loucura de
crer que se pode ser um soldado e exercer seus direitos de ser humano;
as autoridades militares puniram"-no severamente. É verdade, ele
servira seu país durante quinze anos, durante os quais seu dossiê fora
impecável.

Segundo o general Funston, que reduziu a condenação de Buwalda a três
anos de prisão, “o primeiro dever de um oficial ou de um engajado é
obedecer cega e lealmente ao governo. O fato de ele aprovar ou não o
governo não deve ser levado em consideração”. Essa declaração esclarece
o verdadeiro caráter da vassalagem patriótica. Segundo o general
Funston, o fato de entrar no exército anula os princípios da Declaração
de Independência.

Que estranho resultado produz esse patriotismo que transforma um
ser pensante numa máquina leal!

Para justificar a escandalosa condenação de Buwalda, o general Funston
explica aos americanos que esse soldado cometeu “um crime grave, que
equivale à traição”. Do que se trata exatamente? William Buwalda
assistiu a um comício com 1.500 pessoas em San Francisco. Em seguida --- que
horror! --- ele apertou a mão da oradora: Emma Goldman. Um terrível
crime, efetivamente, que o general Funston qualifica de “grave crime
militar, infinitamente mais grave do que a deserção”!

Que argumento mais aterrador podemos invocar contra o patriotismo do que
o fato de estigmatizar esse homem como um criminoso, jogá"-lo na
prisão e furtar"-lhe o fruto de quinze anos de bons e leais serviços?

Buwalda deu a seu país os melhores anos de sua vida adulta. Mas tudo
isso não conta. Assim como todos os monstros insaciáveis, o patriotismo
inflexível exige um devotamento absoluto. Não admite que um soldado
também seja um ser humano, que ele tenha direito de ter suas opiniões e
seus sentimentos pessoais, seus pendores e suas próprias ideias. Não, o
patriotismo não o admite. Buwalda teve de aprender essa lição pagando
elevado preço, mas não inútil. Quando saiu da prisão, havia perdido sua
posição no exército, mas havia reconquistado o respeito de si mesmo. Em
todo caso, isso lhe valeu três anos de prisão.

Um jornalista publicou recentemente um artigo sobre o poder que os
militares alemães exercem sobre os civis. Esse senhor pensa,
notadamente, que, se nossa República não tivesse outra função senão
garantir a todos os cidadãos direitos iguais, sua existência já seria
plenamente justificada. Estou convencida de que esse jornalista não se
encontrava no Colorado durante o regime patriótico do general Ball. Ele
provavelmente teria mudado de opinião se tivesse visto a maneira como,
em nome do patriotismo e da República, jogavam-se os homens em celas
comuns, depois os faziam sair para atravessar a fronteira e
submetê"-los a todos os tipos de tratamentos indignos. E o incidente
ocorrido no Colorado não é um incidente isolado no desenvolvimento do
poder militar nos Estados Unidos. É raro que uma greve ocorra sem que o
exército ou as milícias acudam os poderosos, e então esses homens agem
de modo tão arrogante e brutal quanto aqueles que portam o uniforme do
Kaiser. Além do mais, temos a lei militar Dick. Esse jornalista
esqueceu"-se dela?

O grande problema dos jornalistas é que, geralmente, eles ignoram os
acontecimentos correntes ou, na ausência de honestidade, nunca os
evocam. É assim que a lei militar Dick foi introduzida precipitadamente
ante o Congresso, sem ser verdadeiramente discutida e sem que se
falasse dela na imprensa. Essa lei dá ao Presidente o direito de
transformar um pacífico cidadão num assassino sedento de sangue, em
teoria para defender seu país, mas na realidade para proteger os
interesses do partido do qual o Presidente é porta"-voz.

Nosso jornalista sustenta que o militarismo jamais poderá adquirir tanto
poder na América quanto em outros países, porquanto não conhecemos o
alistamento obrigatório como no Velho Mundo. Esse senhor esquece dois
fatos muito importantes. De início, esse recrutamento criou na Europa
um profundo ódio contra o militarismo, ódio enraizado em todas as
classes da sociedade. Milhares de jovens recrutas protestam no momento
de sua incorporação e, uma vez no exército, muito amiúde tentam, por
todos os meios, desertar. Em segundo lugar, nosso jornalista não leva
em consideração que a conscrição obrigatória criou um movimento
antimilitarista muito importante que as potências europeias temem mais
do que tudo. Com efeito, o militarismo é a muralha mais sólida do
capitalismo. Assim que ele for estremecido, o capitalismo vacilará em
suas bases. É verdade: na América, não temos serviço militar
obrigatório; os homens não são obrigados a alistar"-se no exército;
mas desenvolvemos uma força bem mais exigente e rígida: a necessidade.
Durante as crises econômicas, o número de engajados não aumenta
vertiginosamente? O ofício de militar talvez seja menos lucrativo ou
honorável do que outros, mas é melhor ser soldado do que errar em todo
o país à procura de trabalho, fazer fila para uma sopa popular, ou
dormir em asilos noturnos. Tudo bem pesado, um soldado recebe
atualmente treze dólares por mês, faz três refeições diárias e tem um
lugar para dormir. Entretanto, a necessidade não é um fator bastante
forte para humanizar o exército. Não surpreende que nossas
autoridades militares queixem"-se da “má qualidade” dos elementos que
se engajam. Essa confissão é muito estimulante. Prova que o espírito de
independência e o amor pela liberdade ainda estão suficientemente
disseminados entre os americanos para incitá"-los a preferir morrer de
fome a vestir o uniforme.

Os homens e as mulheres que refletem neste mundo começam a compreender
que o patriotismo é uma concepção demasiado estreita e limitada para
responder às necessidades de nossa época. A centralização do poder
criou um sentimento internacional de solidariedade entre as nações
oprimidas do mundo, solidariedade que revela uma maior comunidade de
interesses entre os operários americanos e seus irmãos de classe no
estrangeiro do que entre um mineiro americano e seu compatriota que o
explora; uma solidariedade que não teme nenhuma invasão estrangeira,
porque ela conduzirá todos os operários a dizer um dia a seus patrões:
“Ide matar"-vos, se desejardes. Quanto a nós, faz muitíssimo tempo que
combatemos no lugar de vocês.”

Essa solidariedade desperta igualmente a consciência dos soldados, que
também fazem parte da grande família humana. Essa solidariedade
revelou"-se infalível várias vezes durante as lutas passadas. Ela 
levou os soldados parisienses, durante a Comuna de 1871, a recusar"-se
a obedecer quando receberam ordens para disparar contra seus irmãos.
Ela deu coragem aos marinheiros que se amotinaram recentemente nos
navios de guerra russos. E ela provocará um dia a sublevação de todos
os oprimidos e a revolta contra seus exploradores internacionais.

O proletariado europeu compreendeu a grande força dessa solidariedade e
começou uma guerra contra o patriotismo e seu espectro, o niilismo.
Milhares de homens enchem as prisões da França, da Alemanha, da Rússia
e dos países escandinavos porque ousaram desafiar uma muito antiga
superstição. E esse movimento não se limita à classe operária: ele
abrange todas as categorias sociais; seus principais porta"-vozes
são homens e mulheres eminentes no campo das artes, das ciências e das
letras.

A América tomará um dia o mesmo caminho. O espírito do militarismo já
invade todos os campos da vida social. Estou convencida de que o
militarismo tornar"-se"-á um perigo mais importante na América do que
em qualquer outro lugar no mundo, porque o capitalismo sabe corromper
aqueles que deseja destruir.\label{dequeomilitarismo}

O processo já está engajado nas escolas. Evidentemente, o governo
defende a velha concepção jesuítica: “Dai"-me o espírito de uma
criança e eu a modelarei”. Ensinam às crianças o interesse pelas
táticas militares, exaltam"-lhes as grande vitórias, e os espíritos
jovens são pervertidos no interesse do governo. Além disso, editam
extraordinários cartazes para incitar os jovens do país a engajar"-se.
“Uma oportunidade para percorrer o mundo!” --- exclamam os lacaios do
governo. E é assim que forçam moralmente jovens inocentes a perder"-se
no patriotismo e que o Moloch militar continua a conquistar a nação.

Durante as greves, o operário americano sofreu terrivelmente com as
intervenções dos soldados, porque estes foram enviados contra ele,
quer pelo Estado local, quer pelo governo federal. É, pois, completamente
normal que o operário despreze os parasitas fardados e manifeste sua
oposição a eles. No entanto, não bastará uma simples diatribe para
resolver esse grave problema. Necessitamos de uma propaganda que faça a
educação do soldado: uma literatura antipatriótica que informe acerca
dos verdadeiros horrores de seu ofício e o faça tomar consciência de
sua relação com aqueles cujo trabalho lhe permite existir. É
precisamente disso que as autoridades mais têm medo. Um soldado que
assiste a uma reunião revolucionária já comete um crime de alta
traição. É certo que eles condenarão igualmente à mesma pena o soldado
que ler uma brochura revolucionária. A autoridade não denunciou como
uma traição, desde tempos imemoriais, todo passo rumo ao progresso?
Aqueles que lutam seriamente pela reconstrução social são perfeitamente
capazes de bem conduzir essa tarefa, pois é provavelmente mais
importante portar a mensagem da verdade nas casernas do que nas
fábricas.

Uma vez que tivermos desvelado a mentira patriótica, teremos aberto o
caminho para o advento da grande estrutura em que todas as nacionalidades
se unirão numa fraternidade universal: uma sociedade autenticamente
livre.

\chapter{A revolução social é portadora de uma mudança radical de valores}

\hedramarkboth{A revolução social é portadora\ldots}{Emma Goldman}

\sectionitem
Os críticos socialistas do fracasso da Rússia, mas não os bolcheviques,
afirmam que a revolução fracassou porque a indústria não havia
alcançado um nível de desenvolvimento suficiente nesse país. Eles se
referem a Marx, para quem a revolução social seria possível unicamente
nos países dotados de um sistema industrial altamente desenvolvido,
com os antagonismos sociais que deles decorrem. Esses críticos deduzem
disso que a Revolução Russa não podia ser uma revolução social e que,
historicamente, estava condenada a passar por uma etapa
constitucional, democrática, completada pelo desenvolvimento de uma
indústria antes que o país se tornasse economicamente maduro para uma
mudança fundamental.

Esse marxismo ortodoxo ignora um fator mais importante e, talvez, até
mais essencial para a possibilidade e o sucesso de uma revolução
social do que o fator industrial. Refiro"-me à consciência das massas
num dado momento. Por que a revolução social não eclodiu, por exemplo,
nos Estados Unidos, na França ou mesmo na Alemanha? Esses países
certamente alcançaram o nível de desenvolvimento industrial fixado por
Marx como o estágio culminante. Na verdade, o desenvolvimento
industrial e as poderosas contradições sociais não são em nenhum caso
suficientes para dar origem a uma nova sociedade ou desencadear uma
revolução social. A consciência social e a psicologia necessárias às
massas estão ausentes em países como os Estados Unidos e os outros que
acabo de mencionar. Eis por que nenhuma revolução social ocorreu nessas
regiões.

Desse ponto de vista, a Rússia possuía uma vantagem sobre os países
mais industrializados e “civilizados”. É verdade, ela era menos
avançada no plano industrial que seus vizinhos ocidentais, mas a
consciência das massas russas, inspirada e agudizada pela Revolução de
Fevereiro, progredia tão rapidamente que, em alguns meses, o povo
estava pronto a aceitar \textit{slogans} ultra"-revolucionários como “Todo
poder aos sovietes” e “A terra aos camponeses, as fábricas aos
operários”.

Não se deve subestimar a significação dessas palavras de ordem. Elas
exprimiam, em larga medida, a vontade instintiva e semiconsciente do
povo, a necessidade de uma completa reorganização social, econômica e
industrial da Rússia. Que país, na Europa ou na América, está pronto a
pôr em prática tais \textit{slogans} revolucionários? Todavia, na Rússia,
durante os meses de junho e julho de 1917, essas palavras de ordem
tornaram"-se populares; elas foram retomadas ativamente, com
entusiasmo, sob a forma da ação direta, pela maioria da população
camponesa e operária de um país de mais de 150 milhões de habitantes.
Isso prova a “aptidão”, a preparação do povo russo para a revolução
social.

No que concerne à “maturidade” econômica, no sentido marxiano do termo,
não se deve esquecer que a Rússia é, sobretudo, um país agrário. O
raciocínio implacável de Marx pressupõe a transformação da população
camponesa numa sociedade industrial, altamente desenvolvida, que fará
amadurecer as condições sociais necessárias a uma revolução.

Mas os acontecimentos na Rússia, em 1917, mostraram que a revolução não
espera esse processo de industrialização e --- mais importante ainda ---
que não se pode fazer a revolução esperar. Os camponeses russos
começaram a expropriar os proprietários rurais e os operários
apoderaram"-se das fábricas sem tomar conhecimento dos teoremas
marxistas. Essa ação do povo, pela virtude de sua própria lógica,
introduziu a revolução social na Rússia, transtornando todos os
cálculos marxianos. A psicologia do eslavo provou que era mais sólida
que todas as teorias social"-democratas.

Essa consciência fundava"-se num desejo ardente de liberdade, nutrido
por um século de agitação revolucionária entre todas as classes da
sociedade. Felizmente, o povo russo permaneceu assaz robusto no plano
político: ele não foi infectado pela corrupção e pela confusão criadas
no proletariado de outros países pela ideologia das liberdades
“democráticas” e do “governo a serviço do povo”. Os russos
permaneceram, nesse plano, um povo simples e natural, que ignora as
sutilezas da política, os arranjos parlamentares e as argúcias jurídicas.
Por outro lado, seu sentido primitivo da justiça e do bem era robusto,
enérgico, nunca foi contaminado pelas espertezas destrutivas da
pseudocivilização. O povo russo sabia o que queria e não esperou que
“circunstâncias históricas inevitáveis” lho trouxessem numa bandeja:
recorreu à ação direta. Para ele, a revolução era uma realidade, não
uma simples teoria digna de discussão.

Foi assim que a revolução social eclodiu na Rússia, a despeito do atraso
industrial do país. Mas fazer a revolução não era suficiente. Também
era preciso que ela progredisse e se ampliasse, que resultasse numa
reconstrução econômica e social. Essa fase da revolução implicava que
as iniciativas pessoais e os esforços coletivos pudessem se exercer
livremente. O desenvolvimento e o sucesso da revolução dependiam da
extensão mais ampla possível do gênio criativo do povo, da colaboração
entre os intelectuais e o proletariado manual. O interesse comum é o
\textit{leitmotiv} de todos os esforços revolucionários, sobretudo de um ponto
de vista construtivo.

Esse objetivo comum e essa solidariedade arrastaram a Rússia numa onda
poderosa, no transcurso dos primeiros dias da Revolução Russa, em
outubro"-novembro de 1917. Essas forças entusiastas teriam podido
deslocar montanhas se a preocupação exclusiva de realizar o
bem"-estar do povo as tivesse inteligentemente guiado. Existia um meio
eficaz para isso: as organizações dos trabalhadores e as cooperativas,
que recobriam a Rússia com uma rede ligando e unindo as cidades aos
campos; os sovietes, que se multiplicavam para responder às
necessidades do povo russo; e, enfim, a \textit{intelligentsia}, cujas
tradições, desde há um século, haviam servido de modo heroico à causa
da emancipação da Rússia.

Mas tal evolução não estava absolutamente presente no programa dos
bolcheviques. Durante os primeiros meses que se seguiram a Outubro,
eles toleraram a expressão das forças populares; deixaram o povo
desenvolver a revolução no seio de organizações de poderes
incessantemente mais amplos. Todavia, tão logo o Partido Comunista
sentiu"-se suficientemente instalado no governo, começou a limitar a
extensão das atividades do povo. Todos os atos dos bolcheviques que se
seguiram --- sua política, suas mudanças de linha, seus compromissos e
seus recuos, seus métodos de repressão e de perseguição, seu terror e a
liquidação de todos os outros grupos políticos ---, tudo isso só
representava meios a serviço de um fim: a concentração do poder do
Estado nas mãos do Partido. De fato, os próprios bolcheviques, na
Rússia, não fizeram mistério disso. O Partido Comunista, afirmavam
eles, encarnava a vanguarda do proletariado, e a ditadura devia
permanecer em suas mãos. Infelizmente para eles, os bolcheviques não
tinham levado em conta seu hóspede, o campesinato, que nem a
\textit{razvyortska} (a Tcheca), nem os fuzilamentos maciços persuadiram a
apoiar o regime bolchevique. O campesinato tornou"-se o recife contra
o qual todos os planos e projetos concebidos por Lênin foram se chocar.
Lênin, hábil acrobata, soube operar, malgrado uma margem de manobra
extremamente estreita. A \textsc{nep} (Nova Política Econômica) foi introduzida
bem na hora para evitar o desastre que, lenta mas seguramente, iria
varrer todo o edifício comunista.

\sectionitem
A \textsc{nep} surpreendeu e chocou a maioria dos comunistas. Eles viram nessa
guinada a derrubada de tudo o que seu Partido havia proclamado --- a
rejeição do próprio comunismo. Para protestar, alguns dos mais antigos
membros do Partido, homens que haviam enfrentado o perigo e as
perseguições sob o antigo regime, enquanto Lênin e Trotski viviam no
estrangeiro em toda segurança, esses homens abandonaram o Partido
Comunista, amargurados e decepcionados. Os dirigentes decidiram, então,
fazer uma espécie de greve. Eles ordenaram que o Partido fosse purgado
de todos os seus elementos “duvidosos”. Quem quer que fosse suspeito de
ter uma atitude independente, e todos os que não aceitaram a nova
política econômica como a última verdade da sabedoria revolucionária,
foram excluídos. Entre estes, encontravam"-se comunistas que, durante
anos, haviam lealmente servido à causa. Alguns deles, feridos no
coração por esse procedimento brutal e injusto, e transtornados pelo
desmoronamento daquilo que veneravam, recorreram inclusive ao suicídio.
Entretanto, era preciso que o novo Evangelho segundo Lênin pudesse ser
difundido com tranquilidade, evangelho que doravante prega --- em
meio às ruínas provocadas por quatro anos de revolução --- a
intangibilidade da propriedade privada bem como a impiedosa liberdade
de concorrência.

Todavia, a indignação comunista contra a \textsc{nep} não exprimia senão a
confusão mental dos opositores de Lênin. Como explicar de outra maneira
que militantes, que sempre aprovaram as múltiplas acrobacias e efeitos
especiais políticos de seu chefe, indignassem"-se de repente ante
seu último salto perigoso, que constitui seu desfecho lógico? Os
comunistas devotos têm um grave problema: eles se agarram ao dogma da
Imaculada Conceição do Estado socialista, Estado tido como salvador do
mundo graças à Revolução. Mas a maioria dos dirigentes comunistas
jamais partilhou de tais ilusões. Lênin menos ainda que os outros.

Desde a minha primeira conversa com ele, compreendi que eu tratava com
um político astuto: ele sabia exatamente o que queria e parecia
decidido a não se prender a nenhum escrúpulo para chegar a seus
fins. Depois de tê"-lo ouvido falar em diversas ocasiões e ter lido
suas obras, creio que Lênin não se interessava de modo algum pela
revolução, e que o comunismo não era para ele senão um objetivo muito
distante. Em contrapartida, o Estado político centralizado era a
divindade de Lênin, a serviço da qual era preciso tudo sacrificar.
Alguém declarou um dia que Lênin estava pronto a sacrificar a revolução
para salvar a Rússia. Sua política, contudo, provou que ele estava
pronto para sacrificar simultaneamente a revolução e o país, ou, em
todo o caso, uma parte deste último, a fim de aplicar seu projeto
político no que restasse da Rússia.

Lênin era certamente o político mais flexível da História. Ele podia
ser ao mesmo tempo um super"-revolucionário, um homem de compromisso e
um conservador. Quando o grito de “Todo o poder aos sovietes” 
disseminou"-se como uma poderosa vaga por toda a Rússia, Lênin seguiu a
corrente. Quando os camponeses apoderaram"-se das terras e os
operários das fábricas, Lênin não apenas aprovou esses métodos de ação
direta como foi ainda mais longe. Lançou o famoso \textit{slogan}: “Expropriai
os expropriadores”, \textit{slogan} que semeou a confusão nos espíritos e causou
estragos irreparáveis ao ideal revolucionário. Nunca antes dele um
revolucionário havia interpretado a expropriação social como uma
simples transferência de riquezas de um grupo a outro de indivíduos.
No entanto, era exatamente o que significava o \textit{slogan} de Lênin. Os
ataques cegos e irresponsáveis, a acumulação das riquezas da antiga
burguesia nas mãos da nova burocracia soviética, as provocações
permanentes contra aqueles cujo único crime era seu antigo \textit{status}
social, tudo isso foi o resultado da “expropriação dos
expropriadores”.\footnote{
Esta frase de Lênin alude a uma célebre passagem do Livro \textsc{i} de 
\textit{O Capital}, em que Karl Marx descreve a concorrência encarniçada que
fazem entre si os capitalistas. Lênin retomou essa expressão por sua
conta num contexto histórico completamente diferente, aquele da
expropriação dos capitalistas pelos operários --- de fato, pelo Estado
bolchevique.} 
Toda a história subsequente da Revolução oferece um
caleidoscópio dos compromissos de Lênin e da traição de seus próprios
\textit{slogans}.

Os atos e os métodos dos bolcheviques desde a Revolução de Outubro podem
parecer contradizer a \textsc{nep}. No entanto, de fato, eles fazem parte dos
elos da corrente que iria forjar o governo todo"-poderoso centralizado
e cujo capitalismo de Estado era a expressão econômica. Lênin tinha uma
visão muito clara e uma vontade de ferro. Ele sabia como fazer com que
seus camaradas, tanto no interior da Rússia quanto no exterior, acreditassem
que seu projeto resultaria no autêntico socialismo e que seus métodos
eram revolucionários. Lênin desprezava de tal forma seus partidários
que ele nunca hesitou em lançar"-lhes suas quatro verdades no rosto.
“Só imbecis podem crer que é possível instaurar agora o comunismo na
Rússia”, respondeu aos bolcheviques que se opunham à \textsc{nep}.

De fato, Lênin tinha razão. Ele nunca tentou construir um autêntico
comunismo na Rússia, a menos que se considere que 33 níveis
de salários, um sistema diferenciado de rações alimentares, privilégios
assegurados para alguns e indiferença para a grande massa sejam
comunismo.

No começo da revolução, foi relativamente fácil ao Partido apoderar"-se
do poder. Todos os elementos revolucionários, entusiasmados pelas
promessas ultra"-revolucionárias dos bolcheviques, os ajudaram a tomar
o poder. Uma vez de posse do Estado, os comunistas iniciaram seu
processo de eliminação. Todos os partidos e grupos políticos que se
recusaram a submeter"-se à nova ditadura tiveram de partir. De início,
isso se aplicou aos anarquistas e aos socialistas"-revolucionários de
esquerda; depois, aos mencheviques e aos outros opositores de direita;
e, enfim, a todos aqueles que ousavam ter uma opinião pessoal. Todas as
organizações independentes conheceram o mesmo destino. Ou elas
subordinaram"-se às necessidades do novo Estado ou foram destruídas,
como foi o caso dos sovietes, dos sindicatos e das cooperativas --- os
três grandes pilares das esperanças revolucionárias.

Os sovietes apareceram pela primeira vez durante a Revolução de 1905.
Eles desempenharam um importante papel durante esse período breve mas
significativo. Embora a revolução tenha sido esmagada, a ideia dos
sovietes permaneceu enraizada no espírito e no coração das massas
russas. Desde a aurora que iluminou a Rússia em fevereiro de 1917, os
sovietes reapareceram e floresceram muito rápido. Para o povo, os
sovietes não lesavam de nenhum modo o espírito da revolução. Ao
contrário, a revolução iria encontrar sua expressão prática mais
elevada, mais livre nos sovietes. Eis por que os sovietes
disseminaram"-se tão espontânea e rapidamente em toda a Rússia. Os
bolcheviques compreenderam aonde iam as simpatias do povo e
juntaram"-se ao movimento. Mas quando eles controlaram o governo, os
comunistas deram"-se conta de que os sovietes ameaçavam a supremacia
do Estado.

Ao mesmo tempo, eles não podiam destruí"-los arbitrariamente sem minar
seu próprio prestígio no país e também no estrangeiro, visto
que eles aparecem como os promotores do sistema soviético. Começaram,
então, a privar gradualmente os sovietes de seus poderes para, enfim,
subordiná"-los às suas próprias necessidades.

Os sindicatos russos foram muito mais fáceis de castrar. No plano
numérico e do ponto de vista de sua fibra revolucionária, ainda estavam
em sua primeira infância. Declarando que a adesão aos sindicatos era
obrigatória, as organizações sindicais russas adquiriram uma certa
força numérica, mas seu espírito permaneceu o de uma criancinha. O
Estado comunista tornou"-se então a ama"-de"-leite dos sindicatos.
Em contrapartida, essas organizações serviram de lacaios ao Estado. “A
escola do comunismo”, como declarou Lênin durante a famosa controvérsia
relativa ao papel dos sindicatos. Ele tinha toda a razão. Todavia,
uma escola ultrapassada, onde o espírito da criança é acorrentado e
esmagado por seus professores. Em nenhum país do mundo os sindicatos
são tão submetidos à vontade e aos \textit{diktats} do Estado quanto na Rússia
bolchevique.

O destino das cooperativas é mais bem conhecido para que eu me estenda
em relação a esse assunto. Elas constituíam o laço mais essencial entre
as cidades e o campo. Elas traziam à revolução um meio popular e eficaz
de troca e distribuição, bem como um auxílio de valor incalculável
para reconstruir a Rússia. Os bolcheviques transformaram"-nas em
engrenagens da máquina governamental e elas perderam, simultaneamente,
sua utilidade e sua eficácia.

\sectionitem
Está doravante claro por que a Revolução Russa, dirigida pelo Partido
Comunista, fracassou. O poder político do Partido, organizado e
centralizado no Estado, buscou manter"-se por todos os meios à sua
disposição. As autoridades centrais tentaram canalizar à força as
atividades do povo em formas que correspondessem aos objetivos do Partido.

O único objetivo dos bolcheviques era reforçar o Estado e controlar
todas as atividades econômicas, políticas, sociais e, inclusive,
culturais. A revolução tinha um objetivo totalmente diferente,
já que, por natureza, ela encarnava a própria negação da autoridade
e da centralização. A revolução esforçou"-se para abrir campos cada
vez mais amplos à expressão do proletariado e multiplicar as
possibilidades de iniciativas individuais e coletivas. Os objetivos e
as tendências da revolução eram diametralmente opostos aos do
partido político dominante.

Os \textit{métodos} da revolução e do Estado são também diametralmente opostos. 
Os métodos da revolução são inspirados pelo próprio espírito da
revolução: a emancipação de todas as forças opressivas e limitadoras,
quer dizer, os \textit{princípios libertários}. Os métodos do Estado, ao
contrário --- do Estado bolchevique ou de qualquer governo --- são fundados
na \textit{coerção}, que pouco a pouco se transforma necessariamente numa
violência, numa opressão e num terror sistemáticos. Tais eram as duas
tendências em oposição: o Estado bolchevique e a revolução.
Tratava"-se de uma luta mortal. Tendo objetivos e métodos
contraditórios, essas duas tendências não podiam trabalhar no mesmo
sentido; o triunfo do Estado significava a derrota da revolução.

Seria um erro pensar que a revolução fracassou unicamente por causa da
personalidade dos bolcheviques. Fundamentalmente, a revolução fracassou
por causa da influência dos princípios e dos métodos do bolchevismo. O
espírito e os princípios autoritários do Estado sufocaram as aspirações
libertárias e libertadoras. Se um outro partido político tivesse
governado a Rússia, o resultado teria sido, no essencial, o mesmo. Não
foram tanto os bolcheviques que mataram a Revolução Russa, mas
principalmente sua ideologia. Tratava"-se de uma forma modificada de
marxismo, de um estatismo fanático. Só uma tal explicação das forças
subjacentes que esmagaram a revolução pode esclarecer esse
acontecimento que abalou o mundo. A Revolução Russa reflete, numa
pequena escala, a luta secular entre o princípio libertário e o
princípio autoritário. Com efeito, o que é o progresso senão a
aceitação mais geral dos princípios da liberdade contra aqueles da
coerção? A Revolução Russa representava um movimento libertário que foi
derrotado pelo Estado bolchevique, pela vitória temporária da ideia
reacionária, da ideia estatista.

Essa vitória deve"-se a várias causas. Abordei a maioria delas nos
capítulos precedentes deste livro. Mas a causa principal não era o
atraso industrial da Rússia, como escreveram inúmeros autores. Essa
causa era de ordem cultural e, se ela proporcionava ao povo russo
algumas vantagens sobre seus vizinhos mais sofisticados, ela também
tinha inconvenientes fatais. A Rússia era “culturalmente atrasada” na
medida em que não havia sido maculada pela corrupção política e
parlamentar. Por outro lado, faltava"-lhe experiência ante os jogos
políticos, e ela creu ingenuamente no poder miraculoso do partido que
falava mais alto e brandia mais promessas. Essa fé no poder do Estado
serviu para tornar o povo russo escravo do Partido Comunista, antes
mesmo que as grandes massas percebessem que se lhes haviam colocado o
jugo em torno do pescoço.

O princípio libertário foi poderoso nos primeiros dias da revolução; a
necessidade de liberdade de expressão revelava"-se imperiosa. Mas
quando a primeira onda de entusiasmo recuou para dar lugar às
dificuldades prosaicas da vida cotidiana, eram necessárias sólidas
convicções para manter viva a chama da liberdade. Só um punhado de
homens e mulheres, sobre o vasto território da Rússia, manteve essa
chama acesa: os anarquistas, cujo número era reduzido e cujos esforços,
ferozmente reprimidos sob o czar, não tiveram tempo de dar frutos.
O povo russo, que é, numa certa medida, anarquista por instinto, não
conhecia bem os verdadeiros princípios e métodos anarquistas para
aplicá"-los com eficácia. A maioria dos próprios anarquistas russos
encontrava"-se infelizmente enviscada em pequenos grupos e combates
individuais, em vez de um grande movimento social e coletivo. Um
historiador imparcial certamente admitirá um dia que os anarquistas
desempenharam um papel importantíssimo na revolução russa --- um papel
muito mais significativo e fecundo do que seu número relativamente
limitado podia fazê"-lo crer. Entretanto, a honestidade e a
sinceridade obrigam"-me a reconhecer que seu trabalho teria sido de um
valor prático infinitamente maior se eles estivessem mais bem
organizados e equipados para guiar as energias efervescentes do povo a
fim de reorganizar a vida social segundo fundamentos libertários.

Mas o fracasso dos anarquistas durante a Revolução Russa, no sentido em
que acabo de indicar, não significa absolutamente a derrota da ideia
libertária. Ao contrário, a Revolução Russa provou claramente que o
estatismo, o socialismo de Estado, em todas as suas manifestações
(econômicas, políticas, sociais e educativas), está inteira e
definitivamente condenado ao fracasso. Nunca na história a autoridade, o
governo, o Estado mostraram a que ponto eram, de fato, estáticos,
reacionários e, inclusive, contra"-revolucionários, encarnando a
própria antítese da revolução.

Como atesta a longa história do progresso, só o espírito e o método
libertários podem fazer avançar o homem em sua luta eterna para uma
vida melhor, mais agradável e mais livre. Aplicada às grandes
sublevações sociais que são as revoluções, essa tendência é tão
poderosa quanto no processo da evolução natural. O método autoritário
fracassou durante toda a história da humanidade e agora fracassou de
novo durante a Revolução Russa. Até aqui a inteligência humana não
descobriu outro princípio senão o princípio libertário, pois o homem
compreendeu uma grande verdade quando percebeu que a liberdade é a mãe
da ordem e não sua filha. Malgrado o que sustentam todas as teorias e
todos os partidos políticos, nenhuma revolução pode 
lograr êxito verdadeiro e duradouro se não se opõe ferozmente à tirania e à
centralização, se não luta com determinação para passar na
peneira todos os valores econômicos, sociais e culturais. Não se trata
de substituir um partido por um outro a fim de que ele controle o
governo, nem de camuflar um regime autocrático sob \textit{slogans} proletários,
nem mascarar a ditadura de uma nova classe sobre uma classe mais
antiga, nem se entregar a manobras quaisquer nos bastidores do teatro
político. Trata"-se, sim, de suprimir completamente todos os princípios
autoritários para servir a revolução.

No campo econômico, essa transformação deve ser efetuada pelas massas
operárias: elas têm a escolha entre um industrialismo estatista e o
anarco"-sindicalismo. No primeiro caso, o desenvolvimento construtivo
da nova estrutura social será também ameaçado pelo Estado político. Ele
constituirá um peso morto que vai onerar o crescimento das novas
formas de vida social. É por essa razão que só o sindicalismo não
basta, como bem sabem seus partidários. É só quando o espírito
libertário impregna as organizações econômicas dos trabalhadores que as
múltiplas energias criadoras do povo podem manifestar"-se livremente,
e que a revolução pode ser preservada e defendida. Só a liberdade de
iniciativa e a participação popular nos assuntos da revolução poderão
impedir os terríveis crimes cometidos na Rússia. Por exemplo, tendo em
vista que poços de petróleo erguiam"-se a apenas uma centena de
quilômetros de Petrogrado, essa cidade não teria sofrido o intenso frio
se as organizações econômicas dos trabalhadores de Petrogrado tivessem
podido exercer sua iniciativa em favor do bem comum. Os camponeses da
Ucrânia não teriam tido dificuldade em cultivar suas terras se tivessem
tido acesso aos instrumentos agrícolas estocados nos entrepostos de
Kharkov e dos outros centros industriais que esperavam as ordens de
Moscou para distribuí"-los. Esses poucos exemplos do estatismo e da
centralização bolcheviques deveriam alertar os trabalhadores da Europa
e da América contra os efeitos destruidores do estatismo.

Só o poder industrial das massas, que se exprime por meio de suas
associações libertárias, por meio do anarco"-sindicalismo, pode
organizar eficazmente a vida econômica e dar prosseguimento à produção.
Por outro lado, as cooperativas, trabalhando em harmonia com as
organizações operárias, servem de meios de distribuição e troca entre
as cidades e o campo e, ao mesmo tempo, constituem um laço fraterno
entre as massas operárias e camponesas.  Forma"-se, assim, um laço 
criador de apoio mútuo e serviços mutuais, e esse laço é a muralha mais
sólida da revolução --- bem mais eficaz que o trabalho forçado, o
Exército Vermelho ou o terror. É só desse modo que a revolução pode
agir como uma alavanca que acelera o advento de novas formas de vida
social e incita as massas a realizar coisas mais importantes.

Mas as organizações operárias libertárias e as cooperativas não são os
únicos meios de interação entre as fases complexas da vida social.
Também existem as forças culturais que, conquanto estejam estreitamente
ligadas às atividades econômicas, desempenham seu próprio papel. Na
Rússia, o Estado comunista tornou"-se o único árbitro de todas as
necessidades do corpo social. Disso resultou uma completa estagnação
cultural, e a paralisia de todos os esforços criativos. Se se quiser
evitar tal ruína no futuro, as forças culturais, ainda que permanecendo
enraizadas na economia, devem beneficiar"-se de um campo de atividade
independente e de uma liberdade de expressão total. Não é sua adesão ao
partido político dominante, mas sua devoção à revolução, seu saber, seu
talento e sobretudo seus impulsos criadores que permitirão determinar
sua aptidão ao trabalho cultural. Na Rússia, isso foi tornado
impossível quase desde o começo da Revolução de Outubro, porque
separaram violentamente as massas e a \textit{intelligentsia}. É verdade que o
culpado, de início, foi a própria \textit{intelligentsia}, sobretudo a \textit{intelligentsia}
técnica que, na Rússia, agarrou"-se com tenacidade à burguesia --- como
faz nos outros países. Incapaz de compreender o sentido dos
acontecimentos revolucionários, ela esforçou"-se para represar a vaga
revolucionária praticando a sabotagem. Todavia, na Rússia existia uma
outra fração da \textit{intelligentsia} que tinha um passado revolucionário
glorioso desde há um século. Essa fração conservara sua fé no povo,
embora não aceitasse sem reservas a nova ditadura. O erro fatal dos
bolcheviques foi não fazer qualquer distinção entre as duas
categorias. Eles combateram a sabotagem instaurando um terror cego e
sistemático contra toda a classe da \textit{intelligentsia} e lançaram uma
campanha de ódio ainda mais intensa do que a perseguição da própria
burguesia --- método que criou um abismo entre a \textit{intelligentsia} e o
proletariado e impediu todo trabalho construtivo.

Lênin foi o primeiro a se dar conta dessa falta criminosa. Ele ressaltou
que se tratava de um grave erro fazer crer aos operários que eles
podiam construir indústrias e engajar"-se num trabalho cultural sem a
ajuda e a cooperação da \textit{intelligentsia}. O proletariado não possuía nem
conhecimentos nem formação para conduzir a bom termo essas
tarefas, e era preciso tornar a dar à \textit{intelligentsia} a direção da vida
industrial. Mas o fato de ter reconhecido um erro não impediu Lênin e
seu Partido de cometerem imediatamente um outro. A \textit{intelligentsia}
técnica foi chamada ao socorro, mas de um modo que reforçou
simultaneamente a desintegração social e a hostilidade contra o regime.

Enquanto os operários continuavam a passar fome, os engenheiros, os
especialistas industriais e os técnicos receberam altos salários,
privilégios especiais e as melhores rações. Tornaram"-se os favoritos
do Estado e os novos vigilantes das massas reduzidas à escravidão.
Educadas durante anos na falsa ideia de que apenas os músculos contavam
para assegurar o sucesso da revolução e que só o trabalho manual era
produtivo, e por campanhas de ódio que denunciavam todos os
intelectuais como contra"-revolucionários e especuladores, as massas
não puderam evidentemente fazer a paz com aqueles que lhes ensinaram a
desprezar e a suspeitar.

Infelizmente a Rússia não é o único país em que predomina essa atitude
hostil do proletariado contra a \textit{intelligentsia}. Em toda a parte, os
políticos demagogos jogam com a ignorância das massas, ensinam"-lhes
que a educação e a cultura são preconceitos burgueses, que os operários
podem dispensar isso e que são capazes de reconstruir sozinhos a
sociedade. A Revolução Russa mostrou muito claramente, contudo, que o
cérebro e o músculo são indispensáveis para regenerar a sociedade. O
trabalho intelectual e o trabalho manual cooperam estreitamente no
corpo social, como o cérebro e a mão no corpo humano. Um não pode
funcionar sem o outro.

É verdade que a maioria dos intelectuais considera"-se uma classe
à parte, superior aos operários; todavia, em toda parte as condições
sociais minam rapidamente o pedestal da \textit{intelligentsia}. Os intelectuais
são forçados a admitir que eles também são proletários, e que são até
mesmo mais dependentes dos senhores da economia que os trabalhadores
manuais.

Contrariamente ao proletário manual que trabalha com sua força física,
que pode reunir suas ferramentas e percorrer o mundo com vistas a
melhorar sua situação humilhante, os proletários intelectuais são muito
mais solidamente enraizados em seu meio social específico e não podem
facilmente mudar de ofício ou de modo de vida. Eis por que é essencial
fazer compreender aos operários que os intelectuais estão sendo
rapidamente proletarizados --- o que cria um laço entre eles. Se o mundo
ocidental quiser tirar proveito das lições da Rússia, ele deve pôr um
termo à adulação demagógica das massas bem como à hostilidade cega
contra a \textit{intelligentsia}. Isso não significa, contudo, que os operários
devam recolocar seu destino nas mãos dos intelectuais. Ao contrário, as
massas devem começar imediatamente a preparar"-se, a equipar"-se para
a grande tarefa que a revolução exigirá delas. Deverão adquirir o saber
e a habilidade técnica necessária para gerir e dirigir os mecanismos
complexos das estruturas industriais e sociais de seus países
respectivos. Entretanto, mesmo que exibam todas as suas capacidades, os
operários precisarão da cooperação dos especialistas e dos
intelectuais. Por sua vez, estes últimos devem também compreender que
seus verdadeiros interesses são idênticos aos interesses das massas. Uma vez
que as duas forças sociais aprendam a fundir"-se num todo harmônico,
os aspectos trágicos da revolução russa serão em grande parte
eliminados. Ninguém será fuzilado porque “estudou”. O cientista, o
engenheiro, o especialista, o pesquisador, o professor e o artista
criador, bem como o marceneiro, o maquinista e todos os outros
trabalhadores, fazem integralmente parte da força coletiva que permitirá
à revolução construir o novo edifício social. Ela não empregará o ódio,
mas a unidade; não a hostilidade, mas a camaradagem; não o pelotão de
execução, mas a simpatia --- tais as lições a tirar do grande
fracasso russo tanto pela \textit{intelligentsia} quanto pelos operários. Todos
devem aprender o valor do apoio mútuo e da cooperação libertária.
Entretanto, cada um deve ser capaz de permanecer independente em sua
esfera particular e em harmonia com o melhor que puder proporcionar à
sociedade. É só assim que o trabalho produtivo e os esforços
educativos e culturais exprimir"-se"-ão em formas cada vez mais novas
e mais ricas. Tal é para mim a lição essencial, universal, que me
ensinou a Revolução Russa.


\sectionitem
Tentei explicar por que os princípios, os métodos e as táticas
bolcheviques fracassaram, e por que esses mesmos princípios e métodos
fracassarão amanhã em qualquer outro país, mesmo o mais
industrializado. Mostrei igualmente que não foi só o bolchevismo que
fracassou, mas o próprio marxismo. A experiência da Revolução Russa
demonstrou a falência do estatismo, do princípio autoritário. 

Se eu tivesse de resumir todo o meu pensamento numa única frase, eu diria:
por natureza, o Estado tem a tendência de concentrar, reduzir e controlar
todas as atividades sociais; ao contrário, a revolução tem a vocação de
crescer, ampliar"-se e difundir"-se em círculos cada vez mais largos.
Em outros termos, o Estado é institucional e estático, enquanto a
revolução é fluida e dinâmica. Essas duas tendências são incompatíveis e
condenadas a destruir"-se mutuamente. O estatismo matou a Revolução
Russa e desempenhará o mesmo papel nas revoluções futuras, \textit{a menos
que a ideia libertária o derrote}. 

Mas devo ir mais longe. Não são apenas o bolchevismo, o marxismo e o
estatismo que são fatais à revolução e ao progresso vital da
humanidade. A principal causa da derrota da Revolução Russa é muito
mais profunda. Ela reside na própria concepção socialista da revolução.

A concepção dominante, a mais disseminada, da revolução ---
particularmente entre os socialistas --- é que a revolução provoca uma
violenta mudança das condições sociais durante a qual uma classe
social, a classe operária, torna"-se dominante e triunfa sobre outra
classe, a classe capitalista. Essa concepção é centrada sobre uma
mudança puramente material e, assim, implica sobretudo manobras
políticas de bastidores e remendos institucionais. A ditadura da
burguesia é substituída pela “ditadura do proletariado” --- ou aquela de
sua “vanguarda”, o Partido Comunista. Lênin toma o lugar dos Romanov, o
gabinete imperial é rebatizado de Conselho dos Comissários do Povo,
Trotski é nomeado ministro da Guerra e um trabalhador torna"-se
governador"-militar geral de Moscou. Eis a que se reduz,
essencialmente, a concepção bolchevique da revolução, ao menos quando é
posta em prática. Excetuando pequenos detalhes, é a mesma ideia de
revolução que partilham os outros partidos socialistas.

Essa concepção é, por natureza, falsa e está destinada ao fracasso. A
revolução é, certamente, um processo violento. Todavia, se ela só
resulta numa nova ditadura, numa simples mudança de nomes e de
personalidades no poder, então não tem qualquer utilidade. Um resultado
tão limitado não justifica todos os combates, sacrifícios, perdas
de vidas humanas e atentados aos valores culturais provocados por
todas as revoluções. Se tal revolução proporcionasse um aumento do
bem"-estar social (o que não foi o caso na Rússia), nem assim valeria
o terrível preço a pagar; pode"-se melhorar a sociedade sem recorrer a
uma revolução sangrenta. O objetivo da revolução não é aplicar alguns
paliativos nem algumas reforminhas.

A experiência da Revolução Russa reforçou poderosamente minha convicção
de que a grande missão da revolução, da \textit{revolução social}, é uma mudança
fundamental dos valores sociais e humanos. Os valores humanos são ainda
mais importantes porque são a base de todos os valores sociais. Nossas
instituições e nossas condições sociais repousam sobre ideias
profundamente ancoradas. Se essas condições são mudadas sem tocar nas
ideias e valores subjacentes, tratar"-se"-á, então, de uma
transformação superficial, que não pode ser duradoura nem conduzir a uma
melhora real. Trata"-se apenas de uma mudança de forma, não de
substância, como a Rússia demonstrou tragicamente.

É simultaneamente o grande fracasso e a grande tragédia da Revolução
Russa: ela tentou (sob a direção do partido político dominante) só
mudar as instituições e as condições materiais, ignorando totalmente os
valores humanos e sociais que uma revolução implica. Pior ainda, em sua
louca paixão pelo poder, o Estado comunista inclusive reforçou e
desenvolveu as próprias ideias e concepções que a revolução veio
destruir. O Estado apoiou e encorajou os piores comportamentos
anti"-sociais e sistematicamente sufocou o desenvolvimento dos novos
valores revolucionários. O senso de justiça e de igualdade, o amor pela
liberdade e pela fraternidade humana --- pilares de uma autêntica
regeneração da sociedade --- o Estado comunista os combateu a ponto de
os aniquilar. O sentimento instintivo da equidade foi ironizado como
uma manifestação de sentimentalismo e fraqueza; a liberdade e a
dignidade humanas tornaram"-se superstições burguesas; o caráter
sagrado da vida, que é a própria base da reconstrução social, foi
condenado como a"-revolucionário, quase contra"-revolucionário. Essa
terrível perversão dos valores fundamentais trazia em si mesma o germe
da destruição. Se acrescentarmos a isso a concepção segundo a qual a
revolução constituía apenas um meio de apoderar"-se do poder político,
era inevitável que todos os valores revolucionários fossem subordinados
às necessidades do Estado socialista; pior, que elas fossem exploradas
para aumentar a segurança do novo poder governamental. “A razão de
Estado”, camuflada sob a máscara dos “interesses da Revolução e do
Povo”, tornou"-se o único critério da ação e, também, dos
sentimentos. A violência, inevitabilidade trágica de sublevações
revolucionárias, tornou"-se um costume estabelecido, um hábito, e foi
gabada como uma instituição “ideal”. Zinoviev não canonizou Dzerjinski,
o chefe da sanguinária Tcheka, apresentando"-o como “santo da
Revolução”? O Estado não prestou as maiores honras a Uritski, o
fundador e o chefe sádico da Tcheka de Petrogrado?

Essa perversão dos valores éticos rapidamente cristalizou"-se no \textit{slogan}
onipresente do Partido Comunista: \textit{o fim justifica todos os meios}. No
passado, a Inquisição e os jesuítas adotaram esse \textit{slogan} e
subordinaram"-lhe toda moralidade. Essa máxima vingou"-se dos
jesuítas como ela vingou"-se da Revolução Russa. Esse preceito só fez
encorajar a mentira, o ludíbrio, a hipocrisia, a traição e o
assassínio, público e secreto. Aqueles que se interessam pela
psicologia social deveriam perguntar"-se por que dois movimentos, tão
separados no tempo e de ideias tão diferentes quanto o jesuitismo e o
bolchevismo, \textit{chegaram exatamente aos mesmos resultados} aplicando esse 
princípio. O paralelo histórico, passado quase despercebido até aqui,
contém uma lição fundamental para todas as revoluções futuras e para o
futuro da humanidade.

Nada é mais falso do que crer que os objetivos e as intenções são uma
coisa, os métodos e as táticas outra coisa. Essa concepção ameaça
gravemente a regeneração social. Toda a experiência da humanidade
ensina"-nos que os métodos e os meios não podem ser separados do fim
último. Os meios empregados tornam"-se, por intermédio dos hábitos
individuais e das práticas sociais, parte integrante do objetivo final;
eles o influenciam, modificam"-no, até que os fins e os meios acabam
por se tornar idênticos. Desde o primeiro dia de meu retorno à Rússia
eu o senti, inicialmente de maneira vaga, depois de modo cada vez mais claro e
consciente. Os grandes objetivos que inspiravam a Revolução foram
tão obscurecidos pelos métodos utilizados pelo poder político dominante
que se tornou difícil distinguir entre os meios temporários e o
objetivo final. No plano psicológico e social, os meios influenciam
necessariamente os objetivos e os modificam. Toda a história da
humanidade prova que, tão logo nos privamos dos métodos inspirados por
conceitos éticos, afundamos na desmoralização mais aguda. Essa é a
autêntica tragédia da filosofia bolchevique aplicada à Revolução Russa.
Esperemos que saibamos tirar as lições disso.

Nenhuma revolução jamais se tornará um fator de liberação se os \textit{meios}
utilizados para aprofundá"-la não estiverem em harmonia, em seu
espírito e sua tendência, com os \textit{objetivos} a alcançar. A revolução
representa a negação do existente, um protesto violento contra a
desumanidade do homem em relação ao homem e aos milhares de escravidões
que ela implica. A revolução destrói os valores dominantes, sobre os
quais foi construído um sistema complexo de injustiça e opressão
que repousa sobre a ignorância e a brutalidade. A revolução é o arauto
de \textit{novos valores}, pois ela desemboca na transformação das relações
fundamentais entre os homens, assim como entre os homens e a sociedade.
A revolução não se contenta em sanar alguns males, aplicar alguns
emplastros, mudar as formas e as instituições, redistribuir o
bem"-estar social. É verdade, ela faz tudo isso, mas representa mais,
muito mais. Ela é, de início e antes de tudo, o \textit{vetor} de uma mudança
radical, \textit{portador de novos valores}. Ela \textit{ensina uma nova ética} que
inspira o homem, inculcando nele uma nova concepção da vida e das
relações sociais. A revolução desencadeia uma regeneração mental e
espiritual.

Seu primeiro preceito ético é a identidade entre os meios utilizados e
os objetivos buscados. O objetivo último de toda mudança social
revolucionária é estabelecer o caráter sagrado da vida humana, a
dignidade do homem, o direito de cada ser humano à liberdade e ao
bem"-estar. Se esse não é o objetivo essencial da revolução, então as
mudanças sociais violentas não têm qualquer justificação. Isso porque
transformações sociais \textit{externas} podem ser, e foram, realizadas no 
âmbito do processo normal da evolução. A revolução, ao contrário, não
significa apenas uma mudança \textit{externa}, mas uma mudança \textit{interna}, 
fundamental, essencial. Essa mudança interna das concepções e das
ideias difunde"-se em camadas sociais cada vez mais amplas, para
enfim culminar numa violenta sublevação denominada revolução. Tal
apogeu pode inverter a mudança radical de valores, voltar"-se contra
ela, traí"-la? Foi o que se produziu na Rússia. A revolução deve
acelerar e aprofundar o processo do qual ela é a expressão cumulativa;
sua principal missão é inspirá"-lo, carregá"-lo para as mais elevadas
alturas, dar"-lhe o máximo de espaço para sua livre expressão. É só
desse modo que a revolução é fiel a si mesma.

Na prática, isso significa que a pretensa “etapa transitória” deve
introduzir novas condições sociais. Ela representa o marco de uma \textit{nova
vida}, de uma nova \textit{casa do homem e da humanidade}. Ela deve estar animada
pelo espírito da nova vida, em harmonia com a construção do novo
edifício.

Hoje engendra amanhã. O presente projeta sua sombra muito longe no
futuro. Essa é a lei da vida, quer se trate do indivíduo, quer se trate
da sociedade. A revolução que se livra de seu valores éticos deita as
premissas da injustiça, do ludíbrio e da opressão na sociedade futura.
Os meios utilizados para preparar o futuro tornam"-se sua pedra
angular. Basta observar a trágica condição atual da Rússia. Os métodos
da centralização estatista paralisaram a iniciativa e o esforço
individuais; a tirania da ditadura apavorou o povo, mergulhou"-o numa
submissão servil e apagou totalmente a chama da liberdade; o terror
organizado corrompeu e brutalizou as massas, sufocando todas as
aspirações idealistas; o assassinato institucionalizado depreciou o
valor da vida humana; todas as noções de dignidade humana, de valor da
vida foram eliminadas; a coerção tornou cada esforço mais duro,
transformando o trabalho numa punição; a vida social doravante se reduz
a uma sucessão de ludíbrios mútuos; os instintos mais vis e mais
brutais do homem despertaram novamente. Triste herança para começar uma
nova vida fundada na liberdade e na fraternidade.

Nunca ressaltaremos suficientemente que a revolução não serve para nada
se ela não é inspirada por seu ideal último. Os métodos revolucionários
devem estar em harmonia com os objetivos revolucionários. Os meios
utilizados para aprofundar a revolução devem corresponder a seus
objetivos. Em outros termos, os valores éticos que a revolução
infundirá na nova sociedade devem ser disseminados pelas atividades
revolucionárias do “período de transição”. Este último pode facilitar a
passagem a uma vida melhor, mas somente sob a condição de que seja
construído com os mesmos materiais da nova vida que se quer
construir. A revolução é o espelho dos dias que se seguem; ela é a
criança que anuncia o Homem de amanhã.

\chapter{O comunismo não existe na rússia}

\vspace*{-.8em}

\section*{bolchevismo = comunismo?}
A palavra comunismo está agora em todos os lábios. Alguns falam dele com
o entusiasmo exagerado dos neófitos, outros o temem e o condenam como
uma ameaça social. Mas estou quase segura de que nem seus admiradores
--- a grande maioria deles --- nem aqueles que o denunciam têm uma ideia
muito clara do que é verdadeiramente o “comunismo” ao molho
bolchevique.

Se se quiser dar uma definição dele muito geral, o comunismo representa
um ideal de igualdade e fraternidade humana: ele considera a
exploração do homem pelo homem como a fonte de toda escravidão e de
toda opressão. A desigualdade econômica conduz à injustiça social e é
inimiga do progresso moral e intelectual.

O comunismo visa criar uma sociedade em que as classes serão abolidas,
em que será instaurada a propriedade comum dos meios de produção e
distribuição. O homem não poderá fruir da liberdade, da paz e do
bem"-estar senão numa comunidade sem classes e solidária.

Meu objetivo inicial, ao escrever este artigo, era comparar o ideal
comunista com o modo como ele é aplicado na \textsc{urss}, mas eu me dei
conta de que se tratava de uma tarefa impossível. Na realidade, o
comunismo não existe na Rússia. Nem mesmo um único princípio comunista,
sequer um único elemento de seus ensinamentos é aplicado pelo
Partido Comunista nesse país.

Aos olhos de alguns, minha posição parecerá totalmente absurda; outros
pensarão que eu exagero grosseiramente. Entretanto, estou certa de que
um exame objetivo da situação russa atual convencerá o leitor honesto
de que eu digo a verdade.

Interessemo"-nos, de início, pela ideia fundamental que subtende o
pretenso “comunismo” dos bolcheviques. Sua ideologia abertamente
centralista, autoritária, é fundada quase que exclusivamente na coerção e
na violência estatistas. Longe de ser fundado na livre-associação,
trata"-se de um comunismo estatista obrigatório. Deve"-se reter
isso na memória se se quiser compreender o método utilizado pelo Estado
soviético para aplicar seus projetos e dar"-lhes uma pequena aparência
“comunista”.

\section*{nacionalização ou socialização?}
A primeira condição para que se realize o comunismo é a \textit{socialização} das 
terras, dos instrumentos de produção e da distribuição. Socializa"-se
a terra e as máquinas para que elas sejam utilizadas por indivíduos ou
grupos, em função de suas necessidades. Na Rússia, a terra e os meios
de produção não são socializados, mas \textit{nacionalizados}. O termo 
“nacionalização” é enganador, pois essa palavra não tem qualquer
conteúdo. Na realidade, a riqueza nacional não existe. A “nação” é uma
entidade demasiado abstrata para “possuir” o que quer que seja. Ou a
propriedade é individual, ou ela é partilhada por um grupo de
indivíduos; ela repousa sempre sobre uma realidade quantitativamente
definível. 

Quando um bem não pertence nem a um indivíduo, nem a um grupo, ele é
nacionalizado ou socializado. Se ele é nacionalizado, pertence ao
Estado; de fato, o governo tem seu controle e pode dele dispor segundo
seu bel"-prazer. Mas se um bem é socializado, cada indivíduo tem
livremente acesso a ele e pode utilizá"-lo sem a ingerência de quem
quer que seja.

Na Rússia, nem a terra, nem a produção, nem a dis\-tribuição são
socializadas. Tudo é nacionalizado e pertence ao governo, exatamente
como o correio nos Estados Unidos ou as ferrovias na Alemanha ou em
outros países europeus. Esse estatuto não tem absolutamente nada de
comunista.

A estrutura econômica da \textsc{urss}. não é mais comunista que a terra ou os
meios de produção. Todas as fontes de existência são propriedades do
governo central; este dispõe do monopólio absoluto do comércio
exterior; as gráficas lhe pertencem; cada livro, cada folha de papel
impressa é uma publicação oficial. Em resumo, o país e tudo o que ele
contém são a propriedade do Estado, como ocorria antes, no tempo dos
czares. Os poucos bens que não são nacionalizados, como certas
casas velhas e deterioradas em Moscou, por exemplo, ou pequenas lojas pobres
que dispõem de um miserável estoque de cosméticos, são os únicos
tolerados. A qualquer momento o governo pode exercer seu direito
indiscutível de apoderar"-se deles por simples decreto.

Tal situação diz respeito ao capitalismo de Estado, e seria
extravagante detectar nisso o que quer que seja de comunismo.

\section*{produção e consumo}

Voltemo"-nos agora para a produção e o consumo, alavancas de toda
existência. Talvez descubramos aí uma dose de comunismo, que
justificaria a utilização do termo “comunista” para descrever a
vida na \textsc{urss}, ao menos em certa escala.

Eu já observei que a terra e os instrumentos de produção são propriedades
do Estado. Os métodos de produção e as quantidades que devem ser
produzidas por cada indústria, cada oficina, cada fábrica, cada
usina, são determinadas pelo Estado, pelo governo central --- sediado em
Moscou --- por intermédio de seus diferentes órgãos.

A \textsc{urss} é um país muito extenso, que cobre aproximadamente um sexto da
superfície da Terra. Abrigando uma população heterogênea de 165 milhões
de habitantes, ela comporta várias grandes repúblicas, diferentes
etnias e nacionalidades, e cada região tem suas necessidades e
interesses particulares. Certamente, o planejamento industrial e
econômico tem uma importância vital para o bem"-estar de uma
comunidade.

O autêntico comunismo --- a igualdade econômica entre os homens e entre as
comunidades --- exige que cada comunidade organize o melhor e mais eficaz
planejamento, fundamentando"-se em suas necessidades e possibilidades
locais. Tal planejamento se fundamenta na completa liberdade de cada
comunidade de produzir e dispor de seus produtos segundo suas
necessidades, necessidades que ela própria deve fixar; cada comunidade
deve trocar seu excedente com outras comunidades independentes sem que
qualquer autoridade externa intervenha.

Essa é a natureza fundamental do comunismo no plano político e econômico.
Isso não pode funcionar nem é possível sobre outras bases. O
comunismo é necessariamente libertário. Anarquista.

Não se percebe o mínimo vestígio de comunismo --- do mínimo
comunismo --- na Rússia soviética. Na realidade, a única alusão a tal
organização é considerada ali como um crime, e toda tentativa de
pô"-la em prática seria punida com a morte.

O planejamento industrial, bem como todos os processos de produção e
distribuição, encontra"-se nas mãos do governo central. O Conselho
econômico supremo é submetido unicamente à autoridade do Partido
Comunista.

Ele é totalmente independente da vontade ou dos desejos das pessoas que
formam a União das Repúblicas Socialistas Soviéticas. Seu trabalho está
condicionado pelas políticas e pelas decisões do Kremlin. Eis por que a
Rússia soviética exportou enormes quantidades de trigo e outros cereais,
enquanto vastas regiões no sul e no sudeste da Rússia eram golpeadas
pela penúria, de tal modo que mais de dois milhões de pessoas morreram
de fome em 1932 e 1933.

A “razão de Estado” é inteiramente responsável por essa situação. Essa
expressão sempre serviu para mascarar a tirania, a exploração e a
determinação dos dirigentes em prolongar e perpetuar sua dominação.

De passagem, assinalarei que, malgrado a penúria que afetou todo o país
e a falta dos recursos mais elementares para viver na Rússia, o
primeiro plano quinquenal visava unicamente desenvolver a indústria
pesada, indústria que serve ou pode servir a objetivos \textit{militares}. 

O mesmo acontece com a distribuição e todas as outras formas de
atividade. Não apenas os burgos e as cidades, mas todas as partes
constitutivas da União Soviética são privadas de existência
independente. Visto que elas são apenas simples vassalos de Moscou,
suas atividades econômicas, sociais e culturais são conhecidas,
planejadas e severamente controladas pela “ditadura do proletariado” em
Moscou. Pior: a vida de cada localidade, e até mesmo de cada indivíduo,
nas pretensas repúblicas “socialistas” é gerida nos mínimos detalhes
pela “linha geral” fixada pelo “centro”. Em outros termos, pelo Comitê
Central e pelo Bureau Político do Partido, ambos controlados com mão de
ferro por um único homem. Como alguns podem chamar de comunismo essa
ditadura, essa autocracia mais poderosa e mais absoluta do que qualquer
czar, já ultrapassa a minha imaginação.
\section*{a vida cotidiana na \textsc{urss}} 
Examinemos agora como o “comunismo” bolchevique influencia a vida das
massas e do indivíduo.

Alguns ingênuos creem que ao menos algumas características do comunismo
foram introduzidas na vida do povo russo. Eu gostaria que isso fosse
verdade, pois seria uma garantia de esperança, a promessa de um
desenvolvimento potencial nessa direção. Infelizmente, em nenhum dos
aspectos da vida soviética, nem nas relações sociais, nem nas relações
individuais, jamais se tentou aplicar os princípios comunistas sob uma
forma ou outra. Como ressaltei anteriormente, o próprio fato de sugerir
que o comunismo possa ser livre e voluntário é tabu na Rússia. Tal
concepção é considerada contra"-revolucionária e diz respeito à
alta traição contra o infalível Stálin e o sacrossanto Partido
“comunista”.

Coloquemos de lado, por um instante, o comunismo libertário, anarquista.
Não encontramos sequer o mínimo vestígio, na Rússia soviética, de
uma manifestação qualquer de comunismo de Estado, ainda que sob uma
forma autoritária, como o demonstra a observação dos fatos da vida
cotidiana nesse país.

A essência do comunismo, mesmo de tipo coercitivo, é a ausência de
classes sociais. A introdução da igualdade econômica constitui a
primeira etapa. Tal foi a base de todas as filosofias comunistas, ainda
que difiram entre si em relação a outros aspectos. Seu objetivo comum
era assegurar a justiça social; todas afirmavam que não se podia chegar
à justiça social sem estabelecer a igualdade econômica. Mesmo Platão,
que previa a existência de diferentes categorias intelectuais e morais
em sua República, havia pronunciado"-se em favor da igualdade
econômica absoluta, pois as classes dirigentes não deviam fruir de
direitos ou privilégios mais importantes do que aqueles situados
na parte de baixo da escala social.

A Rússia soviética representa o caso exatamente oposto. O bolchevismo
não aboliu as classes na Rússia: apenas inverteu suas relações
anteriores. De fato, ele até mesmo agravou as divisões sociais que
existiam antes da Revolução.

\section*{rações e privilégios}
Quando retornei à Rússia em janeiro de 1920, descobri inúmeras
categorias econômicas, fundadas nas rações alimentares distribuídas
pelo governo. O marinheiro recebia a melhor ração, superior em
qualidade, quantidade e variedade aos alimentos que o resto da
população comia. Era a aristocracia da Revolução; no plano econômico e
social, todos consideravam que ele pertencia às novas classes
privilegiadas. Atrás dele vinha o soldado, o homem do Exército
Vermelho, que recebia uma ração bem menor, e menos pão. Após o soldado
encontrava"-se o operário das indústrias de armamentos;
depois os outros operários, eles próprios divididos em operários
qualificados, artesãos, sem qualificação etc.

Cada categoria recebia um pouco menos de pão, banha, açúcar, tabaco e
outros produtos (quando havia). Os membros da antiga burguesia, classe
oficialmente abolida e expropriada, pertenciam à última categoria
econômica e não recebiam praticamente nada. A maioria deles não podia
ter trabalho e moradia, e ninguém se preocupava com a maneira como eles
sobreviveriam, sem se pôr a roubar ou a juntar"-se aos exércitos
contra"-revolucionários ou aos bandos de ladrões.

O proprietário de uma carteira vermelha, membro do Partido Comunista,
ocupava um lugar situado acima de todos aqueles de quem acabo de falar.
Ele beneficiava"-se de uma ração especial, podendo comer na \textit{stolovaya} 
(cantina) do Partido, e tinha o direito, sobretudo se estivesse
recomendado por um responsável mais elevado, a roupas de baixo quentes,
botas de couro, um casaco de pele e outros artigos de valor. Os
bolcheviques mais eminentes dispunham de seus próprios restaurantes,
aos quais os militantes de base não tinham acesso. Em Smolny, que
abrigava, então, o quartel"-general do governo de Petrogrado, existiam
dois restaurantes, um para os comunistas mais bem situados, um outro
para os bolcheviques menos importantes. Zinoviev, então presidente do
soviete de Petrogrado e autêntico autocrata do Distrito do Norte, bem
como outros membros do governo, faziam suas refeições “em casa”, no
Astória, outrora o melhor hotel da cidade, tornado a primeira Casa do
Soviete, onde viviam com suas famílias.

Mais tarde, constatei uma situação idêntica em Moscou, Kharkov, Kiev,
Odessa --- em toda a Rússia soviética.

Eis o que era o “comunismo” bolchevique. Esse sistema teve consequências
desastrosas: suscitou a insatisfação, o ressentimento e a hostilidade
em todo o país; provocou sabotagens nas fábricas e no campo, greves e
revoltas incessantes. “O homem não vive só de pão”, segundo parece. É
verdade, mas ele morre se não tiver nada para comer. Para o homem da
rua, para as massas russas que haviam vertido seu sangue esperando
libertar o país, o sistema diferenciado de rações simbolizava o novo
regime. O bolchevismo representava para eles uma enorme mentira, pois
ele não mantivera sua promessa de instaurar a liberdade; com efeito,
para eles, liberdade significava justiça social e igualdade
econômica. O instinto das massas raramente as engana; nesse caso,
revelou"-se profético. Por que se surpreender, consequentemente, com o
fato de que o entusiasmo geral pela revolução tenha muito rápido se
transformado em decepção e amargura, hostilidade e ódio? Quantas vezes
operários russos queixaram"-se a mim: “Para nós, é indiferente trabalhar
duro e passar fome. É a injustiça que nos revolta. Se um país é pobre,
se há pouco pão, então partilhemos entre todos o pouco que há, mas
partilhemos de modo equitativo. Atualmente, a situação é a mesma que
antes da revolução; alguns recebem muito, outros menos, e outros
absolutamente nada.''

A desigualdade e os privilégios criados pelos bolcheviques tiveram
rapidamente consequências inevitáveis: esse sistema aprofundou os
antagonismos sociais, afastou as massas da Revolução, paralisou seu
interesse por ela, sufocou suas energias e contribuiu para aniquilar
todos os projetos revolucionários.

Esse sistema não igualitário, fundado em privilégios, reforçou"-se,
aperfeiçoou"-se e grassa ainda hoje.

A Revolução Russa era, no sentido mais profundo, uma transformação
social: sua tendência fundamental era libertária, seu objetivo
essencial a igualdade econômica e social. Bem antes da revolução de
outubro"-novembro de 1917, o proletariado urbano havia começado a
apoderar"-se das oficinas, das fábricas e das usinas, enquanto os
camponeses expropriavam as grandes propriedades e cultivavam as terras
em comum. O desenvolvimento contínuo da revolução num sentido comunista
dependia da unidade das forças revolucionárias e da iniciativa direta,
criadora, das massas laboriosas. O povo estava entusiasmado pelos
grandes objetivos que ele tinha à sua frente; aplicava"-se com paixão e
energia para reconstruir uma nova sociedade. Com efeito, só aqueles
que haviam sido explorados durante séculos eram capazes de encontrar
livremente o caminho rumo a uma sociedade nova, regenerada.

Mas os dogmas bolcheviques e o estatismo “comunista” constituíram um
obstáculo fatal para as atividades criadoras do povo. A característica
fundamental da psicologia bolchevique era sua desconfiança em relação
às massas. As teorias marxistas, que queriam exclusivamente concentrar
o poder nas mãos do Partido, resultaram rapidamente no desaparecimento
de toda colaboração entre os revolucionários, na eliminação brutal e
arbitrária dos outros partidos e movimentos políticos. A política
bolchevique resultou na eliminação do mínimo sinal de descontentamento,
no sufocamento das críticas e das opiniões independentes, bem como no
esmagamento dos esforços ou das iniciativas populares. A centralização
de todos os meios de produção nas mãos da ditadura comunista
desfavorece as atividades econômicas e industriais do país. As massas
não puderam modelar a política da Revolução, nem tomar parte da
administração de seus próprios assuntos. Os sindicatos eram
estatizados e contentavam"-se em transmitir as ordens do governo. As
cooperativas populares --- instrumento essencial da solidariedade ativa e
do apoio mútuo entre as cidades e o campo --- foram liquidadas, os sovietes de
camponeses e operários esvaziados de seu conteúdo e transformados em
comitês de sustentação ao regime. O governo pôs"-se a controlar todas
as áreas da vida social. Criou"-se uma máquina burocrática ineficaz,
corrupta e brutal. Ao afastar"-se do povo, a revolução condenou"-se à
morte; acima de todos, pairava a temível espada do terror bolchevique.

Tal era o comunismo dos “bolcheviques” durante as primeiras etapas da
Revolução. Todos sabem que ele provocou a completa paralisia da
indústria, da agricultura e dos transportes. Era o período do
“comunismo de guerra”, da conscrição camponesa e operária, da
destruição total dos vilarejos camponeses pela artilharia bolchevique ---
todas essas medidas sociais e econômicas que resultaram na terrível
penúria de 1921.

\section*{o que mudou desde 1921?}
E o que acontece hoje? O “comunismo” não mudou de natureza? Ele é
verdadeiramente diferente do “comunismo” de 1921? Para minha tristeza,
sou obrigada a afirmar que, malgrado todas as decisões políticas e
medidas econômicas ruidosamente anunciadas, o bolchevismo “comunista” é
fundamentalmente o mesmo que em 1921.

Hoje o campesinato, na Rússia soviética, está inteiramente destituído
de sua terra. Os \textit{sovkhozes} são fazendas governamentais nas quais os
camponeses trabalham em troca de um salário, exatamente como o operário
em uma fábrica. Os bolcheviques chamam isso de “industrialização” da
agricultura, de “transformação do camponês em proletário”. No \textit{kolkhoze},
a terra pertence apenas nominalmente ao vilarejo. De fato, ela é
propriedade do Estado. Este pode a qualquer momento --- e o faz bem
amiúde --- requisitar os membros do \textit{kolkhoze} e ordenar"-lhes que vão
trabalhar em outras regiões ou exilá"-los em longínquos vilarejos
porque não obedeceram às suas ordens. Os \textit{kolkhozes} são geridos
coletivamente, mas o controle governamental é tanto que, de fato, a terra é que foi
expropriada pelo Estado. Este fixa os impostos que ele quer,
decide o preço dos cereais ou dos outros produtos que adquire. Nem o
camponês individual nem o vilarejo soviético podem dizer algo. Impondo
inúmeros saques e empréstimos estatistas compulsórios, o governo
apropria"-se dos produtos dos \textit{kolkhozes}. Arroga"-se igualmente o
direito, invocando delitos reais ou supostos, de puni"-los,
requisitando todos os seus cereais.

Concordamos em dizer que a terrível fome que se abateu em 1921 foi
provocada sobretudo pela \textit{razverstka}, a expropriação brutal em voga na
época. Foi por causa dessa penúria, e da revolta que dela resultou, que
Lênin decidiu introduzir a \textsc{nep} --- a Nova Política Econômica ---, que
limitou as expropriações feitas pelo Estado e permitiu aos camponeses
dispor de um pouco do excedente para seu próprio uso. A \textsc{nep} melhorou de
imediato as condições econômicas no país. A penúria de 1932"-1933 foi
desencadeada por métodos “comunistas” semelhantes: a vontade de impor
a coletivização.

Encontramos a mesma situação que em 1921, o que forçou Stálin a revisar
um pouco sua política. Compreendeu que o bem"-estar de um país,
sobretudo predominantemente agrário como a Rússia, depende
principalmente do campesinato. O \textit{slogan} foi lançado: era preciso dar ao
camponês a possibilidade de alcançar um “bem"-estar” maior. Essa
“nova” política é apenas uma astúcia, um descanso temporário para o
camponês. Ela não é mais comunista do que a precedente política
agrícola. Desde o início da ditadura bolchevique, o Estado não fez
outra coisa senão dar continuidade à expropriação, com maior ou menor
intensidade, mas sempre da mesma maneira; ele despoja o campesinato ao
instituir leis repressivas, empregando a violência, multiplicando
chicanas e represálias, impondo todos os tipos de interdições,
exatamente como nos piores dias do czarismo e da Primeira Guerra. A
política atual é apenas uma variante do “comunismo de guerra” de
1920"-1921 --- com cada vez mais “guerra” (de repressão armada) e cada
vez menos “comunismo”. Sua “igualdade” é aquela de uma penitenciária;
sua “liberdade”, aquela de um grupo de condenados a trabalho forçado. Assim, não
nos surpreende que os bolcheviques afirmem que a liberdade é um
preconceito burguês.

Os turibulários da União Soviética insistem no fato de que o “comunismo
de guerra” era justificado no início da Revolução, na época do bloqueio
e dos \textit{fronts} militares. Contudo, mais de dezesseis anos passaram. Já não há
mais bloqueio nem combates nos \textit{fronts}, nem contra"-revolução
ameaçadora. Todos os grandes Estados do mundo reconheceram a \textsc{urss}. O
governo soviético insiste em sua boa vontade em relação aos Estados
burgueses, solicita sua cooperação e comercializa muito com eles.
Mantém, inclusive, relações amigáveis com Mussolini e Hitler, esses
famosos campeões da liberdade. Ajuda o capitalismo a enfrentar suas
tempestades econômicas comprando milhões de dólares em mercadorias e
abrindo"-lhe novos mercados.

Eis, pois, em grandes linhas, o que a Rússia soviética realizou durante
os dezessete anos que se seguiram à Revolução. Mas, no que concerne ao
comunismo propriamente dito, o governo bolchevique segue exatamente a
mesma política anterior. Ele efetuou algumas mudanças políticas e
econômicas superficiais, mas, fundamentalmente, trata"-se sempre do
mesmo Estado, fundado sobre o mesmo princípio de violência e coerção,
que emprega os mesmos métodos de terror e coação que empregou durante o período
1920"-1921.

\section*{a multiplicação das classes}
Existe um número bem maior de classes na Rússia hoje do que em 1917, e
do que na maioria dos outros países. Os bolcheviques criaram uma vasta
burocracia soviética, que goza de privilégios especiais e de uma
autoridade quase ilimitada sobre as massas operárias e camponesas. Essa
burocracia é, ela própria, comandada por uma classe ainda mais
privilegiada de “camaradas responsáveis” --- a nova aristocracia
soviética.

A classe operária é dividida e subdividida em muitas categorias: os
\textit{udarniki} (as tropas de choque dos trabalhadores, a quem se concede
diferentes privilégios), os “especialistas”, os artesãos, os simples
operários e os sem qualificação. Há as “células” de fábricas, os
comitês de fábricas, os pioneiros, os \textit{komsomols}, os membros do Partido,
que gozam de vantagens materiais e de uma parcela de autoridade.

Também existe a vasta classe dos \textit{lishenti}, as pessoas privadas de
direitos cívicos; a maioria não tem a possibilidade de trabalhar,
nem o direito de viver em certos lugares: elas são praticamente
privadas de todo meio de existência. O famoso “livro de registros” da época
czarista, que proibia os judeus de viver em certas regiões do país, foi
reinstaurado para toda a população graças à criação do novo passaporte
soviético.

Acima de todas essas classes, reina a \textsc{gpu}, instituição temida,
secreta, poderosa e arbitrária, autêntico governo no interior do
governo. A \textsc{gpu}, por sua vez, possui suas próprias categorias
sociais. Ela tem suas forças armadas, seus estabelecimentos comerciais
e industriais, suas leis e seus regulamentos, e dispõe de um vasto
exército de escravos: a população penitenciária. Mesmo nas prisões e
nos campos de concentração, encontram"-se diferentes classes
beneficiando de privilégios especiais.

Na indústria reina o mesmo tipo de comunismo que na agricultura. Um
sistema Taylor sovietizado funciona em toda a Rússia, combinando normas
de qualidade muito baixas e trabalho por peça --- sistema mais
intensivo de exploração e degradação humana, que suscita inumeráveis
diferenças de salários e remunerações.

Os pagamentos são feitos em dinheiro, em rações, em reduções sobre os
encargos (aluguéis, eletricidade etc.), sem falar dos prêmios e
recompensas especiais para os \textit{udarniki}. Em resumo, é o \textit{salariato} que 
funciona na Rússia.

\section*{uma ditadura cada vez mais impiedosa}
Essas são as principais características do sistema soviético atual. É
preciso dar provas de uma ingenuidade imperdoável, ou de uma hipocrisia
ainda mais inescusável, para sustentar, como fazem os zeladores do
bolchevismo, que o trabalho forçado na Rússia demonstra as capacidades
de “auto"-organização das massas no campo da produção”.

Estranhamente, encontrei indivíduos aparentemente inteligentes que
sustentam que, graças a tais métodos, os bolcheviques “estão
construindo o comunismo”. Aparentemente, alguns creem que construir uma
nova sociedade consiste em destruir brutalmente, fisicamente e moralmente os mais
elevados valores da humanidade. Outros sustentam que a via da liberdade
e da cooperação passa pela escravidão dos operários e pela eliminação
dos intelectuais. Segundo eles, destilar o veneno do ódio e da inveja,
instaurar um sistema generalizado de espionagem e terror, são a
melhor maneira de a humanidade preparar"-se para o espírito
fraternal do comunismo!

Estou, evidentemente, em total desacordo com essas concepções. Nada é
mais pernicioso do que aviltar um ser humano e fazer dele a engrenagem
de uma máquina sem alma, transformá"-lo em servo, em espião ou em
vítima desse espião. Nada é mais corruptor do que a escravidão e o
despotismo.

O absolutismo político e a ditadura têm inúmeros pontos comuns: os meios
e os métodos utilizados para alcançar um determinado objetivo acabam
por tornar"-se o objetivo. O ideal do comunismo, do socialismo, deixou
há muito tempo de inspirar os chefes bolcheviques. O poder e o reforço
do poder tornaram"-se seu único fim. Mas a submissão abjeta, a
exploração e o aviltamento dos homens transformaram a mentalidade do
povo.

A nova geração é o produto dos princípios e métodos bolcheviques, o
resultado de dezesseis anos de propagação de opiniões oficiais, as únicas
permitidas nesse país. Tendo crescido em um regime no qual
todas as ideias e todos os valores são ditados e controlados pelo
Estado, a juventude soviética conhece poucas coisas sobre a própria
Rússia, e ainda menos sobre outros países. Essa juventude conta
inúmeros fanáticos cegos, de espírito estreito e intolerante; ela é
privada de toda percepção moral, desprovida de senso de justiça e
direito. A esse elemento vem se somar a influência da vasta classe dos
carreiristas, dos arrivistas e dos egoístas educados no dogma
bolchevique: “O fim justifica os meios”. Todavia, existem exceções nas
fileiras da juventude russa. Um bom número destas é profundamente
sincero, heroico e idealista. Eles veem e sentem a força dos ideais que
o Partido reivindica ruidosamente. Dão"-se conta de que as massas
foram traídas. Sofrem profundamente com o cinismo e o desprezo que o
Partido preconiza em relação a toda emoção humana. A presença dos
\textit{komsomols} nas prisões políticas soviéticas, nos campos de concentração
e no exílio, e os incríveis riscos que alguns deles assumem para fugir
desse país provam que a jovem geração não é apenas composta de
indivíduos servis ou medrosos. Não, nem toda a juventude russa foi
transformada em fantoches, em fanáticos ou em adoradores do trono de
Stálin e do mausoléu de Lênin.

A ditadura tornou"-se uma necessidade absoluta para a sobrevivência do
regime, pois onde reinam um sistema de classes e a desigualdade
social, o Estado deve recorrer à força e à repressão. A brutalidade desse
regime é sempre proporcional à amargura e ao ressentimento que as
massas experimentam. O terror estatista é mais forte na Rússia
soviética do que em qualquer outro país do mundo civilizado atual,
porque Stálin deve vencer e reduzir à escravidão uns 100 milhões de
camponeses obstinados. É porque o povo odeia o regime que a sabotagem
industrial é tão desenvolvida na Rússia, que os transportes são tão
desorganizados, depois de mais de dezesseis anos de gestão praticamente
militarizada; não se pode explicar de outra forma a terrível penúria no
sul e no sudeste, a despeito das condições naturais favoráveis,
malgrado as medidas mais severas tomadas para obrigar os camponeses a
semear e colher, e apesar do extermínio e da deportação de mais de um
milhão de camponeses aos campos de trabalho forçado.

A ditadura bolchevique encarna uma forma de absolutismo que deve
incessantemente endurecer para sobreviver, suprimindo toda opinião
independente e toda crítica ao Partido, inclusive no interior de seus
círculos mais elevados e mais fechados. É significativo, por exemplo,
que os bolcheviques e seus agentes, estipendiados ou benévolos, não
cessem de assegurar ao resto do mundo que “tudo vai bem na Rússia
soviética” e que “a situação melhora constantemente”. Esse tipo de
discurso é tão crível quanto os discursos pacifistas pronunciados por
Hitler, enquanto aumenta freneticamente sua força militar.

\section*{tomada de reféns e patriotismo}
Longe de atenuar"-se, a ditadura é a cada dia mais impiedosa. O último
decreto contra os pretensos contra"-revolucionários, ou os traidores
do Estado soviético, deveria convencer inclusive alguns dos mais
ardentes incensadores dos milagres realizados na Rússia. Esse decreto
reforça as leis já existentes contra toda pessoa que não pode, ou não
quer, respeitar a infalibilidade da Santíssima Trindade --- Marx, Lênin e
Stálin. E os efeitos desse decreto são ainda mais drásticos e cruéis
contra toda pessoa julgada culpada. É verdade, a tomada de reféns não
é uma novidade na \textsc{urss}: Piotr Kropotkin e Vera Figner protestaram em
vão contra essa mancha negra sobre o estandarte da Revolução Russa.
Agora, ao cabo de dezessete anos de dominação bolchevique, o poder
julgou necessário promulgar um novo decreto. Não apenas renova com a
prática da tomada de reféns como também pune cruelmente todo adulto
pertencente à família do criminoso --- suposto ou real. Eis como o novo
decreto define a traição em relação ao Estado: “todo ato cometido por
um cidadão da \textsc{urss} e que seja nocivo às forças armadas da \textsc{urss},
à independência ou à inviolabilidade do território, tal como a
espionagem, a traição de segredos militares ou de segredos de Estado, a
passagem para o inimigo, a fuga ou a partida de avião para um país
estrangeiro”.

Os traidores sempre foram, evidentemente, fuzilados. O que torna esse
novo decreto ainda mais terrificante é a cruel punição que ele exige
para todo indivíduo que vive com a infeliz vítima ou que lhe conceda
auxílio, quer o “cúmplice” esteja a par do delito, quer ignore a
existência. Ele pode ser preso, exilado ou, inclusive, fuzilado;
pode perder seus direitos cívicos e ser desapossado de todos os seus bens.
Em outros termos, esse novo decreto institucionaliza um prêmio para
todos os informantes que, a fim de salvar sua própria pele, colaborarem
com a \textsc{gpu} para se fazer notar e denunciarem aos homens de ação do Estado 
russo o desafortunado parente que ofendeu aos Sovietes.

Esse novo decreto deveria varrer definitivamente toda dúvida que ainda subsistia
em relação à existência do comunismo na Rússia. Esse texto
jurídico não tenciona sequer defender o internacionalismo e os
interesses do proletariado. O velho hino internacionalista
transformou"-se agora numa canção pagã que exalta a pátria e que a
imprensa soviética servil incensa ruidosamente: “A defesa da Pátria é a
lei suprema da vida, e aquele que ergue a mão contra ela, que a trai,
deve ser eliminado”.

Doravante, é evidente que a Rússia soviética é, no plano político, um
regime de despotismo absoluto e, no plano econômico, a forma mais
grosseira do capitalismo de Estado.


\chapter{Trotski protesta em demasia}

\textsc{Este panfleto} desenvolve as ideias expostas em um artigo de \textit{Vanguard},
mensal anarquista editado em Nova York. Ele foi publicado no número de
julho de 1938, mas como esta revista dispunha de um espaço limitado, só
uma parte do manuscrito original foi posta à disposição dos leitores.
Apresento aqui uma versão simultaneamente corrigida e desenvolvida.

Leon Trotski afirmará certamente que toda crítica de seu papel durante a
tragédia de Kronstadt só faz reforçar e encorajar seu inimigo mortal:
Stálin. Mas é porque Trotski não pode conceber que alguém possa
detestar o selvagem que reina no Kremlin e o cruel regime que ele
dirige, e ao mesmo tempo não isentar Leon Trotski do crime que
cometeu contra os marinheiros de Kronstadt.

Na minha opinião, nenhuma diferença fundamental separa os dois
protagonistas desse generoso sistema ditatorial, com a diferença de que
Leon Trotski já não se encontra no poder para prodigalizar seus
favores, ao contrário de Stálin. Não, não defendo o atual
dirigente da Rússia. 
Devo, contudo, ressaltar que Stálin não desceu do céu para vir perseguir
de repente o desafortunado povo russo. Ele se contenta em continuar a
tradição bolchevique, embora o faça de uma maneira mais impiedosa.

O processo que consistiu em destituir as massas russas de sua revolução
começou quase imediatamente após a tomada do poder por Lênin e seu
partido. A instauração de uma discriminação grosseira no racionamento
e na moradia, a supressão de todas as liberdades políticas, as
perseguições e as prisões contínuas tornaram"-se o cotidiano das
massas russas.
É verdade que os expurgos da época não visavam aos membros do partido,
ainda que alguns comunistas também tivessem sido jogados nas prisões e
nos campos de concentração. É preciso ressaltar que os militantes da
primeira Oposição operária e seus dirigentes foram rapidamente
eliminados. Chliapnikov foi enviado para “descansar” no Cáucaso e Alexandra
Kollontai colocada em prisão domiciliar.
Mas todos os outros oponentes políticos (mencheviques,
socialistas"-revolucionários, anarquistas, bem como uma grande parte
dos intelectuais liberais) e inúmeros operários e camponeses foram
jogados brutalmente nas prisões da Tcheka ou exilados em regiões
distantes da Rússia e da Sibéria, onde eram condenados a uma morte
lenta.
Em outros termos, não foi Stálin quem inventou a teoria e os métodos que
esmagaram a Revolução Russa e forjaram novas cadeias ao povo russo.

É verdade --- admito"-o de bom grado ---, a ditadura tornou"-se monstruosa
sob o reinado de Stálin. Mas isso não diminui, no entanto, a
culpabilidade de Leon Trotski, que foi um dos atores do drama
revolucionário do qual Kronstadt constituiu uma das cenas mais
sangrentas.
\asterisc

Tenho diante de mim dois números (de fevereiro e abril de 1938) de \textit{New
International}, o órgão oficial de Trotski. Eles contêm artigos de John
G. Wright, cem por cento trotskista, e do próprio Grande Patrão.
Esses textos tencionam refutar as acusações lançadas contra Trotski em
relação a Kronstadt. O sr.~Wright faz sobretudo eco à voz de seu senhor
e seus documentos não são de primeira mão. Além disso, ele não se
encontrava pessoalmente na Rússia em 1921. Prefiro, pois,
interessar"-me principalmente pelos propósitos de Leon Trotski.
Ao menos, tem ele o sinistro mérito de ter participado da “liquidação”
de Kronstadt.

Entretanto, o artigo de Wright contém algumas inexatidões imprudentes
que devem ser desmascaradas de imediato. Eu as denunciarei de início,
rapidamente, e me ocuparei em seguida dos argumentos de seu mestre
pensador.

John G.~Wright sustenta que \textit{A Revolta de Kronstadt} de Alexandre Berkman
“só faz reformular interpretações e pretensos fatos fornecidos pelos
socialistas revolucionários de direita, e coletados em ‘A verdade sobre
a Rússia de Volya’, editado em Praga, em 1921”.

Esse senhor acusa, em seguida, Alexandre Berkman “de ser um homem pouco
escrupuloso, um plagiário que se entrega a insignificantes retoques e
tem por hábito dissimular a autêntica fonte do que ele apresenta como
sua própria análise”. A vida e a obra de Alexandre Berkman fazem dele
um dos maiores pensadores e combatentes revolucionários, um homem
inteiramente devotado a seu ideal. Aqueles que o conheceram podem
testemunhar sua honestidade em todas as suas ações, assim como sua
integridade como escritor. [\ldots]\footnote{ Permiti"-me, aqui, cortar algumas 
linhas em que Emma Goldman repete palavra por palavra seus argumentos 
em favor de Alexandre Berkman.}

O comunista médio, seja fiel a Trotski ou a Stálin, conhece
aproximadamente tanto a literatura anarquista e seus autores quanto,
digamos, um católico conhece Voltaire ou Thomas Paine. A ideia segundo
a qual se deve buscar informações relativas à posição de seus
adversários políticos antes de condená"-los às chamas é considerada 
uma heresia pela hierarquia comunista. Não penso, pois, que John G.
Wright mente de modo deliberado em relação a Alexandre Berkman. Creio
que ele é profundamente ignorante.

Durante toda a sua vida, Alexandre Berkman manteve diários pessoais.
Mesmo durante os catorze anos de suplícios que suportou na Western
Penitentiary, nos Estados Unidos, Alexandre Berkman sempre conseguiu
manter um diário que ele me enviava clandestinamente naquela época. No
navio, o S.S. Buford, que nos levou à Rússia, em uma longa e perigosa
viagem de 28 dias, meu camarada continuou a escrever seu diário e
manteve esse antigo hábito durante os 23 meses que passamos na Rússia.

A obra \textit{As memórias de prisão de um anarquista}, que até mesmo críticos
conservadores a compararam à \textit{Casa dos mortos} de Fiódor Dostoiévski, foi
concebida a partir de seu diário. \textit{A Revolta de Kronstadt} e \textit{O mito
bolchevique} também são produto de suas anotações, tomadas
cotidianamente na Rússia. É, portanto, estúpido acusar a brochura de
Berkman sobre Kronstadt de “reformular fatos inventados”, apresentados
anteriormente em um livro dos socialistas revolucionários editado em
Praga.

Igualmente fantasiosa é a acusação feita por Wright contra Alexandre
Berkman de ter negado a presença do general Kozlovski em Kronstadt.

Em \textit{A Revolta de Kronstadt} (p.~15), meu velho amigo escreve, com efeito:
“O ex"-general Kozlovski encontrava"-se efetivamente em Kronstadt.
Foi Trotski quem o enviou para lá na condição de especialista da
artilharia. Ele não desempenhou absolutamente qualquer papel nos
acontecimentos de Kronstadt”. E Zinoviev em pessoa confirmou"-o,
quando se encontrava no zênite de sua glória. Durante a sessão
extraordinária do soviete de Petrogrado, em 4 de março de 1921, sessão
convocada para decidir o destino de Kronstadt, Zinoviev declarou:
“Evidentemente, Kozlovski é velho e nada pode fazer, mas os oficiais
brancos estão por trás dele e eles enganam os marinheiros”. E Alexandre
Berkman ressaltou que os marinheiros não tinham aceitado os serviços de
qualquer general queridinho de Trotski, e que haviam recusado as
provisões e outras ajudas propostas por Victor Tchernov, dirigente
dos socialistas"-revolucionários de direita em Paris.

Os trotskistas consideram certamente que é dar provas de sentimentalismo
burguês permitir aos marinheiros caluniados exprimir"-se e
defender"-se. Essa concepção das relações com um adversário político,
esse jesuitismo detestável, fez muito mais para destruir o movimento
operário em seu conjunto do que qualquer das táticas “sagradas” do
bolchevismo.

Para que o leitor possa decidir quem tem razão, se os acusadores de
Kronstadt ou os marinheiros que se exprimiram claramente à época,
reproduzo aqui a mensagem de rádio enviada aos operários do mundo
inteiro em 6 de março de 1921: 

\begin{hedraquote}
Nossa causa é justa: somos partidários
do poder dos sovietes, não dos partidos. Somos favoráveis à eleição
livre de representantes das massas trabalhadoras. Os sovietes fantoches
manipulados pelo Partido Comunista sempre foram surdos às nossas
necessidades e reivindicações; só recebemos uma resposta: a metralhadora
[\ldots]. Camaradas! Não apenas eles vos enganam como também travestem
deliberadamente a verdade e difamam"-nos do modo mais desprezível
[\ldots]. Em Kronstadt, todo o poder está exclusivamente nas mãos dos
marinheiros, soldados e operários revolucionários --- não nas mãos dos
contra"-revolucionários dirigidos por um certo Kozlovski, como a rádio
de Moscou tenta mentirosamente fazer"-vos crer [\ldots]. Não tardai,
camaradas! Reuni"-vos, contatai"-vos; pedi para que vossos delegados
possam vir visitar"-nos em Kronstadt. Só vossos delegados poderão
dizer"-vos a verdade e denunciar as abomináveis calúnias relativas ao
pão ofertado pelos finlandeses e à ajuda proposta pela Entente. Viva o
proletariado e o campesinato revolucionários! Viva o poder dos sovietes
livremente eleitos!
\end{hedraquote}

Os marinheiros pretensamente “dirigidos” por Kozlovski pedem aos
operários do mundo inteiro para enviar delegados a fim de que eles
verifiquem se as ignóbeis calúnias difundidas pela imprensa soviética
contra eles têm o mínimo fundamento!

Leon Trotski é surpreendido e fica indignado quando alguém ousa protestar
contra a repressão de Kronstadt. Apesar de tudo, esses acontecimentos
ocorreram há muito tempo, há dezessete anos, e se trataria apenas de um
“episódio na história das relações entre a cidade proletária e o
vilarejo pequeno"-burguês”. Por que fazer tanto barulho hoje? A menos
que se queira “descreditar a única corrente revolucionária que nunca
renegou sua bandeira, que nunca se comprometeu com o inimigo, e a única
a representar o futuro”. O egotismo de Leon Trotski, que seus amigos e
partidários conhecem muito bem, sempre foi extraordinário. Desde que as
perseguições de seu inimigo mortal dotaram"-no de uma espécie de
varinha de condão, sua suficiência alcançou proporções alarmantes.

Leon Trotski sente"-se ultrajado com o fato de alguém referir"-se de
novo ao “episódio” de Kronstadt e interrogar"-se sobre seu papel
nesses acontecimentos. Ele não compreende que aqueles que o defenderam
contra seu detrator tenham igualmente o direito de perguntar"-lhe que
métodos ele empregou quando estava no poder. Eles têm o direito de
perguntar"-lhe como tratou aqueles que não consideravam suas opiniões
como uma verdade de Evangelho. Evidentemente, seria ridículo esperar
que ele fizesse seu \textit{mea"-culpa} e proclamasse: “Eu também era apenas um
homem e cometi erros. Eu também pequei e matei meus irmãos ou ordenei
que os matassem”. Só sublimes profetas souberam alcançar tais ápices de
coragem. Leon Trotski não faz parte deles. Ao contrário, ele continua a
querer apresentar"-se como onipotente, a crer que todos os seus atos e
juízos foram sensatos, e cobrir de anátemas aqueles que
são loucos o bastante para sugerir que o grande deus Leon Trotski também tem
pés de barro.

Ele zomba das provas escritas deixadas pelos marinheiros de Kronstadt e do
testemunho daqueles que se encontravam suficientemente próximos da
cidade rebelde para ver e ouvir o que se passou durante o horrível
cerco. Ele os chama de “falsas etiquetas”. Isso não o impede, contudo,
de assegurar a seus leitores que sua explicação da revolta de Kronstadt
pode ser “corroborada e ilustrada por inúmeros fatos e documentos”. As
pessoas inteligentes podem perguntar"-se por que Leon Trotski não tem
sequer a decência de apresentar essas “falsas etiquetas”, para
que elas tenham condições de forjar, por si próprias, uma 
opinião.

Até mesmo os tribunais burgueses garantem ao acusado o direito de
apresentar provas para se defender. Mas não é o caso de Leon Trotski,
porta"-voz de uma única verdade, ele que “nunca renegou sua bandeira e
nunca se comprometeu com seus inimigos”.

Podemos compreender essa falta elementar de decência por parte de um
indivíduo como John G.~Wright. Afinal, como eu já disse, ele não
faz outra coisa senão citar as Sagradas Escrituras bolcheviques. Mas,
para um personagem da envergadura mundial de Leon Trotski, o fato de
silenciar quanto às provas apresentadas pelos marinheiros de Kronstadt
indica, na minha opinião, que esse homem é realmente desonesto. O velho
ditado: “Um leopardo muda de manchas, mas nunca de natureza” aplica"-se
perfeitamente a Leon Trotski. O calvário que ele sofreu durante seus
anos de exílio, o trágico desaparecimento de seus próximos, de seus
entes amados e, de modo ainda mais dramático, a traição de seus
ex"-companheiros de armas nada lho ensinaram, infelizmente. Nem sequer
uma gota de ternura, de doçura, irrigou o espírito rancoroso de
Trotski.

Que pena para ele que, às vezes, se dê mais atenção ao silêncio dos mortos
que à palavra dos vivos! De fato, as vozes sufocadas em Kronstadt
fizeram"-se ouvir cada vez mais ruidosamente no transcurso dos
dezessete últimos anos. É por essa razão que seu som desagrada tanto
Leon Trotski?

Segundo o fundador do Exército Vermelho, “Marx já dizia que não se podia
julgar os partidos nem os indivíduos sobre o que eles dizem de si
mesmos”. Que pena que Trotski não perceba a que ponto essa frase
aplica"-se perfeitamente a seu próprio caso! Entre os bolcheviques
capazes de escrever com certo talento, nenhum autor conseguiu
projetar"-se tanto quanto Trotski. Ninguém se vangloriou tanto quanto
ele de ter participado da Revolução Russa e dos acontecimentos que se
seguiram. Se aplicarmos a Trotski o critério de seu mestre"-pensador,
deveríamos deduzir que seus escritos não têm qualquer valor ---
raciocínio evidentemente absurdo.

Zeloso em desacreditar os motivos da revolta de Kronstadt, Leon Trotski
faz a seguinte observação: “Aconteceu"-me de enviar de diferentes
\textit{fronts} dezenas de telegramas reivindicando a mobilização de novos
destacamentos ‘seguros’, formados por operários de Petrogrado e de
marinheiros do Báltico. Todavia, desde fins de 1918 e, em todo o caso,
não depois de 1919, os \textit{fronts} começaram a se queixar de que os novos
destacamentos de marinheiros de Kronstadt não eram bons, que eram
exigentes, indisciplinados, pouco seguros no combate, em suma, mais
nocivos do que úteis”. Mais à frente, na mesma página, Trotski afirma:
``Quando a situação tornou"-se particularmente difícil em Petrogrado
faminta, examinou"-se mais de uma vez, no Bureau Político, a questão
de saber se não se deveria tomar um \textit{empréstimo interno} em
Kronstadt, onde ainda restavam importantes reservas de variados gêneros
alimentícios.'' Mas os delegados dos operários de Petrogrado respondiam:
“Eles nada nos darão de boa vontade. Traficam os lençóis, o carvão, o
pão. Em Kronstadt, hoje, toda a canalha ergueu a cabeça”. Triste
exemplo de um procedimento tipicamente bolchevique: não apenas liquidam
fisicamente seus adversários políticos como também maculam sua memória.
Seguindo as pegadas de Marx e Engels, Lênin, Trotski e, depois, Stálin
utilizaram os mesmos métodos.

Não tenho a intenção de discutir aqui o comportamento dos marinheiros de
Kronstadt em 1918 ou em 1919. Cheguei na Rússia apenas em janeiro de
1920. Do início de 1920 até a “liquidação” de Kronstadt, quinze meses
mais tarde, os marinheiros da frota do Báltico foram apresentados como
homens de valor, que sempre deram provas de uma coragem inquebrantável.
Em múltiplas vezes, anarquistas, mencheviques,
socialistas"-revolucionários e também inúmeros comunistas
disseram"-me que os marinheiros formavam a espinha dorsal da
Revolução. Durante a manifestação do 1º de maio de 1920, e durante
outras festividades organizadas em honra da visita da primeira missão
do Partido Trabalhista britânico, os marinheiros de Kronstadt
constituíram um importante contingente, perfeitamente visível. Eles
foram saudados como grandes heróis que haviam salvado a revolução
contra Kerenski, e Petrogrado contra Iudenitch. Durante o aniversário
da Revolução de Outubro, os marinheiros encontravam"-se novamente nas
primeiras fileiras, e multidões compactas aplaudiram quando eles
representaram a tomada do Palácio de Inverno.

É possível que os dirigentes do Partido, à exceção de Leon Trotski, não
estivessem a par da corrupção e da desmoralização de Kronstadt que nos
descreve o fundador do Exército Vermelho? Não creio. Por sinal, duvido
que o próprio Trotski tivesse tido essa opinião antes de março de 1921.
Seu relato atual resulta de dúvidas que ele experimentou então, ou se
trata de uma tentativa de justificar, depois do fato, a “liquidação”
insensata de Kronstadt?

Ainda que se admita que os marinheiros não eram os mesmos de 1917,\footnote{ 
Segundo o historiador inglês Israel Getzler, em seu livro \textit{Kronstadt
1917"-1921}, 75\% dos marinheiros de Kronstadt engajaram"-se antes
de 1918.} é
evidente que os kronstadinos de 1921 nada tinham a ver com o sinistro
quadro pintado por Trotski e seu discípulo Wright. De fato, os
marinheiros só conheceram seu terrível destino por causa de sua
profunda solidariedade, de seus laços estreitos com os operários de
Petrogrado que sofreram a fome e o frio até se revoltar durante uma
série de greves em fevereiro de 1921. Por que Trotski e seus
partidários não mencionam esse fato? Leon Trotski sabe perfeitamente,
se Wright o ignora, que a primeira cena do drama de Kronstadt aconteceu
em Petrogrado, em 24 de fevereiro, e não foi encenada pelos marinheiros,
mas pelos grevistas. Isso porque foi nesse dia que os grevistas
manifestaram sua cólera acumulada contra a indiferença brutal dos
homens que não paravam de discorrer sobre a ditadura do proletariado,
ditadura que se transformou desde o início na ditadura impiedosa do
Partido Comunista.

Em seu diário, Alexandre Berkman diz:

\begin{hedraquote}
Os operários da fábrica de
Trubotchny puseram"-se em greve. Durante a distribuição das roupas de
inverno, os comunistas foram mais bem servidos do que aqueles que não eram
membros do Partido, queixavam"-se os grevistas. O governo recusa-se a levar
em consideração suas reivindicações enquanto os operários não
retornarem ao trabalho. 

Multidões de grevistas reuniram"-se nas ruas
próximas às fábricas, e soldados foram enviados para dispersá"-los.
Eram \textit{kursanti}, jovens comunistas da Academia militar. Não houve
violências.

Agora trabalhadores dos entrepostos do Almirantado e dos cais de
Calernaya juntaram"-se aos grevistas. A hostilidade aumenta contra a
atitude arrogante do governo. Eles tentaram manifestar-se na rua mas as
tropas montadas intervieram para impedi"-los.
\end{hedraquote}

Foi só depois de terem se informado da verdadeira situação dos operários
de Petrogrado que os marinheiros de Kronstadt fizeram em 1921 o que
haviam feito em 1917: solidarizaram"-se de imediato com os operários.
Por causa de seu papel em 1917, os marinheiros haviam sido considerados
sempre como o glorioso florão da Revolução. Em 1921, eles agiram do
mesmo modo, mas foram denunciados aos olhos do mundo como traidores,
contra"-revolucionários. Evidentemente, em 1917, os marinheiros de
Kronstadt tinham ajudado a pôr na sela os bolcheviques. Em 1921, eles
pediam contas pelas falsas esperanças que o Partido fizera nascer nas
massas, e pelas belas promessas que os bolcheviques haviam renegado tão
logo se julgaram solidamente instalados no poder. Crime abominável, na
verdade. Mas o mais importante nesse crime é que os marinheiros de
Kronstadt não se “amotinaram” num contexto sereno. Sua rebelião estava
profundamente enraizada no sofrimento dos trabalhadores russos: tanto o
proletariado das cidades quanto o campesinato.

Certamente, nosso ex"-comissário do povo assegura"-nos: “Os camponeses
fiam"-se nas requisições como num mal temporário. Mas a guerra civil
durou três anos. A cidade não dava quase nada ao vilarejo e
tomava"-lhe quase tudo, sobretudo para as necessidades da guerra. Os
camponeses haviam aprovado os ‘bolcheviques’, mas se tornavam cada vez
mais hostis aos ‘comunistas’”. Infelizmente, esses argumentos dizem
respeito à mais pura ficção, como provam inúmeros fatos, especialmente a
liquidação dos sovietes camponeses dirigidos por Maria Spiridonova, e o 
dilúvio de ferro e fogo lançado contra os camponeses para obrigá"-los
a entregar todos os seus produtos, inclusive seus grãos para a
semeadura da primavera.

De fato, os camponeses detestavam o regime quase desde o começo da
revolução ou, em todo o caso, certamente desde o momento em que o \textit{slogan}
de Lênin, “Expropriai os expropriadores”, tornou"-se “Expropriai os
camponeses para a glória da ditadura comunista”. Eis por que eles
protestavam constantemente contra a ditadura bolchevique, como
testemunha particularmente a sublevação dos camponeses da Carélia,
esmagada no sangue pelo general czarista Slastchev"-Krimsky. Se os
camponeses apreciavam tanto o regime soviético quanto Trotski queria
nos fazer crer, por que tiveram de enviar esse homem sanguinário para a
Carélia?

Slastchev"-Krimsky combatera a Revolução desde o início e dirigira
alguns dos exércitos de Wrangel na Crimeia. Cometeu atos bárbaros
contra prisioneiros de guerra e organizou abjetos \textit{pogroms}. E agora
esse general arrependia"-se e retornava à “sua pátria”. Esse rematado
contra"-revolucionário, esse massacrador de judeus, recebeu as
honrarias militares por parte dos bolcheviques, em companhia de vários
generais czaristas e oficiais dos exércitos brancos. Certo, pode-se
considerar como um justo castigo o fato de anti"-semitas serem
obrigados a saudar um judeu, Trotski, seu superior hierárquico, e
obedecer"-lhe. Mas para a Revolução e o povo russo, o retorno triunfal
desses imperialistas era um insulto.

A fim de recompensá"-lo por seu novo amor bem recente para com a pátria
socialista, confiou"-se a Slastchev"-Krimsky a missão de esmagar os
camponeses da Carélia que pediam a autodeterminação e melhores
condições de vida.

Leon Trotski conta"-nos que os marinheiros de Kronstadt em 1919 não
teriam dado suas provisões se lhes tivessem pedido gentilmente --- como
se os bolcheviques alguma vez tivessem utilizado a gentileza! De fato,
essa palavra não faz parte do vocabulário deles. Entretanto, foram
esses marinheiros pretensamente desmoralizados, esses “especuladores”,
essa “canalha” etc., que foram em defesa do proletariado das cidades em
1921, e cuja primeira reivindicação era a igualdade das rações. Que
gângsters esses kronstadinos, não é mesmo?

Wright e Trotski tentam desacreditar os marinheiros de Kronstadt porque
estes últimos formaram rapidamente um Comitê Revolucionário Provisório.
Lembremos, de início, que eles não premeditaram sua revolta, mas que se
reuniram em 1º de março de 1921 para discutir sobre a maneira de ajudar
seus camaradas de Petrogrado. De fato, John G.~Wright fornece"-nos a
resposta quando escreve: “Não está absolutamente excluído que as
autoridades locais de Kronstadt não tenham sabido administrar
habilmente a situação [\ldots]. Sabemos que Kalinin e o comissário do povo
Kuzmin não eram de modo algum estimados por Lênin e seus colegas [\ldots].
Na medida em que as autoridades locais não estavam conscientes da
importância do perigo e não tomaram medidas eficazes e adequadas para
tratar a crise, suas inabilidades certamente desempenharam um papel no
transcurso dos acontecimentos [\ldots]”.

O comentário relativo à opinião de Lênin sobre Kalinin e Kuzmin é
apenas, infelizmente, um velho truque dos bolcheviques: escolhem um
bode expiatório entre uns subalternos inábeis para isentar a
responsabilidade dos dirigentes.

É fato que as autoridades locais de Kronstadt cometeram uma
“inabilidade”. Kuzmin atacou violentamente os marinheiros e
ameaçou"-os com terríveis represálias. Os marinheiros sabiam
evidentemente o que os aguardava. Sabiam que, se Kuzmin e Vassíliev
obtivessem carta branca, sua primeira medida seria privar Kronstadt de
suas armas e de suas reservas de alimentos. Foi a razão pela qual os
marinheiros formaram seu Comitê Revolucionário Provisório. E eles foram
encorajados em sua decisão, quando souberam que uma delegação de trinta
marinheiros, enviada a Petrogrado para discutir com os operários,
teve recusado o direito de retornar a Kronstadt, que seus membros haviam
sido presos e colocados nas mãos da Tcheka.

Wright e Trotski dão uma enorme importância a um rumor anunciado durante
a reunião do 1º de março: um caminhão lotado de soldados pesadamente
armados ia juntar"-se a Kronstadt. É evidente que Wright nunca viveu
sob uma ditadura hermética. Eu sim. Quando as redes pelas quais passam
os contatos humanos são interrompidas, quando todo pensamento é
encerrado em si mesmo e a liberdade de expressão é sufocada, então
os rumores espalham"-se com velocidade do relâmpago e assumem
dimensões terrificantes. Além disso, caminhões repletos de soldados e
tchekistas armados até os dentes patrulhavam frequentemente as ruas
durante o dia. Eles lançavam suas redes durante a noite e conduziam
suas presas até a Tcheka. Esse espetáculo era frequente em Petrogrado e
Moscou na época em que eu me encontrava na Rússia. No clima de tensão
instaurado pelo discurso ameaçador de Kuzmin, era perfeitamente normal
que rumores circulassem e que se desse crédito a eles.

Durante a campanha contra os marinheiros de Kronstadt, afirmou"-se
também que o fato de que notícias sobre Kronstadt tivessem
aparecido na imprensa parisiense duas semanas antes do começo da
revolta era a prova de que os marinheiros tinham sido manipulados pelas
potências imperialistas e que essa revolta havia sido, de fato, tramada
em Paris. É evidente que essa calúnia tinha por única utilidade
desacreditar os kronstadinos aos olhos dos operários.

Na realidade, essas notícias antecipadas nada tinham de extraordinário.
Não era a primeira vez que tais rumores nasciam em Paris, Riga ou
Helsinki, e geralmente elas não coincidiam com as declarações dos
agentes da contra"-revolução no estrangeiro. Por outro lado, muitos
acontecimentos que se produziram na União Soviética teriam
alegrado o coração da Entente, dos quais nunca se ouvia falar ---
acontecimentos bem mais nocivos à Revolução Russa e causados pela
ditadura do próprio Partido Comunista. Por exemplo, o fato de que a
Tcheka destruiu inúmeras realizações de Outubro e que, em 1921, ela já
havia se tornado uma excrescência mortal sobre o corpo da Revolução. Eu
poderia mencionar muitos outros acontecimentos semelhantes, que me
obrigariam a desenvolvimentos demasiado longos no âmbito desse artigo.

Não. As notícias antecipadas surgidas na imprensa parisiense não têm
qualquer relação com a revolta de Kronstadt. De fato, em 1921, em
Petrogrado, ninguém acreditava na existência de qualquer elo,
incluindo uma grande parte dos comunistas. Como eu já disse, John G.
Wright é apenas um simples discípulo de Leon Trotski e ignora o que a
maioria das pessoas, no interior e no exterior do Partido bolchevique,
pensava desse pretenso “elo” em 1921.

Os futuros historiadores apreciarão certamente o “motim” de Kronstadt em
seu verdadeiro valor. Se eles o fizerem, e quando isso acontecer,
estou persuadida de que chegarão à conclusão de que a sublevação não
teria podido ocorrer em melhor momento se ela não tivesse sido
deliberadamente planejada.

O fator determinante que decidiu o destino de Kronstadt foi a \textsc{nep}
(Nova Política Econômica). Lênin estava perfeitamente consciente de que
esse novo esquema “revolucionário” provocaria uma oposição considerável
no Partido. Ele precisava de uma ameaça imediata para fazer passar a
\textsc{nep}, de modo simultaneamente rápido e tranquilo. Kronstadt
produziu"-se em um momento muito útil para ele. Toda a máquina de
propaganda pôs"-se em marcha para demonstrar que os marinheiros
estavam em conluio com as potências imperialistas e com os elementos
contra"-revolucionários que queriam destruir o Estado comunista. Isso
funcionou às mil maravilhas. A \textsc{nep} foi imposta sem a mínima
dificuldade.

Acabarão por descobrir o custo pavoroso dessa manobra. Os trezentos
delegados, a flor da juventude comunista, que deixaram precipitadamente
o congresso do Partido para ir esmagar Kronstadt, representavam apenas
um punhado das milhares de vidas que foram cinicamente sacrificadas.
Eles partiram crendo com fervor nas mentiras e nas calúnias dos
bolcheviques. Aqueles que sobreviveram tiveram um rude despertar.

Recordo"-me de ter encontrado num hospital um jovem comunista ferido.
Contei essa anedota em \textit{Como perdi minhas ilusões sobre a Rússia}. Esse
testemunho nada perdeu de seu valor, apesar dos anos:

\begin{hedraquote}
Muitos daqueles que haviam sido feridos durante o ataque contra
Kronstadt foram conduzidos ao mesmo hospital, e eram sobretudo
\textit{kursanti}, jovens comunistas. Tive a oportunidade de discutir com um
deles. Sua dor física, disse"-me, nada representava diante de seus
sofrimentos psicológicos. Ele se dera conta tarde demais de que fora
enganado pelo \textit{slogan} da “contra"-revolução”. Nenhum general
czarista, nem mesmo um único guarda"-branco, assumira o comando dos
marinheiros de Kronstadt; ele lutara contra seus próprios camaradas,
marinheiros, soldados e operários que haviam heroicamente combatido
pela Revolução.
\end{hedraquote}

Nenhuma pessoa sensata verá a mínima semelhança entre a \textsc{nep} e a
reivindicação dos marinheiros de Kronstadt de trocar livremente os
produtos. A \textsc{nep} só fez reintroduzir os terríveis males que a
Revolução Russa tentara eliminar. A livre troca dos produtos entre os
operários e os camponeses, entre a cidade e o campo, encarnava a
própria razão de ser da Revolução. Evidentemente, “os anarquistas eram
hostis à \textsc{nep}”. Mas o mercado livre, como Zinoviev me dissera em
1920, “não tem qualquer espaço em nosso plano centralizado”. Pobre
Zinoviev! Não podia imaginar que monstro iria nascer da centralização
do poder!

Essa obsessão pela centralização da ditadura desenvolveu muito cedo
a divisão entre a cidade e a vila, os operários e os camponeses. Não
foi, como afirma Trotski, porque “a primeira é proletária [\ldots] e a
segunda, pequeno"-burguesa”, mas porque a ditadura bolchevique
paralisou simultaneamente as iniciativas do proletariado urbano e
aquelas do campesinato.

Segundo Leon Trotski, “A sublevação de Kronstadt não atraiu, mas afastou
os operários de Petrogrado. A demarcação operou"-se segundo a linha
das classes. Os operários sentiram imediatamente que os rebeldes de
Kronstadt encontravam"-se do outro lado da barricada, e eles apoiaram
o poder soviético”. Trotski se esquece de explicar a principal razão da
indiferença aparente dos operários de Petrogrado. Com efeito, a
campanha de mentiras, calúnias e difamação contra os marinheiros
começou em 2 de março de 1921. A imprensa soviética destilou
tranquilamente seu veneno contra os marinheiros. As acusações mais
desprezíveis foram lançadas contra eles e isso continuou até o
esmagamento de Kronstadt, em 17 de março de 1921. Além disso,
Petrogrado encontrava"-se sob lei marcial. Várias fábricas foram
fechadas e os operários, assim destituídos de seu ganha"-pão, começavam
a reunir"-se entre si. Citemos o diário de Alexandre Berkman:

\begin{hedraquote}
Ocorreram muitas prisões. Grupos de grevistas cercados por tchekistas
são amiúde conduzidos à prisão. Uma grande tensão nervosa reina na
cidade. Todos os tipos de precaução são tomados para proteger as
instituições governamentais. Colocaram metralhadoras em frente ao Hotel
Astoria, onde residem Zinoviev e outros dirigentes bolcheviques.
Proclamações oficiais ordenam aos grevistas para retornar ao trabalho
[\ldots] e lembram a população de que é proibido reunir"-se nas ruas. O
Comitê de Defesa começou uma “limpeza da cidade”. Muitos operários,
suspeitos de simpatizar com Kronstadt, foram presos. Todos os
marinheiros de Petrogrado e uma parte da guarnição, julgados “pouco
confiáveis”, foram enviados a locais distantes, enquanto as famílias dos
marinheiros de Kronstadt que viviam em Petrogrado foram tomadas como
reféns. O Comitê de Defesa informou Kronstadt que os “prisioneiros são
considerados como garantias” para a segurança do comissário da frota do
mar Báltico, N.~N.~Kuzmin, do presidente do soviete de Kronstadt, 
T.~Vassíliev e de outros comunistas. “Se nossos camaradas sofrerem o menor
mau trato, os reféns pagarão com suas vidas.”
\end{hedraquote}

Sob tal regime de ferro, era fisicamente impossível aos operários de
Petrogrado aliar"-se aos insurretos de Kronstadt, ainda mais porque
nem sequer uma linha dos manifestos publicados pelos marinheiros chegou aos
operários de Petrogrado. Em outros termos, Leon Trotski falsifica
deliberadamente os fatos. Os operários teriam certamente tomado o
partido dos marinheiros, porque eles sabiam que estes não eram
amotinados nem contra"-revolucionários, e que haviam se mostrado
solidários aos operários em 1905, bem como em março e outubro de 1917.
Eis por que posso afirmar que Trotski, completamente consciente,
insulta grosseiramente a memória dos marinheiros de Kronstadt. 

Em \textit{New International} (p.~106), Trotski assegura a seus leitores que “ninguém,
diga"-se de passagem, pensava naqueles dias na doutrina anarquista”.
Isso não se encaixa, infelizmente, com a incessante perseguição aos
anarquistas que começou em 1918, quando Leon Trotski liquidou o
quartel"-general anarquista em Moscou a metralhadas. Desde essa época,
o processo de eliminação dos anarquistas pôs"-se em marcha. Mesmo
hoje, muito tempo depois, os campos de concentração do governo
soviético estão repletos de anarquistas, aqueles que sobreviveram. De
fato, antes da insurreição de Kronstadt, em outubro de 1920, quando
Trotski mudou de opinião em relação a Makhno, porque necessitava de sua
ajuda e de seu exército para liquidar Wrangel, e consentiu que
se realizasse um congresso anarquista em Kharkov, várias centenas de
anarquistas foram presos e enviados à prisão de Butirka, onde
permaneceram até abril de 1921, sem que lhes fossem comunicado o menor
motivo de acusação. Depois, em companhia de outros militantes de
esquerda, eles desapareceram na calada da noite, enviados 
secretamente para prisões e campos de concentração na Rússia e na
Sibéria. Mas isso é uma outra página da história soviética. O que
importa ressaltar aqui é que se “pensava” muito nos anarquistas naquela
época; caso contrário, por que razão eles teriam sido presos e enviados
aos quatro cantos da Rússia e da Sibéria, como no tempo do czarismo?

Leon Trotski escarnece da reivindicação dos “sovietes livres”. Os
marinheiros tinham, com efeito, a ingenuidade de crer que sovietes
livres poderiam coexistir com uma ditadura. De fato, os sovietes livres
cessaram de existir muito mais cedo, assim como os sindicatos e as
cooperativas. Eles foram todos amarrados ao carro do aparelho de Estado
bolchevique. Um dia, Lênin declarou"-me com uma expressão de
satisfação: “Vosso grande homem, Errico Malatesta, é favorável aos
nossos sovietes”. E apressei"-me em corrigi"-lo: “Quereis dizer
sovietes livres, camarada Lênin. Eu também lhes sou favorável”.
Imediatamente Lênin mudou de assunto. Mas logo descobri por que os
sovietes livres haviam deixado de existir na Rússia.

John G.~Wright sustentará, sem dúvida, que não existia qualquer
problema em Petrogrado até o dia 22 de fevereiro. Isso se encaixa
muito bem com a maneira com que ele remaneja “a história” do Partido.
Mas o descontentamento e a agitação dos operários eram muito visíveis
quando chegamos à Rússia. Em cada fábrica que visitei, pude constatar
o descontentamento e a cólera dos trabalhadores, porque a ditadura do
proletariado tornara"-se a ditadura esmagadora do Partido
Comunista, fundada num sistema de racionamento diferenciado e
discriminações de toda sorte. Se o descontentamento dos operários não
explodiu antes de 1921, foi apenas porque eles agarravam"-se à
esperança tenaz de que, quando os \textit{fronts} fossem liquidados, as
promessas de Outubro seriam enfim cumpridas. E foi Kronstadt que fez
estourar sua última bolha de ilusão.

Os marinheiros haviam ousado tomar o partido dos operários descontentes.
Eles ousaram exigir que as promessas da Revolução --- “Todo o poder aos
sovietes” --- fossem enfim cumpridas. A ditadura política havia matado a
ditadura do proletariado. Essa foi sua única ofensa imperdoável contra o
Espírito Santo do bolchevismo.

Em uma nota de seu artigo (p.~49), Wright afirma que Victor Serge teria
recentemente declarado, em relação a Kronstadt, que “os bolcheviques,
uma vez confrontados com o motim, não tiveram outra solução senão
esmagá"-lo”. Victor Serge não reside mais nas terras hospitaleiras da
“pátria” dos trabalhadores. Se essa declaração citada por Wright é
verdadeira, não me parece desleal afirmar que Victor Serge simplesmente
não diz a verdade. Enquanto em 1921 ele pertencia à Seção Francesa da
Internacional Comunista, Serge estava tão transtornado e horrorizado
quanto Alexandre Berkman, eu mesma e muitos outros revolucionários ante
a carnificina que Leon Trotski preparava, segundo sua promessa de
“matar os marinheiros como perdizes”.\footnote{
Esta declaração não é de Trotski, mas figurava num panfleto
distribuído em Kronstadt pelos bolcheviques.} 
Cada vez que Serge tinha um
momento livre, ele irrompia em nosso cômodo, caminhava de um lado para
o outro, puxava os cabelos e golpeava seus punhos um contra o outro, de
tanto que estava indignado. “É preciso fazer alguma coisa, é preciso
fazer alguma coisa para deter esse horrível massacre”, repetia. Quando
nós lhe perguntamos por que ele, que era membro do Partido, não erguia
a voz para protestar, respondeu"-nos que isso não teria qualquer
utilidade para os marinheiros. Além disso, isso o assinalaria à Tcheka
e resultaria sem dúvida em seu discreto desaparecimento. Sua única
desculpa é que, naquela época, ele tinha uma jovem mulher e um bebê. Mas
se ele realmente declarou hoje, dezessete anos depois, que “os
bolcheviques, uma vez confrontados com o motim, não tiveram outra solução
senão esmagá"-lo”, tal atitude é, no mínimo, inescusável. Victor Serge
sabe tão bem quanto eu que \textit{não houve motim} em Kronstadt, que os 
marinheiros em nenhum momento utilizaram suas armas antes do início dos
bombardeios. Ele sabe igualmente que nenhum dos comissários comunistas
presos, sequer um único comunista, foi vítima de maus tratos. Eu
exorto, portanto, Victor Serge a dizer a verdade. Que ele tenha podido
continuar a viver na Rússia sob o regime de seus camaradas Lênin e
Trotski, enquanto inúmeros infelizes eram assassinados por terem
adquirido consciência de todos os horrores que aconteciam, é 
problema seu. Mas não posso deixá"-lo dizer que os bolcheviques tiveram
razão em crucificar os marinheiros.

Leon Trotski tem uma atitude sarcástica quando é acusado de ter
assassinado 1500 marinheiros. Não, suas mãos não estão sujas de
sangue. Ele confiou a Tukhatchevski a tarefa de matar os marinheiros
“como perdizes”, segundo sua expressão. Tukhatchevski aplicou suas
ordens com grande consciência profissional. Centenas de homens foram
massacrados, e os que sobreviveram aos tiros incessantes da artilharia
dos bolcheviques foram colocados nas mãos de Dybenko, célebre por sua
humanidade e seu senso de justiça.

Tukhatchevski e Dybenko são os heróis e os salvadores da ditadura! A
história parece ter um modo particular de fazer justiça.

Leon Trotski tenta nos exibir uma de suas cartas"-mestras quando se
pergunta “onde e quando seus grandes princípios foram confirmados na
prática, ao menos parcialmente, ao menos tendencialmente?” Essa
carta, como todas aquelas que ele já jogou durante a sua vida, não lhe
permitirá ganhar a partida. Na verdade, os princípios anarquistas foram
confirmados, na prática e tendencialmente, na Espanha. É verdade, isso só
pôde ser feito parcialmente. Como poderia ter sido diferente quando
todas as forças conspiravam contra a Revolução Espanhola? O trabalho
construtivo empreendido pela \textsc{cnt} (Confederación Nacional del Trabajo)
e pela \textsc{fai} (Federación Anarquista Ibérica) constitui uma
realização inimaginável aos olhos do regime bolchevique, e a
coletivização das terras e das fábricas na Espanha representa o maior
êxito de todos os períodos revolucionários. Além disso, ainda que
Franco ganhe e que os anarquistas espanhóis sejam exterminados, o
trabalho que eles começaram continuará a viver. Os princípios e as
tendências anarquistas estão tão profundamente implantados na terra da
Espanha que nada e ninguém os erradicará.

