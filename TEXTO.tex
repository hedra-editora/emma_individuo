\begin{resumopage} 


\item[Emma Goldman] (Kovno [atual Kaunas], 1869--Toronto,
1940). Revolucionária anarquista de origem russa, emigrou para Rochester,
Estados Unidos, em 1886.  Como grande parte dos emigrantes do leste europeu,
trabalha em uma fábrica de roupas, onde toma contato com as doutrinas socialista
e anarquista. Em 1899, muda-se para Nova York e conhece Alexander Berkman,
anarquista condenado em 1892 pela tentativa de assassinato do industrial Henry
Clay Frick. Em 1901, Leon Czolgosz assassina o presidente William McKinley, e
alega ter sido inspirado pelos ensinamentos de Emma. Ativista dos direitos da
mulher, une-se a Margaret Sanger na luta pelo controle de natalidade, dando
palestras por todo os \textsc{eua}. Em 1906, com a soltura de Berkman, retoma as
atividades em conjunto com seu companheiro e funda o periódico \textit{Mother
Earth} (1906-1917). Em 1910, publica  \textit{Anarchism and Other Essays}, dois
anos após ter a cidadania americana revogada pelo governo.  Deportada dos
\textsc{eua} em 1919, juntamente com Berkman, alcança a Rússia e lá permanece
até a revolta de Kronstadt (1921). Decepcionada com a onda de perseguições e a
repressão que se seguiram à Revolução Russa, parte para a Europa ocidental no
mesmo ano, e em 1923 publica \textit{My Disillusionment in Russia}, crítica
severa ao sistema soviético. Perseguida pelos agentes do \textsc{fbi} grande
parte de sua vida, foi presa seis vezes entre 1893 e 1921, acusada de incitar
rebeliões, preconizar o controle de natalidade e opor-se à Primeira Guerra
Mundial e ao alistamento militar, entre outras acusações. Em 1931, publica sua
autobiografia \textit{Living My Life}, e mantém intensa atividade como
palestrante, residindo nos principais países da Europa. Durante a Guerra Civil
Espanhola (1936) apoiou ativamente os anarquistas na luta contra o fascismo.
Faleceu em Toronto, Canadá, em 1940.  

\item[O indivíduo, a sociedade e o Estado] foi publicado pelo Free Society
Forum, Chicago, Illinois, em 1940. Defesa intransigente da liberdade do
indíviduo e crítica ferrenha à submissão ao poder estatal, esse texto, inspirado
em Kropotkin e Malatesta, já antecipava muitas das questões fundamentais do
século \textsc{xx}, como a militarização estratégica dos \textsc{eua}. A
presente edição conta ainda com o posfácio do livro \textit{My disillusionment
in Russia} (1923), e \textit{O comunismo não existe na Rússia}, artigo publicado
em 1935, no qual Emma critica o autoritarismo e a centralização de poder dos
sovietes.  

\item[Plínio Augusto Coêlho]  fundou em 1984 a Novos Tempos Editora, em
Brasília, dedicada à publicação de obras libertárias. Em 1989, transfere-se para
São Paulo, onde cria a Editora Imaginário, mantendo a mesma linha de
publicações. É idealizador e co-fundador do \textsc{iel} (Instituto de Estudos
Libertários).  

\item[Carlo Romani] é doutor em História Cultural pela Universidade de Campinas
(Unicamp) e pesquisador vinculado ao \textsc{nupaub/usp} (Antropologia Caiçara).
Publicou a biografia histórica \textit{Oreste Ristori: Uma aventura anarquista}
(Annablume, 2002), e atualmente ensina História Contemporânea na Universidade
Federal do Ceará (\textsc{ufc}).  

\item[Série Estudos Libertários:] as obras reunidas nesta série, em sua maioria
inéditas em língua portuguesa, foram escritas pelos expoentes da corrente
libertária do socialismo.  Importante base teórica para a interpretação das
grandes lutas sociais travadas desde a segunda metade do século \textsc{xix},
explicitam a evolução da idéia e da experimentação libertárias nos campos
político, social e econômico, à luz dos princípios federalista e
autogestionário.          

\end{resumopage}
