\documentclass[semcabeco,showtrims,trimframe,12pt,conselho,spreadimages]{memoir}

\usepackage[largepost]{hedraoptions} %% << %%%%%%%%%%%%%%%%
\usepackage[baruch]{hedrastyles}
\usepackage[xetex,chicagofootnotes]{tipografia}
\usepackage[standart,sempontinhos]{toc}
\usepackage{hedraextra}
\usepackage{penalidades}
\usepackage{graficos}
\usepackage{hedralogo}
\usepackage{hifensextras}
\usepackage{fichatecnica}
\usepackage[standart]{aparatos}
\usepackage{tabelas}
\usepackage{versos}
\usepackage{gitrevisioninfo}

\newcommand{\forceindent}{\leavevmode{\parindent=1,4em\indent}}

\linespread{1.15}

\usepackage{endnotes}
\renewcommand{\notesname}{Notas}

\usepackage{makeidx,hedraindex}  % cria índice
\makeindex	

%\counterwithin*{endnote}{part}
%\counterwithin*{endnote}{chapter}

\let\latexchapter\chapter
\makeatletter
\renewcommand\enoteheading{%
  \setcounter{secnumdepth}{-2}
  \latexchapter*{\notesname\markboth{NOTAS}{}}
  \mbox{}\par\vskip-\baselineskip
  \let\@afterindentfalse\@afterindenttrue
}
\makeatother
%\usepackage{fancyhdr}
%\pagestyle{fancy}
%\setlength{\headheight}{9mm}
%\fancyhf{}
%\fancyhead[R]{\thepage}
%\renewcommand{\headrulewidth}{0pt}

%\lhead[\fancyplain{}]{}
%\chead[\fancyplain{}]{}
%\rhead[\fancyplain{}]{\cnvt{\thepage} -- \thepage}

%\newcommand*{\cnvt}[1]{\the\numexpr#1-1\relax}

%\fancypagestyle{chapter}{
%\pagestyle{fancy}
%\setlength{\headheight}{5mm}
%\fancyhf{}
%\fancyhead[R]{\thepage}
%\renewcommand{\headrulewidth}{0pt}}


\usepackage{footmisc}

\renewcommand*\footnoterule{}
%\fancyhf[RO]{\cnvt{\thepage} -- \thepage}
%\fancyfoot{}
%\renewcommand{\headrulewidth}{0pt}
%\renewcommand{\footrulewidth}{0pt}}

\usepackage{fontspec}

%\usepackage{Formular}
\newfontfamily\Formular{Formular-Regular}[
BoldFont = Formular-Bold.otf]	

%--------------------------------------------ALTERAR DISTÃNCIA ENTRE TÍTULO DO SUMÁRIO E CAPÍTULOS
%\addtocontents{toc}{\vskip-15pt}
%--------------------------------------------
\usepackage{afterpage}

\newcommand\blankpage{%
    \null
    \thispagestyle{empty}%
    \addtocounter{page}{0}%
    \newpage}



%\usepackage{imakeidx} 
%\makeindex[program=xindy, options=-C utf8 -L portuguese]
%\newcommand\gobbleone[1]{}
%\newcommand*{\seeonly}[2]{\ (\emph{\seename} #1)}
%\newcommand*{\also}[2]{\emph{cf.} #1}
%\newcommand{\Also}[2]{\emph{See also} #1}
%\renewcommand\indexname{Índice onomástico}
%\makeindex[intoc]

\setcounter{tocdepth}{0}
\setcounter{secnumdepth}{-2} 
%\linespread{1.08}
\usepackage{commands}

\usepackage{setspace}

\makeatletter
\newenvironment{Parskip}{%
   \par
   \parskip=0.3\baselineskip \advance\parskip by 0pt plus 2pt
   \parindent=\z@
   \def\@listI{\leftmargin\leftmargini
      \topsep\z@ \parsep\parskip \itemsep\z@}
   \let\@listi\@listI
   \@listi
   \def\@listii{\leftmargin\leftmarginii
      \labelwidth\leftmarginii\advance\labelwidth-\labelsep
      \topsep\z@ \parsep\parskip \itemsep\z@}
   \def\@listiii{\leftmargin\leftmarginiii
       \labelwidth\leftmarginiii\advance\labelwidth-\labelsep
       \topsep\z@ \parsep\parskip \itemsep\z@}
   \partopsep=\z@
}{\par}
\makeatother

\makeatletter
\newenvironment{myParskip}{%
   \par
   \parskip=0.2\baselineskip \advance\parskip by 0pt plus 2pt
   \parindent=\z@
   \def\@listI{\leftmargin\leftmargini
      \topsep\z@ \parsep\parskip \itemsep\z@}
   \let\@listi\@listI
   \@listi
   \def\@listii{\leftmargin\leftmarginii
      \labelwidth\leftmarginii\advance\labelwidth-\labelsep
      \topsep\z@ \parsep\parskip \itemsep\z@}
   \def\@listiii{\leftmargin\leftmarginiii
       \labelwidth\leftmarginiii\advance\labelwidth-\labelsep
       \topsep\z@ \parsep\parskip \itemsep\z@}
   \partopsep=\z@
}{\par}
\makeatother

\newcommand{\mystar}{{\fontfamily{lmr}\selectfont$\star$}}

%\makeatletter
%\renewcommand{\@chapapp}{}% Not necessary...
%\newenvironment{chapquote}[2][2em]
%  {\setlength{\@tempdima}{#1}%
%   \def\chapquote@author{#2}%
%   \parshape 1 \@tempdima \dimexpr\textwidth-2\@tempdima\relax%
%   \itshape}
%  {\par\scriptsize\hfill-- \chapquote@author\hspace*{\@tempdima}\par\bigskip}
%\makeatother

%\newcommand\Chapter[2]{\chapter
%  [#1\hfil\hbox{}\protect\linebreak{\itshape#1}]%
%  {#1\\[2ex]\Large\itshape#2}%
%}

\begin{document}

%!TEX root=./LIVRO.tex
\hyphenation{so-cie-da-de vio-len-ta}
\chapter[Introdução, por Carlo Romani]{Introdução}
\hedramarkboth{Introdução}{Carlo Romani}



\epigraph{A dúvida reina no espírito dos homens, pois nossa civilização treme em
suas bases. As instituições atuais não mais inspiram confiança e os
mais inteligentes compreendem que a industrialização capitalista vai
contra os próprios objetivos que diz perseguir.}


\section*{sobre individualismo e revolução social}

Este poderia ser o discurso de alguma liderança do bem
comportado Greenpeace, ou a fala de algum anarquista ativista de
ecologia social do Earth Fire, ou ainda o desabafo frustrado de um
ex"-candidato à presidência dos \textsc{eua} em seu documentário sobre os
impactos ao ambiente causados pelo capitalismo contemporâneo. Talvez,
inclusive, já que quase todos se tornaram “ambientalistas” de última
hora, seja a conclusão filosófica encontrada no último relatório 
da \textsc{onu}
sobre as mudanças climáticas globais. Nem precisaríamos ir tão longe:
hoje em dia, qualquer pessoa medianamente inteligente já
compreende a dimensão da catástrofe que se aproxima.

Mas, apesar de toda a sua atualidade, essas palavras sobre o
labirinto em que a modernidade capitalista acabou por jogar a vida
humana e todas as demais formas de vida sobre o planeta foram o mote
inicial com que a ativista anarquista Emma Goldman abriu seu texto “O
indivíduo, a sociedade e o Estado”, escrito pouco 
antes do início da \textsc{ii}
Guerra Mundial e publicado nesta presente edição. E poucos sabem disto,
pois a riqueza, a intensidade e a atualidade das reflexões de
Emma Goldman sobre o ser humano e suas relações em sociedade são
praticamente desconhecidas do público brasileiro. 

Antes de tudo – me parece –, pelo fato de ela ser uma mulher, com a
agravante, ainda, de ser uma anarquista. Substantivo e adjetivo que
durante muitos anos se complementariam como garantia certa para o
confinamento intelectual e o abandono editorial a que ficaram
relegadas, seja pelo pensamento dominante, seja até por alguns dos
próprios companheiros de luta. Mas não somente essa – o que já seria
muito – me parece ter sido a razão do pouco caso com que foi tratada Emma
Goldman no Brasil. Penso que o fato de ela não ter seguido uma
trajetória formal de educação e não ter alcançado a universidade
acabou por se tornar um dos motivos de certo menosprezo pela sua
produção quando comparada à de outros autores anarquistas da mesma
época. Emma se constituiu em uma livre-pensadora através de sua própria
experiência de vida como operária, ativista de várias causas e
perseguida por diferentes regimes políticos, construindo suas ideias e
seu saber dentro dos círculos anarquistas dos quais participou e
foi organizadora. A escola oficial, tida por ela como “doutrinadora dos
espíritos servis”, quase nada influenciou em sua formação intelectual.

De origem judaica, nasceu em 1869 na Lituânia, estado então sob o
domínio do Império Russo. Anos depois, quando sua família foi vítima de
violenta perseguição antijudaica e teve seus bens confiscados, emigrou
para São Petersburgo. Na vizinha cidade báltica frequentou os bancos
escolares somente até completar 13 anos de idade, quando teve de
deixar a rígida disciplina russa para entrar na ainda mais rigorosa
disciplina do trabalho fabril. Conheceu no ambiente de fábrica
anarquistas de orientação individualista que pregavam a morte de todos
os tiranos e o uso da violência de classe como instrumento de
transformação social, no mais puro sentimento vingador eslavo herdeiro
de Bakunin. Espírito rebelde e irreconciliável, enfrentou o pai e, aos 
17 anos, emigrou para os \textsc{eua}, onde foi viver com a irmã mais
velha em um bairro pobre no interior do estado de Nova York. Para a
adolescente recém"-chegada, a imigração significou ainda mais
sofrimento: 14 horas de trabalho duro como costureira em ambientes
insalubres, pequenas oficinas onde as operárias eram exploradas,
apelidadas de \textit{sweatshops}, numa referência à umidade existente nesses
lugares. Acompanhou os acontecimentos do tumulto de Haymarket Riot
durante a greve geral de Chicago, cujo desenrolar levou sete
trabalhadores à condenação à pena de morte, exclusivamente pelo
“delito hediondo de serem anarquistas”, como sentenciou o juiz. 

A vida de exploração em que vivia, o evento marcante de Chicago, a
perseguição aos que não aceitavam a escravidão e o inconformismo de seu
caráter, transformaram"-na em uma convicta e atuante anarquista. Com
20 anos de idade mudou"-se para a cidade de Nova York, onde,
inicialmente, frequentou os círculos de anarquistas defensores da
violência como meio de transformação social. Conheceu Alexander
Berkman, que será seu companheiro daí em diante. Durante catorze anos 
seguidos Emma lutou pela libertação de Berkman da penitenciária, após
este ser condenado pela tentativa de assassinato de um gerente de
fábrica. Para os anarquistas, fora um ato de vingança, pois o
feitor teria ordenado a invasão policial que causou a morte de
trabalhadores inocentes que ocupavam a fábrica durante uma greve. Aos
poucos, a insurgência violenta presente na juventude foi cedendo espaço
para a filosofia do comunismo libertário, e a leitura da obra de
Kropotkin exercerá influência fundamental na construção de suas ideias 
e nos escritos que deixará para sempre como parte da filosofia política
anarquista.

Nesta breve biografia busquei mostrar como a constituição do pensamento
de Emma Goldman se fez totalmente à revelia do sistema oficial de
ensino. Sua educação não formal, autodidata e envolvendo os
companheiros de luta nos grêmios e sindicatos foi uma das principais
características de todo o anarquismo daquela época. Os círculos sociais
libertários, os grêmios operários mantidos pelos anarquistas e pelos
sindicalistas, as escolas e as bibliotecas por eles sustentadas
fundaram as bases para a construção de um movimento social forte, tanto em
suas ações diretas de luta econômica e política quanto na
formação cultural e intelectual de seus ativistas e simpatizantes. A
rede de círculos, grêmios e escolas garantia aos participantes do
movimento uma sociabilidade libertária, onde podiam trocar experiências
de vida bastante diferentes daquelas oferecidas pelo mundo burguês ou,
até mesmo, das condutas existentes em outros segmentos operários. 

Essa integração de diferentes círculos sociais articulados em rede, base
do projeto federativo do anarquismo, permitiu aos ativistas 
organizarem suas lutas, enquanto trabalhadores, para enfrentar o
Estado e o Capital. Permitiu"-lhes, também, criar as condições para
imaginar e pôr em prática um novo ideário de vida, uma filosofia
alternativa ao modelo dominante hierárquico e padronizado,
continuadamente repetido aos mais jovens pelo ensino oficial. Essa rede
cultural própria garantiu em muitos países, inclusive no Brasil das
primeiras décadas do século \textsc{xx}, a vitalidade e a força histórica do
pensamento e da prática anarquista. Emma Goldman foi uma das grandes
mulheres protagonistas da história dessa cultura alternativa.

Seu nome e sua obra tornaram"-se internacionalmente conhecidos e
ultrapassaram o público de simpatizantes libertários, sendo divulgados
principalmente pelo movimento feminista. No Brasil, contudo, o nome de
Emma manteve"-se praticamente desconhecido do público até a década de
1960, quando seus escritos sobre a emancipação feminina e seus artigos
em defesa das individualidades humanas foram difundidos no país durante
a passagem de Julian Beck e do grupo Living Theater pelo país. A 
maioria desses textos foi escrita no começo do século \textsc{xx}, já em sua
fase de maturidade intelectual, no período estadunidense de sua vida, e
publicada na revista \textit{Mother Earth}, por ela criada em 1906. Opiniões
bombásticas sobre a mulher, como “é apenas uma questão de grau se ela se
vende a um único homem, dentro ou fora do casamento, ou a vários”, 
falar abertamente sobre a necessidade que a mulher tem de “aplacar
seus desejos mais intensos” e sobre o absurdo que é uma mulher ter de
“abster"-se da experiência sexual” para seguir padrões de
comportamento sociais ditados por uma sociedade conservadora e
machista, tudo isso causou escândalo em uma América do Norte majoritariamente
colonizada pelo ascetismo puritano.

Se no Brasil, que teve uma ativista como Maria Lacerda de Moura, a 
divulgação e o estudo do feminismo anarquista demorou a ser realizado,
nos \textsc{eua} ele goza de ampla repercussão, e Emma Goldman, reputada como a
maior radical feminista que passou pelo país, é fruto desse interesse
que ultrapassa os limites do próprio anarquismo. Durante sua passagem
como docente em Connecticut, a historiadora Margareth Rago, uma das
pioneiras na pesquisa sobre gênero na universidade brasileira e autora
da biografia da libertária italiana Luce Fabbri, mostrou"-se
impressionada com a quantidade de trabalhos sobre a militante russa e a
disponibilidade de acesso às suas obras nas bibliotecas
norte"-americanas. Atualmente, há em andamento em
Berkeley, Califórnia, um projeto de levantamento e reedição de todos os seus
textos. 

Porém, a questão da mulher e de sua emancipação é apenas a porta de
entrada para o universo libertário de Emma. \textit{Mother Earth} foi, durante
os dez anos de sua publicação, o veículo pelo qual ela discorreu sobre
todas as microlutas de caráter tipicamente libertário. Para ela, o
ideal anarquista não é somente um fim a ser alcançado, é uma prática
cotidiana e progressiva a ser realizada dentro dos próprios espaços
existentes e abertos pela sociedade, transformando"-a continuamente,
como veremos adiante dentro de sua concepção libertária de “revolução”.
A libertação do indivíduo de suas opressões cotidianas, o
individualismo em contraste com a uniformidade do comportamento, o
antimilitarismo e a oposição sistemática à guerra, a crítica ao
patriotismo, a luta contra o sistema prisional, na qual ela teve uma
ação incansável pela libertação de seu companheiro, foram os temas
dessa revista, cujo nome, não por acaso, remete"-nos a uma integração
da humanidade com o planeta infelizmente esquecida pela civilização
capitalista.

Para Goldman, o critério para se medir o grau de civilização da
humanidade é o “grau de emancipação real do indivíduo”. Vontade de
liberdade e de dignidade, ausência de castas privilegiadas, exercício
da cooperação social entre os indivíduos, são os critérios que ela
adotaria para definir uma civilização anárquica e humanista. Não por
acaso, é o \textit{Apoio Mútuo} de Kropotkin a referência mais citada em seu
ensaio sobre o indivíduo e a sociedade, no qual ela faz uma crítica
contundente de todo e qualquer tipo de Estado, por definição usurpador
e autoritário. Nesse livro clássico, o célebre anarquista russo
contrapõe-se à teoria da seleção natural de Darwin, mostrando ser a 
cooperação e não a competição entre os integrantes de uma mesma espécie
o que permite a continuidade dessa espécie ou, no caso humano, a
continuidade do grupo social. Em seu ensaio, Emma retoma o tema
do apoio a seu modo, afirmando a exaltação da individualidade humana como
a meta a ser perseguida pela sociedade. Individualismo que somente pode
ser alcançado de forma plena quando a sociedade, livre dos poderes
coercitivos do Estado, fizer da cooperação entre os indivíduos o
caminho para sua sobrevivência.

Em sua defesa do indivíduo, ela encontrou na terra prometida americana
uma tradição individualista de rebeldia e insubmissão enraizada já há
bastante tempo. Talvez resultado do encontro mítico de deserdados
europeus na nova terra com a insubmissão selvagem de índios guerreiros
de tribos como Delaware, Cheyenne e Lakota, na América do Norte
nasceram os mais intransigentes defensores das liberdades individuais e
críticos da submissão aos poderes do Estado. Em seu artigo sobre a
preparação militar, escrito pouco antes do ingresso dos \textsc{eua} na \textsc{i} Guerra
Mundial, Emma reconhece essa tradição, fazendo a defesa do que ela
chama de “princípios fundamentais dos valores americanos”. A começar
com Jefferson, o pai fundador, logo após a guerra da independência, 
para quem o melhor governo é aquele que governa o menos possível –
declaração radicalizada por Thoureau, o pai da desobediência civil, 
quase um barnabé simbólico da autêntica vida caipira, para quem o melhor
governo é aquele que não governa. Os valores fundamentais da autonomia
federativa e da democracia americana, o ideal de liberdade política e
igualdade social, percorreram o mundo levando a esperança do bom
recomeço a milhões de imigrantes oprimidos, inclusive a ela. 

Ledo engano. O lugar onde, em 1832, o francês Tocqueville viu nascer um 
povo formado na civilização e na democracia, onde ele imaginava ser
impossível o surgimento da desigualdade de condições e a emergência de
uma classe privilegiada, trinta e poucos anos depois, com o início da
explosão migratória, essa mesma América se tornou o lugar que traiu
seus princípios fundadores. Nos Estados Unidos da América de fins do
século \textsc{xix}, terra de oportunidades, nada mudou para os que dependiam da força
de seu trabalho para sobreviver. Goldman se vale da imagem de uma
escultura para retratar melhor no que essa América se transformou para
os trabalhadores: “uma mão cruel de longos e finos dedos que esmagam
sem piedade a cabeça de um imigrante, fazendo escorrer o sangue para
dele fazer dólares e embalar o imigrante de esperanças rompidas e
aspirações sufocadas”. 

Mas, retrucaria a elite nativista americana, esse é o sofrimento
necessário pelo qual deve passar o estrangeiro para se forjar como
homem livre, um verdadeiro norte"-americano: liberdade conquistada
através da luta, da guerra; recrutamento obrigatório dos jovens
imigrantes e filhos destes para os pelotões de frente de todas as
guerras ianques, a começar pela própria guerra civil. Queres ser
cidadão americano? Deves lutar e morrer pela pátria. Em 1823, James
Monroe, com sua doutrina da “América para os americanos”, já havia 
abandonado completamente os princípios fundadores de liberdade e
igualdade. E seguiram"-se as guerras: guerra de extermínio das nações
indígenas do oeste selvagem; guerra contra o México; guerra contra a Espanha; guerra contra
as Filipinas; guerra pelo controle do Panamá; seguidas intervenções na
América Central e no Caribe; e o grande passo, o ingresso em 1915 na
grande guerra europeia. Surgia, assim, a força do império americano. Em
dois ensaios complementares, um sobre patriotismo e o outro sobre
militarismo, ambos publicados nesta edição, Emma desmascara o mito
democrático americano. A campanha contra a \textsc{i} Guerra Mundial e a
participação americana nela trouxe"-lhe severa perseguição do Estado
ianque, levando"-a novamente à prisão em 1917.  Finalmente, em fins do
ano de 1919, foi deportada como estrangeira subversiva de volta ao país
natal.

O patriotismo leva ao militarismo, que leva à guerra, que fortalece o
Estado e o Capital, que mantêm a indústria de armamentos, que leva ao
aumento da violência, que leva à xenofobia, que robustece o fogo do
patriotismo, e assim seguimos em um círculo vicioso espiralado em
direção à violência que contamina o mundo contemporâneo. O império do
conforto e consumo norte"-americanos tornou"-se o triste exemplo
massificado de uma civilização narcisista, violenta e paranoica. Emma
prenunciou que o caminho seguido pelos \textsc{eua} seria similar ao dos estados
nacionais europeus, trilha que desembocou em duas guerras mundiais.
Aliás, para ela estava claro que os \textsc{eua} se tornariam um país ainda mais
militarizado do que as próprias potências europeias porque lá o Estado
se encontrava a serviço do crescimento do Capital. Os velhos
fabricantes de armamento, as famílias Du Pont, Remington, Winchester,
fazem parte do cotidiano doméstico da família média americana. Da
defesa do lar para a defesa dos capitalistas em todo o mundo o salto
foi rápido, e o Estado americano subvencionou e fortaleceu a grande
indústria da guerra. Visionária, ela estava convencida 
``de que o
militarismo tornar"-se"-á um perigo mais importante na América do que
em qualquer outro lugar no mundo, porque o capitalismo sabe corromper
aqueles que deseja destruir''.\footnote{ Ver página \pageref{dequeomilitarismo}.}

Ao mesmo tempo em que antecipou o fato de os Estados Unidos terem se
transformado no maior estado militar da história da humanidade, também
alertou inutilmente – ironia do destino – ao soldado proletário, um
pequeno tiranete, como diria Etienne da La Boétie (1530"-1563), defensor até a morte de
seus patrões algozes. A cooptação da população pobre pelo capitalismo
como forma de defesa de uma liberdade quimérica, pois inexistente, foi
outro tema que despertou a ira dos poderosos americanos. Ainda mais
revolta ao \textit{status quo} causou sua receita para combater o militarismo: o
incitamento à deserção, à desobediência civil e à não submissão à
autoridade. O ataque à figura inabalável do Exército americano não
poderia ser aceita por um Estado que tem na guerra preventiva seu
princípio de defesa. 

Para enfrentar o perigo no qual uma guerra se torna para os jovens
recrutados, Emma apelou para a solidariedade entre os trabalhadores em
todo o mundo como forma de combater o que ela entendia ser a maior das
escravidões: a submissão voluntária de um soldado em luta matando seus
próprios irmãos. Antes de tudo, e ela é contundente na afirmação, “a
guerra de classes pressupõe todas as guerras entre as nações”.  Não há
guerra entre nações, o que há é uma guerra permanente de poder de uns
sobre os outros, guerra de classe. Revela"-se aí também sua visão
classista do anarquismo. Alcançamos o momento de sua trajetória de vida
em que a libertária russa, notória por seu individualismo,
aproxima"-se definitivamente do comunismo libertário, das opiniões de
Kropotkin e de Malatesta, fundamentos que marcarão a maior parte de 
seus escritos daí em diante. Foi a violenta crítica ao militarismo
americano e, por extensão, ao espírito de senhor da guerra do império,
com o incitamento à deserção e à ação direta, a causa central para sua
deportação definitiva da América do Norte.

 

A segunda parte desta edição é dedicada a dois escritos sobre a
revolução social e o comunismo soviético. Uma retomada crítica dessa
discussão mostra"-se absolutamente pertinente ainda, quando
alguns saudosistas ainda insistem em comemorar os 90 anos da Revolução
de 1917 em vez de chorar o desperdício inútil de milhões de vidas
humanas na antiga \textsc{urss}. O primeiro ensaio trata do fracasso da
revolução russa e foi publicado em 1923, no calor da hora, como
posfácio do livro \textit{My further disillusionment in Russia}. Entende"-se o
título, pois Emma, que havia sido forçada a emigrar em 1886, ao
retornar esperançosa à Rússia revolucionária em 1919 junto com seu
companheiro Alexander Berkman, tornou"-se testemunha participante da
condução pragmática dos destinos da revolução pelo bolchevismo do
Partido Comunista liderado por Lenin, desiludindo"-se pela segunda
vez.

A análise sócio"-econômica etapista da revolução russa entendeu como
correta a condução dada a ela pelos bolchevistas. Inicialmente a Rússia
deveria se desenvolver industrialmente para construir as condições
históricas – sociais e econômicas – necessárias para, num segundo
momento, alcançar o comunismo. Em outras palavras, na teoria evolutiva
de Marx, primeiro é preciso um capitalismo industrial para depois haver
o comunismo – mesmo porque ele menosprezou todas as civilizações
protocomunistas que não tiveram como único objetivo existencial o
desenvolvimento tecnológico e econômico, reduzindo"-as ao termo
conceitual de povos pré"-políticos. A partir de 1921, definitivamente,
essa via de mão única marxista foi arrebatada pelo Estado russo, não
pela revolução, como frisa e muito bem diferencia Emma Goldman.

Na ausência de um liberalismo clássico, do espírito empreendedor, da
livre-iniciativa burguesa como instrumento para se atingir níveis
econômicos mais elevados, o positivismo de esquerda encontrou no Estado
soviético o papel de construtor do capitalismo. Teria sido bastante
coerente se socialistas científicos e membros do Partido Comunista
abdicassem do nome comunismo e reconhecessem a nova política econômica
implantada na Rússia leninista apenas como um capitalismo de Estado,
expressão usada por Emma nesse texto. Teria havido menos confusão.
Contudo, o ocultamento das informações, o silenciamento da memória e a
manipulação das ideias se constituíram na marca registrada dos
bolchevistas. Infelizmente, a deliberada deturpação posta em prática na
ideia de comunismo, um nome que se tornou sinônimo de ditadura
burocrática, a confusão criada entre os termos de socialização e
estatização dos bens e dos meios de produção, funcionou como um freio
na luta de emancipação dos trabalhadores, que se tornariam bem
comportados consumidores de classe média nos países mais desenvolvidos.


Nunca, em nenhum outro país, o sistema taylorista de controle da 
produção e do operário foi adotado de forma tão obsessiva e opressiva
como nas corporações do Estado soviético. Em decorrência disso,
aumentou ainda mais a divisão social do trabalho, contrariando os
próprios objetivos do comunismo marxista, e a inevitável separação
hierárquica das funções produtivas colocou os técnicos e gestores do
antigo regime nos patamares mais elevados das novas classes sociais
soviéticas; subordinados a eles, todos os demais trabalhadores,
camponeses e operários, hierarquizados social e economicamente segundo
suas aptidões e divididos em até 23 faixas salariais diferentes. 
Planejamento e a racionalidade da organização capitalista a serviço da
burocracia e, esta, da ideologia.

Acima de todos, os líderes do partido. A política já não é mais a
expressão dos conflitos da sociedade, mas a expressão das divisões
mesquinhas internas e dos conchavos pelo poder dentro da estrutura do
partido único, o \textsc{pcus}, o mais numeroso do planeta. Saiu escorraçada uma
antiga classe dominante meio nobre, meio burguesa, e em seu lugar
entrou outra classe dirigente. Para sua profunda desilusão, Emma
presenciou a emergência dessa outra classe e assistiu, impotente, como
ela escreveu, “a acumulação das riquezas da antiga burguesia nas mãos
da nova burocracia soviética, as provocações permanentes contra aqueles
cujo único crime era seu antigo \textit{status} social, tudo isso foi o
resultado da ‘expropriação dos expropriadores’”. Daí o título
esclarecedor do ensaio seguinte: ``O comunismo não existe na Rússia''.

Em suas críticas sobre os descaminhos da Revolução Russa de 1917, Emma
inverterá a clássica análise de Marx e de seus seguidores, para quem a
revolução comunista somente poderia dar certo naquelas regiões do
planeta onde o desenvolvimento industrial das forças produtivas
provocasse o acirramento do conflito social e a emergência de uma
“consciência de classe”. O que poderia ter se desenhado como uma ampla
revolução social – e, para Emma, o povo russo estava propenso a esse
acontecimento – naufragou no autoritarismo coercitivo e na centralização
do poder nas mãos da ditadura da “maioria”, na verdade uma minoria
violenta de astutos que soube manipular as decisões partidárias em
benefício pessoal e de seus grupos de apoio. Nada mais distante de
uma visão libertária do comunismo. 

Mas esse deveria ter sido o destino inexorável de toda a luta do povo
russo?

Desde a revolução derrotada de 1905, a ideia do soviete como célula
nuclear da construção ascendente da nova sociedade já era de
conhecimento e fazia parte da realidade de grande parte dos
trabalhadores russos. Quando eclodiu a primeira revolução, em fevereiro
de 1917, a população russa abraçou o \textit{slogan} “todo poder aos sovietes” e
participou ativamente da agitação revolucionária. Nos meses de junho e
julho, as palavras de ordem “terra aos camponeses” e “fábrica aos
operários” foram postas em prática pela população russa sob a forma de
ação direta. Nas cidades, as fábricas foram ocupadas pelos operários.
No campo, a expropriação dos proprietários rurais ocorreu de forma
direta com grupos armados de camponeses enfrentando as milícias
particulares. A participação dos grupos anarquistas organizados por
Nestor Makhno em defesa da revolução foi fundamental para seu sucesso 
em terras ucranianas. 

A ação, a prática, sobrepujou a teoria. A onda revolucionária,
espontânea e popular, ocorreu num curto espaço de tempo desde o início
do processo revolucionário e seguiu"-se até a tomada definitiva do
Kremlin. Essa consciência de classe e de seu poder não ocorreu num
lugar onde o desenvolvimento industrial e a organização sindical estavam
mais avançados, como previa Marx. Ocorreu na atrasada Rússia agrária,
numa população secularmente submetida ao tirânico regime dos czares.
Para Emma, esses são sinais evidentes da clara “aptidão do povo russo
para a revolução social”. E quais teriam sido as causas que permitiram
essa aptidão?

Antes de tudo, a população russa estava acostumada a ondas
revolucionárias anteriores; encontrava"-se presente na sociedade um
sentimento forte contra o czar, uma revolta contida, característica de
populações ainda sob o domínio do antigo regime. Como a monarquia
permaneceu quase que absolutista, a população e o exercício da política
não foram corrompidos nem sofreram as influências enganosas da
“ideologia das liberdades democráticas e do governo a serviço do povo”.
Com essas palavras, Emma procura mostrar que os regimes
constitucionalistas resultantes da expansão das revoluções no século
\textsc{xix} serviram como amortecedores das reivindicações e das vontades
populares mais autênticas e praticamente impediram a revolução
socialista de vingar nesses países europeus. A essa corrupção
social"-democrata dentro do capitalismo ela deu o nome de “espertezas
destrutivas da pseudo"-civilização”. Na Rússia czarista não. A
população tiranicamente explorada conservou o sentido primitivo da
justiça.

Tal análise histórica do advento da Revolução Russa é
diametralmente oposta às análises marxistas. Lá onde esses últimos
encontraram falta de consciência política, fraqueza teórica e
subdesenvolvimento econômico, Emma Goldman encontrou no robusto
proletariado russo um senso agudo de justiça popular em busca do bem
coletivo, ou seja, vontade de fazer socialismo. A inexistência de um
governo liberal, de uma falsa democracia com retoque social, garantiram
ao povo a continuidade de seu espírito franco e até ingênuo, fato que
lhe fez brotar o germe de raiva necessário contra o poder tirânico e
aristocrático, elemento vital para a eclosão de uma revolução social. 

O que Emma Goldman escreve em seu texto sobre a revolução social é que
nessa Rússia de 1917 existiam sim as condições históricas para o
sucesso revolucionário, principalmente porque havia uma enorme vontade
popular. Fundamental para que o espírito de transformação seguisse
adiante, esse desejo incontido do povo russo não se encontrava
metafisicamente solto no ar, pois estava escorado por um sistema de
organização dos trabalhadores através de sindicatos e de um sistema de
produção e distribuição econômica baseado nas cooperativas que ligavam
o país em forma de rede. A administração dessa rede comunitária
era facilitada pela existência dos sovietes, experiência
histórica que já era parte integrante da administração russa e que se
proliferou durante toda a revolução.

Uma libertária comunista como Emma em nenhum momento colocou a origem de
classe do indivíduo como elemento definidor do bem a ser valorizado e
do mal a ser erradicado, meta da futura política bolchevista. Por isso,
ela defendeu a participação e a importância de toda a população no
processo revolucionário, cada qual envolvido com suas aptidões, com
suas qualidades pessoais, uma soma de valores compondo um todo coletivo,
e não habilidades individuais a serem valorizadas de forma
diferenciada. Assim, ela entendeu que, para a continuidade da
revolução, deveria ter sido fundamental a participação daquela parcela
da \textit{intelligentsia}, aqueles intelectuais russos que tradicionalmente não
se encontravam presos aos poderes aristocrático"-burgueses nem
diferenciavam a população com critérios baseados no nome de origem ou
na quantidade de dinheiro sob controle. Em suas palavras, “o sucesso da
revolução dependia da extensão mais ampla possível do gênio criativo do
povo, da colaboração entre os intelectuais e o proletariado manual”.

E foi este último passo que não ocorreu. Com a vitória da revolução e o
controle do antigo Estado czarista pelo partido da maioria, os
bolchevistas iniciaram, já em meados de 1918 e efetivamente de
1919 em diante, ou seja, justamente na época da chegada de Emma à União
Soviética, uma contínua perseguição a todos aqueles
que não concordavam, numa perspectiva comunista, com o processo de
centralização política a que a administração dos sovietes
paulatinamente foi sendo submetida, quebrando suas características
originalmente libertárias. Seja para solucionar pragmaticamente a falta
de consenso em algumas decisões, seja sob o pretexto de defesa da
revolução contra o inimigo burguês, ou ainda por puro revanchismo, a
emergente ditadura do proletariado, ou melhor, o partido único, foi se
impondo de forma coercitiva, através do medo e do terror, sobre os
demais discordantes, literalmente matando as dissidências e
transformando o sonho da revolução comunista no pesadelo de um Estado
autoritário como jamais se estabelecera até então. 

Por isso, a crítica anárquica é germinal em relação à existência do
Estado, o primeiro e máximo poder a ser enfrentado para uma mudança
radical de valores humanos que seja voltada para a emancipação e a
liberdade. Emma Goldman, testemunha ocular e participante da Revolução
Russa, como pensadora libertária concluiu que revolução social e
manutenção do Estado são ações absolutamente incompatíveis.

Os métodos da revolução são inspirados pelo próprio espírito da
revolução: a emancipação de todas as forças opressivas e limitadoras,
quer dizer, os princípios libertários. Os métodos do Estado, ao
contrário – do Estado bolchevique ou de qualquer governo –, são fundados
na coerção, que progressivamente se transforma necessariamente em
violência, opressão e terror sistemáticos.

É claro que a manifestação pública dessas opiniões na União
Soviética leninista lhe trouxe sérios problemas, e Emma e Berkman
entenderam ser necessária e urgente sua saída do país, o que ocorreria
em 1921, tendo a Inglaterra como destino. Essa passagem de Emma Goldman
pela \textsc{urss} foi registrada no cinema pelo filme \textit{Reds} (dirigido por Warren
Beatty), baseado na vida do jornalista norte"-americano e militante
comunista John Reed, que cumpriu um papel decisivo para a sobrevivência
de Emma e para que sua leitura libertária da Revolução Russa pudesse
chegar até nós.

Fechando esta presente edição, temos um artigo publicado em 1938 que
coroa a análise crítica do reino de opressão e terror montado na antiga
\textsc{urss}. Emma não poupou o comandante do Exército Vermelho pela
responsabilidade no conhecido massacre de Kronstadt, na Ucrânia, em
1921. Os marinheiros amotinados na base naval do mar Báltico defendiam
o lema revolucionário “todo poder ao soviete”, pois de fato lá o
praticavam. A insubmissão à autoridade central do \textsc{pc} era insuportável
para a burocracia dirigente que se instalava. O Exército soviético,
criado e dirigido por Leon Trotski, sufocou a rebelião matando mais de 
mil marinheiros sob a acusação de serem “pequeno"-burgueses
contra"-revolucionários”. O recado era claro: eis o que aconteceria a
quem se opusesse à vontade do Partido. Emma ainda se encontrava na \textsc{urss}
na época do ocorrido, fato que foi a gota d’água para sua “desilusão”. Na
ocasião da publicação desse artigo, Trotski, exilado no México, amante
de Frida Kahlo, circulava pelos ambientes de esquerda criticando o
stalinismo e os descaminhos da revolução, intitulando"-se o verdadeiro
revolucionário e criando uma legião de novos seguidores persistentes
até hoje em dia: os trotskistas. Com sua pena ácida, Emma fez
lembrá"-los de que seu líder agia como lobo em pele de cordeiro e que
entre ele e Stalin não havia diferença alguma. Por ironia do destino,
dois anos depois Trotski seria assassinado. Se por traição política ou
por vingança amorosa, nunca saberemos.

Assim, esta coletânea de artigos de Emma Goldman vem mostrar para o
público brasileiro, além da sua já conhecida e intransigente defesa do
individualismo, da unicidade de cada ser humano como elemento
constitutivo da sociedade, um lado menos conhecido: a defesa de uma
forma de anarquismo que é também comunista e a crítica contundente a
dois modelos políticos e sociais que, embora diferentes em sua forma,
foram parecidos em
seus objetivos de dominação. Ambos os modelos acabados pelos quais a
modernidade se manifestou no século passado tornaram"-se, por razões
diferentes, sufocadores da expressão plena do indivíduo. 

A crítica da pseudoliberdade da democracia capitalista americana e a
crítica da falsa igualdade do comunismo soviético mostra"-se
absolutamente atual neste início de século, quando parece que perdemos
a crença na possibilidade de se construir sociedades baseadas em vidas
livres e dignas para todos os habitantes do planeta. Alguns dirão: mera
utopia, pois a ordem da natureza é hierárquica. Emma retrucaria: a
emancipação é insubmissão, anarquia, desejo inalienável do indivíduo.
Um dos impérios desmascarados em seus textos já desapareceu. O outro
persiste forte, quase onipresente, pairando sobre a Terra como uma
polícia planetária disposta a intimidar os desviantes, pois é através
da força e da coerção que as ordens são mantidas. 

Agora, mais do que nunca, quando o risco de uma dominação coletiva das
mentes transformadas em corpos dóceis consumidores de prazeres fáceis
se esparrama mundo afora, quando o ser humano se enclausura num
padronizado individualismo consumista, a voz anárquica de luta efetiva
pela liberdade dos indivíduos deve se fazer sempre presente. Mais um
motivo para que as palavras de Emma Goldman sejam lembradas e repetidas.

\begin{resumopage} 


\item[Emma Goldman] (Kovno [atual Kaunas], 1869--Toronto,
1940). Revolucionária anarquista de origem russa, emigrou para Rochester,
Estados Unidos, em 1886.  Como grande parte dos emigrantes do leste europeu,
trabalha em uma fábrica de roupas, onde toma contato com as doutrinas socialista
e anarquista. Em 1899, muda-se para Nova York e conhece Alexander Berkman,
anarquista condenado em 1892 pela tentativa de assassinato do industrial Henry
Clay Frick. Em 1901, Leon Czolgosz assassina o presidente William McKinley, e
alega ter sido inspirado pelos ensinamentos de Emma. Ativista dos direitos da
mulher, une-se a Margaret Sanger na luta pelo controle de natalidade, dando
palestras por todo os \textsc{eua}. Em 1906, com a soltura de Berkman, retoma as
atividades em conjunto com seu companheiro e funda o periódico \textit{Mother
Earth} (1906-1917). Em 1910, publica  \textit{Anarchism and Other Essays}, dois
anos após ter a cidadania americana revogada pelo governo.  Deportada dos
\textsc{eua} em 1919, juntamente com Berkman, alcança a Rússia e lá permanece
até a revolta de Kronstadt (1921). Decepcionada com a onda de perseguições e a
repressão que se seguiram à Revolução Russa, parte para a Europa ocidental no
mesmo ano, e em 1923 publica \textit{My Disillusionment in Russia}, crítica
severa ao sistema soviético. Perseguida pelos agentes do \textsc{fbi} grande
parte de sua vida, foi presa seis vezes entre 1893 e 1921, acusada de incitar
rebeliões, preconizar o controle de natalidade e opor-se à Primeira Guerra
Mundial e ao alistamento militar, entre outras acusações. Em 1931, publica sua
autobiografia \textit{Living My Life}, e mantém intensa atividade como
palestrante, residindo nos principais países da Europa. Durante a Guerra Civil
Espanhola (1936) apoiou ativamente os anarquistas na luta contra o fascismo.
Faleceu em Toronto, Canadá, em 1940.  

\item[O indivíduo, a sociedade e o Estado] foi publicado pelo Free Society
Forum, Chicago, Illinois, em 1940. Defesa intransigente da liberdade do
indíviduo e crítica ferrenha à submissão ao poder estatal, esse texto, inspirado
em Kropotkin e Malatesta, já antecipava muitas das questões fundamentais do
século \textsc{xx}, como a militarização estratégica dos \textsc{eua}. A
presente edição conta ainda com o posfácio do livro \textit{My disillusionment
in Russia} (1923), e \textit{O comunismo não existe na Rússia}, artigo publicado
em 1935, no qual Emma critica o autoritarismo e a centralização de poder dos
sovietes.  

\item[Plínio Augusto Coêlho]  fundou em 1984 a Novos Tempos Editora, em
Brasília, dedicada à publicação de obras libertárias. Em 1989, transfere-se para
São Paulo, onde cria a Editora Imaginário, mantendo a mesma linha de
publicações. É idealizador e co-fundador do \textsc{iel} (Instituto de Estudos
Libertários).  

\item[Carlo Romani] é doutor em História Cultural pela Universidade de Campinas
(Unicamp) e pesquisador vinculado ao \textsc{nupaub/usp} (Antropologia Caiçara).
Publicou a biografia histórica \textit{Oreste Ristori: Uma aventura anarquista}
(Annablume, 2002), e atualmente ensina História Contemporânea na Universidade
Federal do Ceará (\textsc{ufc}).  

\item[Série Estudos Libertários:] as obras reunidas nesta série, em sua maioria
inéditas em língua portuguesa, foram escritas pelos expoentes da corrente
libertária do socialismo.  Importante base teórica para a interpretação das
grandes lutas sociais travadas desde a segunda metade do século \textsc{xix},
explicitam a evolução da idéia e da experimentação libertárias nos campos
político, social e econômico, à luz dos princípios federalista e
autogestionário.          

\end{resumopage}

%\input{\printindex}


\printindex

\end{document}
